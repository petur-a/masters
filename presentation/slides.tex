\documentclass{beamer}

\usepackage{xspace}
\usepackage{stmaryrd}
\usepackage{amsmath}
\usepackage{lstrfun}

\def\rfunc{\ensuremath{\mathsf{CoreFun}}\xspace}
\def\rfun{\ensuremath{\mathsf{RFun}}\xspace}

\lstset{language=RFun}

% Environments
\newcommand{\cod}{\text{cod}}
\newcommand{\dom}{\text{dom}}
\newcommand{\Hom}{\text{Hom}}

\newcommand{\leaves}{\text{leaves}}
\newcommand{\sfmp}{\text{staticFMP}}
\newcommand{\PLVal}{\text{PLVal}}

\def\vdashe{\vdash}
\def\vdashd{\vdash\!\!\!\vdash}
\newcommand{\con}{\Sigma;\Gamma\vdash}
\newcommand{\conp}{\Sigma;\Gamma'\vdash}
\newcommand{\conpp}{\Sigma;\Gamma''\vdash}
\newcommand{\conemp}{\Sigma;\emptyset\vdash}
\newcommand{\concup}{\Sigma;\Gamma\cup\Gamma'\vdash}
\newcommand{\concupp}{\Sigma;\Gamma\cup\Gamma''\vdash}
\newcommand{\pcon}{p \vdash}
\newcommand{\storeinv}{p;\sigma\vdash^{-1}}
\newcommand{\emptyinv}{p;\emptyset\vdash^{-1}}

\newcommand{\dcon}{\Delta\vdash}

% Constructs
\newcommand{\inl}[1]{\textbf{inl}(#1)}
\newcommand{\inr}[1]{\textbf{inr}(#1)}

\newcommand{\lett}[3]{\textbf{let } #1 = #2 \textbf{ in } #3}
\newcommand{\caseof}[5]{\textbf{case } #1 \textbf{ of } #2 \Rightarrow #3, #4 \Rightarrow #5}
\newcommand{\caseofu}[5]{\textbf{case } #1 \textbf{ of } #2 \Rightarrow #3, #4 \Rightarrow #5 \textbf{ unsafe }}
\newcommand{\caseofs}[6]{\textbf{case } #1 \textbf{ of } #2 \Rightarrow #3, #4 \Rightarrow #5 \textbf{ safe } #6}
\newcommand{\smpcase}[2]{\textbf{case } #1 \textbf{ of } #2}

\newcommand{\roll}[2]{\textbf{roll } [#1]\ #2}
\newcommand{\unroll}[2]{\textbf{unroll } [#1]\ #2}

\newcommand{\abort}[2]{\textbf{abort}_#1\ #2}

\newcommand{\class}[2]{\textbf{class } #1 \textbf{ where } #2}
\newcommand{\instance}[2]{\textbf{instance } #1 \textbf{ where } #2}

\newcommand{\within}[2]{\textbf{within } #1 : #2 \textbf{ end}}

\newcommand{\defeq}{\stackrel{\>\text{def}}{=}\>}

\definecolor{kugreen}{RGB}{50,93,61}
\definecolor{kugreenlys}{RGB}{132,158,139}
\definecolor{kugreenlyslys}{RGB}{173,190,177}
\definecolor{kugreenlyslyslys}{RGB}{214,223,216}

\usetheme[width=2cm]{PaloAlto}

\setbeamercovered{transparent}
\setbeamertemplate{blocks}[rounded][shadow=true]

%gets rid of bottom navigation symbols
\setbeamertemplate{navigation symbols}{}

\usecolortheme[named=kugreen]{structure}
\useinnertheme{circles}
\usefonttheme[onlymath]{serif}

% \logo{\includegraphics[width=1.5cm]{billeder/logo}}

\title{Design of a Reversible Functional Language}
\subtitle{And Its Type System}
\author{Petur Andrias Højgaard Jacobsen}
\institute
{
  Institute of Computer Science \\
  University of Copenhagen
}
\date{\today}
\subject{Computer Science}

\begin{document}

\frame{\titlepage}

\frame{\frametitle{Table of Contents}

  \begin{enumerate}

    \item Introduction.

    \item Presenting the \rfunc language.

    \item How do we run backwards?

    \item Syntactic sugar.

  \end{enumerate}

}

\section{Introduction}

\frame{\frametitle{Introduction}

  \begin{itemize}

    \item Reversible computing is the study of computational models in which
      individual computation steps can be uniquely and unambiguously inverted.

    \item Originally studied for energy conservation, we wish to use it as an
      property of programming.

    \item Reversible languages include Janus, Pi, Theseus and RFun.

    \item We present a reversible functional programming language \rfunc,
      inspired by \rfun.

  \end{itemize}

}

\frame{\frametitle{Why is a Type System Useful?}

  \begin{itemize}

    \item Well-typed programs cannot go wrong.

    \item Proper resource usage.

    \item Introduction of static information.

    \item Other?

  \end{itemize}

}

\section{Language}

\frame{\frametitle{\rfunc Introduction}

  \rfunc features:

  \begin{itemize}

    \item A relevant type system.

    \item Ancillae parameters.

    \item Rank-1 parametric polymorphism.

    \item Recursive types through iso-recursive treatment.

  \end{itemize}

}

\frame{\frametitle{\rfunc Grammar}

\small

\begin{figure}[t]
\begin{tabular}{p{1\textwidth}}
$q ::= d^*$ \\
$d ::= f\ \alpha^*~v^+ = e$ \\
$e ::= x$ \\
$\hspace{1.9em}\mid ()$ \\
$\hspace{1.9em}\mid \inl{e}$ \\
$\hspace{1.9em}\mid \inr{e}$ \\
$\hspace{1.9em}\mid (e, e)$ \\
$\hspace{1.9em}\mid \lett{l}{e}{e}$ \\
$\hspace{1.9em}\mid \caseof{e}{\inl{x}}{e}{\inr{y}}{e}$ \\
$\hspace{1.9em}\mid f~\alpha^*~e^+$ \\
$\hspace{1.9em}\mid \roll{\tau}{e}$ \\
$\hspace{1.9em}\mid \unroll{\tau}{e}$ \\
$l ::= x$ \\
$\hspace{1.9em}\mid (x, x)$ \\
$v ::= x:\tau_a$ \\
\end{tabular}
\end{figure}

}

\begin{frame}[fragile]\frametitle{Example Program: Map}

  Below is a rather standard \texttt{map} function. Note the explicit
  \textbf{roll} and \textbf{unroll} terms.

  \begin{rfuncode}
map $\alpha$ $\beta$ ($f: \alpha \leftrightarrow \beta$) ($xs: \mu X . 1 + \alpha \times X$) =
  case unroll [$\mu X . 1 + \alpha \times X$] of
    inl(()) => roll [$\mu X . 1 + \beta \times X$] inl(()),
    inr($xs'$) => let ($x, xs''$) = $xs'$
             in let $x'$ = $f$ $x$
             in let $xs'''$ = map $\alpha$ $\beta$ $f$ $xs''$
             in roll [$\mu X . 1 + \beta \times X$] inr(($x', x'''$))
  \end{rfuncode}

\end{frame}

\frame{\frametitle{\rfunc Typing Terms}

  Typing terms. $\tau_f$ are function types, $\tau$ are primitive types and
  $\tau_a$ are ancilla types.

  \vskip -2\abovedisplayskip
  \begin{figure}[t]
  \begin{align*}
    \tau_f & ::= \tau_f \rightarrow \tau'_f
          \mid \tau \rightarrow \tau'_f
          \mid \tau \leftrightarrow \tau' \mid \forall X. \tau_f \\
    \tau & ::=
    1 \mid \tau \times \tau' \mid \tau + \tau' \mid X \mid
             \mu X.\tau \\
    \tau_a & ::= \tau \mid \tau \leftrightarrow \tau'
  \end{align*}
  \end{figure}

  Typing judgement

  \vskip -1.5\abovedisplayskip
  \begin{align*}
    \con e : \tau
  \end{align*}

}

\frame{\frametitle{Ensuring Relevance}

\small

  Two rule for variables, depending on which context they appear in.

  \begin{align*}
    \textsc{T-Var1: }
      \dfrac
        {}
        {\conemp x : \tau} \Sigma(x) = \tau &&
    \textsc{T-Var2: }
      \dfrac
        {}
        {\Sigma; (x \mapsto \tau) \vdashe x : \tau}
  \end{align*}

  Unit requires an empty context. We can always perform Unit elimination.

  \begin{align*}
    \textsc{T-Unit: }
      \dfrac
        {}
        {\conemp () : 1}
  \end{align*}

  \begin{align*}
    \textsc{T-Unit-Elm: }
      \dfrac
        {\begin{array}{ccc}
         \con e : 1 & \quad &
         \conp e' : \tau
         \end{array}}
        {\concup e' : \tau}
  \end{align*}

}

\frame{\frametitle{Ensuring Relevance}

\small

Introduce non-deterministic union of hypotheses when a rule has multiple
premises.

  \vskip -1.5\abovedisplayskip
  \begin{align*}
    \textsc{T-Prod: }
      \dfrac
        {\begin{array}{ccc}
         \con e_1 : \tau & \quad &
         \conp e_2 : \tau'
         \end{array}}
        {\concup (e_1, e_2) : \tau \times \tau'}
  \end{align*}

Allows contraction and exchange. A concrete example:

  \vskip -1.5\abovedisplayskip
  \begin{align*}
    \dfrac
      {\begin{array}{ccc}
          \overset{\vdots}{\emptyset; x \mapsto \tau_x, y \mapsto \tau_y \vdashe \dots} &
          \quad &
          \overset{\vdots}{\emptyset; y \mapsto \tau_y, z \mapsto \tau_z \vdashe \dots}
       \end{array}}
      {\emptyset; x \mapsto \tau_x, y \mapsto \tau_y, z \mapsto \tau_z \vdashe \dots}
  \end{align*}

}

\frame{\frametitle{Ensuring Ancillae Are Static}

  We observe that hypotheses can be put into the static context when no dynamic
  information went into constructing them.

\begin{align*}
  \textsc{T-Let1St: }
    \dfrac
      {\begin{array}{ccc}
       \Sigma; \emptyset \vdash e_1 : \tau' & \quad &
       \Sigma, x : \tau' ; \Gamma, \vdashe e_2 : \tau
       \end{array}}
      {\con \lett{x}{e_1}{e_2} : \tau}
\end{align*}

  Alternatively, We could have gone with restricted weakening.

\begin{align*}
  \textsc{T-Weakening: }
    \dfrac
      {\Sigma, x \mapsto \tau; \Gamma \vdash e : \tau'}
      {\Sigma, x \mapsto \tau; \Gamma, x \mapsto \tau \vdash e : \tau'}
\end{align*}

}

\frame{\frametitle{Evaluation}

  Big step call-by-value semantic with substitution to map values to variables.
  Evaluation judgement

  \vskip -1.5\abovedisplayskip
  \begin{align*}
    \pcon e \downarrow c
  \end{align*}

  The difficult part of ensuring reversibility is ensuring determinism of
  branching. \vspace{0.5em}

  We employ a \emph{First Match Policy}. It can be seen in the \textsc{E-CaseR}
  rule. \vspace{0.5em}

  Now, we define \emph{leaf expressions}, \emph{possible leaf values} and
  \emph{leaves}.

}

\frame{\frametitle{Leaf Expressions, Possible Leaf Values and Leaves}

  Leaf expressions are a subset of expression which may be regarded as the
  final expression. \vspace{0.5em}

  We collect the set of leaves of an expression:

  \vskip -1.5\abovedisplayskip
  \begin{align*}
    \leaves (\caseof{z}{\inl{x}}{x}{\inr{y}}{y}) = \{ x, y \}
  \end{align*}

  A unification of two expressions states if the expressions relate to one
  another. An example:

  \vskip -1.5\abovedisplayskip
  \begin{align*}
    () \rhd ()
  \end{align*}

  The set of possible leaf values is then defined as:

  \vskip -1.5\abovedisplayskip
  \begin{align*}
    \PLVal(e) = \{ e' \in \text{LExpr} \mid e'' \in \leaves(e), e' \rhd e'' \}
  \end{align*}

}

\frame{\frametitle{Backwards Determinism}

  \begin{theorem}[Contextual Backwards Determinism]

    For all open expressions $e$ with free variables $x_1, \dots, x_n$, and all
    canonical forms $v_1, \dots, v_n$ and $w_1, \dots, w_n$, if $\pcon e[v_1 /
    x_1, \dots, v_n / x_n] \downarrow c$ and $\pcon e[w_1 / x_1, \dots, w_n /
    x_n] \downarrow c$ then $v_i = w_i$  for all $1 \le i \le n$.

  \end{theorem}

  \vspace{1em}
  Normally expressed directly by the reduction relation, as in RFun. Too
  restrictive in \rfunc ($e \downarrow c$ and $e' \downarrow c \Leftrightarrow e
  = e'$ is only true in trivial languages).

}

\section{First Match Policy}

\begin{frame}[fragile]\frametitle{First Match Policy Cons}

  Unfortunately, reversibility cannot be fully determined by the type system.
  \vspace{0.5em}

  The First Match Policy is potentially inefficient! See the following plus
  function:

  \begin{rfuncode}
    plus $(n_1: \mu X . 1 + X)$ ($n_2: \mu X . 1 + X$) =
      case unroll [$\mu X . 1 + X$] $n_1$ of
        inl(()) => $n_2$,
        inr($n_1'$) => let $n_2'$ = plus $n_1'$ $n_2$
                   in roll [$\mu X . 1 + X$] inr($n_2'$)
  \end{rfuncode}

\end{frame}

\frame{\frametitle{Alternative Measures: Exit Assertion}

  We will add two things: Exit assertions and static guarantees. \vspace{0.5em}

  An exit assertion is appended to a case-expression and must evaluate to a
  Boolean value.

  \begin{itemize}

    \item Janus if-statement:

    \vskip -1.5\abovedisplayskip
    \begin{align*}
      \textbf{if } e \textbf{ then } s_1 \textbf{ else } s_2 \textbf{ fi } e_a
    \end{align*}

    \item \rfunc case-expression with exit assertion:

    \vskip -1.5\abovedisplayskip
    \begin{align*}
      \caseofs{e}{\inl{x}}{e_2}{\inr{y}}{e_3}{e_a}
    \end{align*}

  \end{itemize}

}

\frame{\frametitle{Alternative Measures: Static Guarantee}

\small

  Two methods to identify a static guarantee:

  \begin{itemize}

    \item ``Traditional'' analysis looks at the open form of the branch arms to
      observe syntactic difference

      \vskip -1.5\abovedisplayskip
      \begin{align*}
        \{ (l_2, l_3) \mid l_2 \in \leaves(e_2)
                         , l_3 \in \leaves(e_3)
                         , l_2 \rhd l_3 \}
        = \emptyset
      \end{align*}

     \item Exhaustive method looks at the domain of a function to see if any
       leaves sets have overlap.

      \vskip -1.5\abovedisplayskip
      \begin{align*}
        \leaves_l = \bigcup_{\substack{(c_1, \dots, c_n) \\ \in \dom{(f)}}} \leaves(e_2') \\
        \leaves_r = \bigcup_{\substack{(c_1, \dots, c_n) \\ \in \dom{(f)}}} \leaves(e_3')
      \end{align*}

      \vskip -1.5\abovedisplayskip
      \begin{align*}
        \{ (l_2, l_3) \mid l_2 \in \leaves_l
                         , l_3 \in \leaves_r
                         , l_2 \rhd l_3 \}
        = \emptyset
      \end{align*}

  \end{itemize}

}

\section{Going Backwards}

\frame{\frametitle{Running Backwards}

  The inverse of any function can be determined by trying all possible outputs,
  known as the Generate-And-Test method. \vspace{0.5em}

  We want it to be efficient though: Meaning, as efficient as forward
  interpretation.

}

\frame{\frametitle{Program Inversion}

  A program inverter literally writes the inverse program from the forward
  program. This program can then be evaluated as normal.

  \vskip -2\abovedisplayskip
  \begin{align*}
    \mathcal{I}_e \llbracket e \rrbracket = e'
  \end{align*}

  It is presented as a set of local inverters, one for each syntactic domain.
  For \rfunc we could define $\mathcal{I}_e, \mathcal{I}_f, \mathcal{I}_p$.
  \vspace{0.5em}

  This is easy for flow-chart languages like Janus, as constructs are
  symmetrical and have clear inverses. \vspace{0.5em}

  Not so easy for reversible functional languages.

}

\frame{\frametitle{Inverse Semantics}

  We will instead use inverse interpretation. \vspace{0.5em}

  Relationship between a function $f$ and its inverse:

  \vskip -1.5\abovedisplayskip
  \begin{align*}
    f~a_1 \dots a_n~x = y
    \Longleftrightarrow f^{-1}~a_1 \dots a_n~y = x
  \end{align*}

  Cannot describe mapping of variables to values with substitution. We instead
  ``predict'' the value of each variable in the forward direction. Inverse
  semantics judgement

  \vskip -1.5\abovedisplayskip
  \begin{align*}
    p; \sigma \vdash^{-1} e, c \leadsto \sigma'
  \end{align*}

}

\frame{\frametitle{Inverse Semantics Correctness}

  The derivation of the forward semantics and the backwards semantics are
  captured by the following lemma:

\begin{lemma}

  If $\pcon e[c / x] \downarrow c'$ (by $\mathcal{E}$) where $x$ is the only
  free variable in $e$, then $p; \emptyset \vdash^{-1} e, c' \leadsto \{x
  \mapsto c\}$ (by $\mathcal{I}$).

\end{lemma}

  We use it directly to argue the correctness of the inverse semantics:

\begin{theorem}[Correctness of Inverse Semantics]

  If $p \vdash f~c_1 \dots c_n \downarrow c$ with $p(f) = \alpha_1 \dots
  \alpha_m~x_1 \dots x_n = e$, then $\sigma(x_n) = c_n$ where $\emptyinv e[c_1
  / x_1, \dots, c_{n-1} / x_{n-1}], c \leadsto \sigma$.

\end{theorem}

}

\frame{\frametitle{Inverse Semantics Expressiveness}

  A program inverter is less expressive than the presented inverse semantics.
  Consider inverting:

\begin{align*}
  \mathcal{I}_f \llbracket f~(x:\tau) = (x, x) \rrbracket &\defeq
  f^{-1}~(x': \tau \times \tau) = \mathcal{I}_e \llbracket x' \rrbracket \\
  \mathcal{I}_e \llbracket x' \rrbracket &\defeq \lett{(x, y)}{x'}{x}
\end{align*}

  We cannot implicitly assume that $x = y$, so program is not well-typed. But
  we can run backwards by the inverse semantics.

}

\section{Transformations}

\frame{\frametitle{Transformations Background}

  \rfunc is as simple as possible. Our next motivation is to lift it into a
  higher-level language to make programming easier. \vspace{0.5em}

  Transformations should have clearly defined translations of expressions from
  a high-level syntax to the core syntax.

  \vskip -1.5\abovedisplayskip
  \begin{align*}
    e \defeq e'
  \end{align*}

  Where $e$ is an expression in the light language and $e'$ is an expression in
  the core language.

}

\frame{\frametitle{Variants}

\small

  Allows us to define new algebraic data types.

  \vskip -1.5\abovedisplayskip
  \begin{align*}
    \beta~\alpha^* = \texttt{v}_1~[\tau\alpha]^* \mid \dots \mid \texttt{v}_n~[\tau\alpha]^*
  \end{align*}

  The translation strips all data type definitions and substitutes directly
  into each expression. \vspace{0.5em}

  Translations of case-expressions over variants unroll the variant.

}

\begin{frame}[fragile]\frametitle{Variant Translation Example}

  Light Program containing a variant:

    \begin{rfuncode}
Choice = Rock | Paper | Scissors

rpsAI ($c$:Choice) =
  case $c$ of
    Rock => Paper
    Paper => Scissors
    Scissors => Rock
    \end{rfuncode}

\end{frame}

\begin{frame}[fragile]\frametitle{Variant Translation Example}

  Translated core program:

  \begin{rfuncode}
rpsAI ($c$:1 + (1 + 1)) =
  case $c$ of
    inl(()) => inr(inl(()))
    inr($w$) => case $w$ of
      inl(()) => inr(inr(()))
      inr(()) => inl(())
  \end{rfuncode}

\end{frame}

\frame{\frametitle{Type Classes}

  Type classes are famous from Haskell. They solve overloading of operations.
  A type class definition:

  \vskip -1.5\abovedisplayskip
  \begin{align*}
    \textbf{class } \kappa~\alpha \textbf{ where } [f \Rightarrow \tau_f]^+
  \end{align*}

  Functions which use some type class operator have their signature restricted:

  \vskip -1.5\abovedisplayskip
  \begin{align*}
    f~\kappa~\alpha \Rightarrow \alpha~.~(x_1:\tau_1) \dots (x_n:\tau_n) = e
  \end{align*}

  The translation erases all type class definitions and generates new top-level
  functions. \vspace{0.5em}

  And the correct top-level functions is applied instead of the generic
  function.

}

\frame{\frametitle{Type Classes}

  Can be used to define a duplication/equality operator (with variants):

  \vskip -1.5\abovedisplayskip
  \begin{align*}
    \texttt{Eq}~\alpha = \texttt{Eq} \mid \texttt{Neq}~\alpha
  \end{align*}

}

\begin{frame}[fragile]\frametitle{Type Class Translation Example}

  Light Program containing a type class:

    \begin{rfuncode}
class Default a where
  def => 1 <-> a

instance Default ($\mu X . 1 + X$) where
  def $u$ => inl(())

defPair : Default $\alpha$, Default $\beta$ => $\alpha$ $\beta$ . 1
defPair $u$ = (def $\alpha$ $u$, def $\beta$ $u$)
    \end{rfuncode}

\end{frame}

\begin{frame}[fragile]\frametitle{Type Class Translation Example}

  Translated core program:

  \begin{rfuncode}
defNat : 1
defNat $u$ = inl(())

defPairNatNat : 1
defPairNatNat $u$ = (defNat $u$, defNat $u$)
  \end{rfuncode}

\end{frame}

\frame{\frametitle{Top-level Cases}

  Functions oftentimes start with looking at the form of the input. We
  introduce a shorthand for this phenomenon.

  \vskip -1.5\abovedisplayskip
  \begin{align*}
    &\texttt{map $f$ []} = \texttt{[]} \\
    &\texttt{map $f$ $(x:xs)$} = \texttt{$f~x:$ map $f$ $xs$}
  \end{align*}

  Only allow top-level sum types and variant types. The coverage must be
  \emph{total}. \vspace{0.5em}

  Requires constructing a tree, as we allow it for multiple parameters at the
  time.

}

\begin{frame}[fragile]\frametitle{Top-level Case Translation Example}

  Light Program containing a top level clause and a list variant:

    \begin{rfuncode}
map :: $\alpha~\beta~.~(\alpha \leftrightarrow \beta) \rightarrow \texttt{List } \alpha$
map $f$ inl(()) = roll [List $\beta$] Nil
map $f$ inr($(x, xs')$) = let $x' = f~\alpha~\beta~x$
                   in let $xs''$ = map $\alpha~\beta~f~xs'$
                   in roll [List $\beta$] (Cons $x'$ $xs''$)
    \end{rfuncode}

\end{frame}

\frame{\frametitle{Guards}

  Guards are built on top of top-level clauses and require a predicate to hold
  on top of the form of the pattern expressions. \vspace{0.5em}

  We require guards to be \emph{total}. One clause should be guarded by an
  \textbf{otherwise}. \vspace{0.5em}

  They are easily translated into a chain of boolean checks, one for each
  guard, in order of definition.

}

\begin{frame}[fragile]\frametitle{Guard Translation Example}

  Light Program containing guards.

    \begin{rfuncode}
tryPred :: $\mu X . 1 + X$
tryPred x | case unroll [$\mu X . 1 + X$] $x$ of
              inl(()) => inl(()),
              inr($x'$) => inr(()) = inl(())
tryPred x | otherwise = let $x'$ = unroll [$\mu X . 1 + X$] $x$
                          in inr($x'$)
    \end{rfuncode}

\end{frame}

\begin{frame}[fragile]\frametitle{Guard Translation Example}

  Translated core program.

    \begin{rfuncode}
tryPred :: $\mu X . 1 + X$
tryPred x = case unroll [$\mu X . 1 + X$] $x$ of
              inl(()) => inl(())
              inr(()) => let $x'$ = unroll [$\mu X . 1 + X$] $x$
                         in inr($x'$)
    \end{rfuncode}

\end{frame}

\frame{\frametitle{Records}

  Records are labeled products (and dual to variants).

  \vskip -1.5\abovedisplayskip
  \begin{align*}
    \gamma = \{ l_1 :: \tau_1, \dots, l_n :: \tau_n  \}
  \end{align*}

  We require them to be total and we disallow projections. Instead, we
  introduce a special projection scope:

  \vskip -1.5\abovedisplayskip
  \begin{align*}
    \within{\gamma}{e^\rho}
  \end{align*}

  Translations of introductions of records construct an $n$-ary product
  directly. \vspace{0.5em}

  A record scope is a chain of let-expressions
  record and introduce

}

\begin{frame}[fragile]\frametitle{Record Translation Example}

  Light Program containing records.

    \begin{rfuncode}
Vector = { x: $\mu X . 1 + X$, y: $\mu X . 1 + X$, z: $\mu X . 1 + X$ }

$f~(x: \mu X . 1 + X)~(y: \mu X . 1 + X)~(z: \mu X . 1 + X)$ =
  let $v$ = Vector { x = $x$, y = $y$, z = $z$ }
  in within $v$:
       $v.x$ = roll [$\mu X . 1 + X$] inr($v.x$)
     end
    \end{rfuncode}

\end{frame}

\begin{frame}[fragile]\frametitle{Record Translation Example}

  Translated core program.

    \begin{rfuncode}
$f~(x: \mu X . 1 + X)~(y: \mu X . 1 + X)~(z: \mu X . 1 + X)$ =
  let $v$ = $(x, (y, z))$
  in let $x'$ = roll [$\mu X . 1 + X$] inr($x$)
    in $(x', (y, z))$
    \end{rfuncode}

\end{frame}


\frame{\frametitle{Misc}

  In the thesis, we presented two more transformations, not worthy of a lengthy
  discussion:

  \begin{itemize}

    \item Arbitrarily sized products.

    \item Multiple let-expressions in a row.

  \end{itemize}

  Both of these have straightforward translations by unfolding.

}

\section{What Else?}


\frame{\frametitle{Anonymous Functions and Binary Operators}

  Let us consider two translations which are not very amenable: anonymous
  functions and binary operators. \vspace{0.5em}

  \begin{itemize}

    \item Anonymous functions are tricky as we need to take them out of their
      lexical scope. What information do they use?

    \item Binary operators are tricky as we transform the right operand, but
      this does not fit well with what is expected:

    \vskip -1.5\abovedisplayskip
    \begin{align*}
      a \odot b = c && 2 + 3 = 5 \\
      a \odot^{-1} c = b && 2 - 5 = 3
    \end{align*}

  \end{itemize}

}

\frame{\frametitle{Higher Order Programming}

  Remember that we want to design a garbage-free language. \vspace{0.5em}

  Supporting general higher-order functions means variables can take on
  arbitrary functions as values. \vspace{0.5em}

  Consider the following function:

  \vskip -1.5\abovedisplayskip
  \begin{align*}
    g~(x : \tau) = \lett{f'}{\texttt{rtwice } f}{f'~x}
  \end{align*}

  How is $f'~x$ computed? The function $f'$ is unknown during inverse
  evaluation. Therefore this generates a \emph{closure}.

}

\section{Introduction$^\dagger$}

\frame{\frametitle{Introduction$^\dagger$}

  We set out to add a type system to \rfun, but ended up designing a new
  language with a relevant type system. \vspace{0.5em}

  We showed evaluation rules and backwards determinism. \vspace{0.5em}

  We showed that it is possible to inverse interpret \rfunc programs. \vspace{0.5em}

  We showed syntactic sugar over the core language to make \rfunc more modern. \vspace{0.5em}

  The work in this thesis has also been made into an article.

}

\end{document}
