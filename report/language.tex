% !TEX root = main.tex

\newcommand{\alignspace}{\vspace{-4mm}}

\chapter{Formalisation of \rfunc}\label{sec:formal}

The following section will present the formalisation of \rfunc. The language is
intended to be minimal, but it will still accommodate future extensions to a
modern style functional language. We first present a core language syntax,
which will work as the base of all formal analysis. Subsequently we present
typing rules and operational semantics over this language. The following is
built on knowledge about implementation of type systems as explained
in~\cite{Pierce:2002:TPL}.

\section{Grammar}

A program is a collection of zero or more function definitions. Each definition
must be defined over some number of input variables as constant functions are
not interesting in a reversible setting. All function definitions will in
interpretation be available though a static context.  A typing of a program is
synonymous with a typing of each function. A function is identified by a name
$f$ and takes 0 or more type parameters, and 1 or more formal parameters as
inputs. Each formal parameter $x$ is associated with a typing term $\tau$ at
the time of definition for each function, which may be one of the type
variables given as type parameter.  The grammar is given in
Fig.~\ref{fig:grammar}.

\begin{figure}[t]
\begin{tabular}{p{.5\textwidth}l}
$q ::= d^*$                                             & Program definition\\
$d ::= f\ \alpha^*~v^+ = e$                             & Function definition\\
$e ::= x$                                               & Variable name\\
$\hspace{1.9em}\mid ()$                                 & Unit term\\
$\hspace{1.9em}\mid \inl{e}$                            & Left of sum term\\
$\hspace{1.9em}\mid \inr{e}$                            & Right of sum term\\
$\hspace{1.9em}\mid (e, e)$                             & Product term\\
$\hspace{1.9em}\mid \lett{l}{e}{e}$                     & Let-in expression \\
$\hspace{1.9em}\mid \caseof{e}{\inl{x}}{e}{\inr{y}}{e}$ & Case-of expression\\
$\hspace{1.9em}\mid f~\alpha^*~e^+$                     & Function application\\
$\hspace{1.9em}\mid \roll{\tau}{e}$                     & Recursive-type construction \\
$\hspace{1.9em}\mid \unroll{\tau}{e}$                   & Recursive-type deconstruction\\
$l ::= x$                                               & Definition of variable\\
$\hspace{1.9em}\mid (x, x)$                             & Definition of product \\
$v ::= x:\tau_a$                                        & Variable declaration
\end{tabular}

\caption{Grammar of \rfunc. Program variables are denoted by $x$, and type
variables by $\alpha$.}\label{fig:grammar}
\end{figure}

\section{Type System}

Linear logic is the foundation for linear type theory. In linear logic, each
hypothesis must be used exactly once. Likewise, values which belong to a linear
type must be used exactly once, and may not be duplicated nor destroyed.
However, if we accept that functions may be partial (a necessity for
\emph{r-Turing completeness}~\cite{AxelsenGlueck:2016}), first-order data may
be duplicated reversibly. For this reason, we may relax the linearity
constraint to relevance, that is that all available variables \emph{must} be
used at least once.

A useful concept in reversible programming is access to ancillae, i.e. values
that remain unchanged across function calls. Such values are often used as a
means to guarantee reversibility in a straightforward manner.  To support such
ancillary variables at the type level, a type system inspired by Polakow’s
combined reasoning system of ordered, linear, and unrestricted intuitionistic
logic~\cite{Polakow:2001} is used. The type system splits the typing contexts
into two parts: a static one (containing ancillary variables and other static
parts of the environment), and a dynamic one (containing variables not
considered ancillary). This gives a typing judgment of $\con e : \tau$, where
$\Sigma$ is the static context and $\Gamma$ is the dynamic context.

\begin{figure}[t]
\begin{align*}
  \tau_f & ::= \tau_f \rightarrow \tau'_f
        \mid \tau \rightarrow \tau'_f
        \mid \tau \leftrightarrow \tau' \mid \forall X. \tau_f \\
  \tau & ::=
  1 \mid \tau \times \tau' \mid \tau + \tau' \mid X \mid
           \mu X.\tau \\
  \tau_a & ::= \tau \mid \tau \leftrightarrow \tau'
\end{align*}
\caption{Typing terms. Note that $X$ in this figure denotes any type
variable.}\label{fig:typing}
\end{figure}

We discern between two sets of typing terms: primitive types and arrow types.
This is motivated by a need to be careful about how we allow manipulation of
functions, as we will treat all functions as statically known.

The grammar for typing terms can be seen in Fig.~\ref{fig:typing}: $\tau_f$
denotes arrow types, $\tau$ primitive types, and $\tau_a$ ancillary types
(i.e., types of data that may be given as ancillary data).

Arrow types are types assigned to functions. For arrow types, we discern
between primitive types and arrow types in the right component of
unidirectional application. We only allow primitive types in bidirectional
application. This is to restrict functions to only being able to be bound to
ancillary parameters. This categorizes \rfunc as a restricted higher-order
language in that functions may be bound to variables, but only if they are
immediately known. It is ill-formed for a type bound in the dynamic context to
be of an arrow type --- if it were well-formed, we would be allowing the
creation of new functions, which would break our assumption that all functions
are statically known. We will detail the motivation for this restriction in
Sect.~\ref{sect:higher_order}.

Primitive types are types assigned to expressions which evaluate to canonical
values by the big step semantics. These are distinctly standard, containing sum
types and product types, as well as (rank-1) parametric polymorphic
types\footnote{A rank-1 polymorphic system may not instantiate type variables
with polymorphic types.} and a fix point operator for recursive data types
(see~\cite{Pierce:2002:TPL} for an introduction to the latter two concepts).

Throughout this thesis, we will write $\tau_1 + \cdots + \tau_n$ for the nested
sum type $\tau_1 + (\tau_2 + (\cdots + (\tau_{n-1} + \tau_n) \cdots))$ and
equivalently for product types $\tau_1 \times \cdots \times \tau_n$. Similarly,
as is usual, we will let arrows associate to the right.

\subsection{Type Rules for Expressions.}

The typing rules for expressions are shown in Fig.~\ref{fig:exprType}. A
combination of features of the typing rules enforces relevant typing:

\begin{enumerate}

  \item [(1)] \emph{Variable Typing Restriction:} The restriction on the
    contents of the dynamic context during certain type rules.

  \item [(2)] \emph{Dynamic Context Union:} A union operator for splitting the
    dynamic contexts in most type rules with more than one premise.

  \item [(3)] \emph{Context Update:} The assignment to the static context with
    new information instead of the dynamic when the dynamic context is empty.

\end{enumerate}

The rules for application are split into three different rules, corresponding
to application of dynamic parameters (\textsc{T-App1}), application of static
parameters (\textsc{T-App2}), and type instantiation for polymorphic functions
(\textsc{T-PApp}). Notice further the somewhat odd \textsc{T-Unit-Elm} rule.
Since relevant type systems treat variables as resources that must be consumed,
specific rules are required when data \emph{can} safely be discarded (such as
the case for data of unit type). What this rule essentially states is that
\emph{any} expression of unit type can be safely discarded; this is dual to the
\textsc{T-Unit} rule, which states that the unique value $()$ of unit type can
be freely produced (i.e.~in the empty context).

\paragraph{Variable Typing Restriction}

When applying certain axiomatic type rules (\textsc{T-Var1} and
\textsc{T-Unit}), we require the dynamic context to be empty. This is necessary
to inhibit unused parts of the dynamic context from being accidentally
``spilled'' through the use of these rules. Simultaneously, we require that
when we do use a variable from the dynamic context, the dynamic context
contains exactly this variable and nothing else. This requirement prohibits
the same sort of spilling.

\paragraph{Dynamic Context Union}

The union of the dynamic context is a method for splitting up the dynamic
context into named parts, which can then be used separately in the premises of
the rule. In logical derivations, splitting the known hypotheses is usually
written as $\Gamma, \Gamma' \vdash \dots$, but we deliberately introduce a
union operator to signify that we allow an overlap in the splitting of the
hypotheses. Were we not to allow overlapping, typing would indeed be linear.
For example, a possible split is:

$$
  \dfrac
    {\begin{array}{ccc}
        \overset{\vdots}{\emptyset; x \mapsto 1, y \mapsto 1 \vdashe \dots} &
        \quad &
        \overset{\vdots}{\emptyset; y \mapsto 1, z \mapsto 1 \vdashe \dots}
     \end{array}}
    {\emptyset; x \mapsto 1, y \mapsto 1, z \mapsto 1 \vdashe \dots}
$$

Here $y$ is part of the dynamic context in both premises.

\paragraph{Context Update}

We overload the rules for let and case-expressions depending on which context
we are going to update with the variable assignments in these rules. This is
motivated by what the form of the expression $e$ we are assigning to the
variable names is. If the dynamic context is empty, $e$ is necessarily one of
two things:

\begin{enumerate}

  \item [(1)] A closed term. A canonical value constructed by a closed term is
    free as no information is consumed in its creation, allowing us to assign
    it to the static context instead.

  \item [(2)] An open term with free variables from the static context. Since
    the static context may grow arbitrarily, and we have not consumed a dynamic
    variable in $e$, no information is lost by assigning the resulting
    canonical value to the static context instead.

\end{enumerate}

Therefore we bind variables introduced in let and case-expressions to the
static context when the dynamic context is empty for the derivation of $e$. The
three instances of overloaded rules can be seen by \textsc{T-Sum} versus
\textsc{T-SumSt}, \textsc{T-Let1} versus \textsc{T-Let1St} and \textsc{T-Let2}
versus \textsc{T-Let2St}.

Alternatively we could have introduced a restricted notion of \emph{weakening}.
Weakening is not regularly supported in linear or relevant logic systems as it
is not resource sensitive. But relevant logic dictates that we must only not
forget information before use, otherwise we may use it freely. With restricted
weakening, we say that if we already know a variable from the static
environment, we may freely forget it in the dynamic environment, as the
information is not lost.

This is more in line with expressing the idea that ancillae are static directly
in a well typed program, as we explicitly require that each ancilla is built
up again after we used it. However, it makes programs more long-winded. The
proposed weakening rule is:

\begin{align*}
  \textsc{T-Weakening: }
    \dfrac
      {\Sigma, x \mapsto \tau; \Gamma, x \mapsto \tau \vdash e : \tau'}
      {\Sigma, x \mapsto \tau; \Gamma \vdash e : \tau'}
\end{align*}

\begin{figure}[t!]

\setlength\fboxsep{0.15cm}
\noindent$\boxed{\text{Judgement: } \con e : \tau}$
\begin{align*}
  \textsc{T-Var1: }
    \dfrac
      {}
      {\conemp x : \tau} \Sigma(x) = \tau &&
  \textsc{T-Var2: }
    \dfrac
      {}
      {\Sigma; (x \mapsto \tau) \vdashe x : \tau}
\end{align*}
\alignspace
\begin{align*}
  \textsc{T-Unit: }
    \dfrac
      {}
      {\conemp () : 1} &&
  \textsc{T-Unit-Elm: }
    \dfrac
      {\begin{array}{ccc}
       \con e : 1 & \quad &
       \conp e' : \tau
       \end{array}}
      {\concup e' : \tau}
\end{align*}
\alignspace
\begin{align*}
  \textsc{T-Inl: }
    \dfrac
      {\con e : \tau}
      {\con \inl{e} : \tau + \tau'} &&
  \textsc{T-Inr: }
    \dfrac
      {\con e : \tau}
      {\con \inr{e} : \tau' + \tau}
\end{align*}
\alignspace
\begin{align*}
  \textsc{T-Prod: }
    \dfrac
      {\begin{array}{ccc}
       \con e_1 : \tau & \quad &
       \conp e_2 : \tau'
       \end{array}}
      {\concup (e_1, e_2) : \tau \times \tau'}
\end{align*}
\alignspace
\begin{align*}
  \textsc{T-App1: }
    \dfrac
      {\begin{array}{ccc}
       \con f : \tau \leftrightarrow \tau' & \quad &
       \conp e : \tau
       \end{array}}
      {\concup f~e : \tau'}
\end{align*}
\alignspace
\begin{align*}
  \textsc{T-App2: }
    \dfrac
      {\begin{array}{ccc}
       \con f : \tau_a \rightarrow \tau_f & \quad &
       \conemp e : \tau_a
       \end{array}}
      {\con f~e : \tau_f}
\end{align*}
\alignspace
\begin{align*}
  \textsc{T-PApp:}
    \dfrac
      {\con f : \forall \alpha.\tau_f}
      {\con f~\tau_a : \tau_f[\tau_a/\alpha]} &&
  \textsc{T-Let1: }
    \dfrac
      {\begin{array}{ccc}
       \con e_1 : \tau' & \quad &
       \Sigma; \Gamma', x : \tau' \vdashe e_2 : \tau
       \end{array}}
      {\concup \lett{x}{e_1}{e_2} : \tau}
\end{align*}
\alignspace
\begin{align*}
  \textsc{T-Let1St: }
    \dfrac
      {\begin{array}{ccc}
       \Sigma; \emptyset \vdash e_1 : \tau' & \quad &
       \Sigma, x : \tau' ; \Gamma, \vdashe e_2 : \tau
       \end{array}}
      {\con \lett{x}{e_1}{e_2} : \tau}
\end{align*}
\alignspace
\begin{align*}
  \textsc{T-Let2: }
    \dfrac
      {\begin{array}{ccc}
       \con e_1 : \tau' \times \tau'' & \quad &
       \Sigma; \Gamma', x : \tau', y : \tau'' \vdashe e_2 : \tau
       \end{array}}
      {\concup \lett{(x, y)}{e_1}{e_2} : \tau}
\end{align*}
\alignspace
\begin{align*}
  \textsc{T-Let2St: }
    \dfrac
      {\begin{array}{ccc}
       \Sigma; \emptyset \vdash e_1 : \tau' \times \tau'' & \quad &
       \Sigma, x : \tau', y : \tau''; \Gamma \vdashe e_2 : \tau
       \end{array}}
      {\con \lett{(x, y)}{e_1}{e_2} : \tau}
\end{align*}
\alignspace
\begin{align*}
  \textsc{T-Sum: }
    \dfrac
      {\begin{array}{ccccc}
       \con e_1 : \tau' + \tau'' & \quad &
       \Sigma; \Gamma', x : \tau' \vdashe e_2 : \tau & \quad &
       \Sigma; \Gamma', y : \tau'' \vdashe e_3 : \tau
       \end{array}}
      {\concup \caseof{e_1}{\inl{x}}{e_2}{\inr{y}}{e_3} : \tau}
\end{align*}
\alignspace
\begin{align*}
  \textsc{T-SumSt: }
    \dfrac
      {\begin{array}{ccccc}
       \Sigma; \emptyset \vdash e_1 : \tau' + \tau'' & \quad &
       \Sigma, x : \tau'; \Gamma \vdashe e_2 : \tau & \quad &
       \Sigma, y : \tau''; \Gamma \vdashe e_3 : \tau
       \end{array}}
      {\con \caseof{e_1}{\inl{x}}{e_2}{\inr{y}}{e_3} : \tau}
\end{align*}
\alignspace
\begin{align*}
  \textsc{T-Roll: }
    \dfrac
      {\con e : \tau' [\mu X . \tau' / X]}
      {\con \roll{\mu X . \tau'}{e} : \mu X . \tau'} &&
  \textsc{T-Unroll: }
    \dfrac
      {\con e : \mu X . \tau}
      {\con \unroll{\mu X . \tau}{e} : \tau' [\mu X . \tau / X]}
\end{align*}
\caption{Expression typing.}\label{fig:exprType}
\end{figure}

\subsection{Type Rules For Function Declarations.}

The type rules for function declarations are shown in Fig.~\ref{fig:funcType}.
Here \textsc{T-PFun} generalizes the type arguments, next \textsc{T-Fun1}
consumes the ancillary variables, and finally \textsc{T-Fun2} handles the last
dynamic variable by applying the expression typing.

\begin{figure}[t]
\setlength\fboxsep{0.15cm}
\noindent$\boxed{\text{Judgement: } \Sigma \vdashd d : \tau}$

\begin{align*}
  \textsc{T-Fun1: }
    \dfrac
      {\Sigma, x : \tau_a \vdashd f~v^+ = e : \tau_f}
      {\Sigma \vdashd f~x{:}\tau_a\ v^+ = e : \tau_a \rightarrow \tau_f} &&
  \textsc{T-Fun2: }
    \dfrac
      {\Sigma; (x \mapsto \tau) \vdashe e : \tau'}
      {\Sigma \vdashd f~x{:}\tau = e : \tau \leftrightarrow \tau'}
\end{align*}
\alignspace
\begin{align*}
  \textsc{T-PFun:}
    \dfrac
      {\Sigma \vdashd f~\beta^*~v^+ = e : \tau_f}
      {\Sigma \vdashd f~\alpha~\beta^*~v^+ = e : \forall \alpha.\tau_f} \alpha
      \notin \text{TV}(\Sigma)
\end{align*}

\caption{Function typing.}\label{fig:funcType}
\end{figure}

We implicitly assume that pointers to all defined functions are placed in the
static context $\Sigma$ as an initial step. For example, when typing an
expression $e$ in a program where a function $f~x = e$ is defined, and we have
been able to establish that $\Sigma \vdashd f~x = e : \tau \leftrightarrow
\tau'$ for some types $\tau, \tau'$, we assume that a variable $f : \tau
\leftrightarrow \tau'$ is placed in the static context in which we will type
$e$ ahead of time. This initial step amounts to a typing rule for the full
program.

Note that we write two very similar application rules \textsc{T-App1} and
\textsc{T-App2}. This discerns between function application of ancillary and
dynamic data, corresponding to the two different arrow types. In particular, as
shown in \textsc{T-App1}, application in the dynamic variable of a function is
only possible when that function is of type $\tau \leftrightarrow \tau'$, where
$\tau$ and $\tau'$ are non-arrow types: This specifically disallows
higher-order functions. Also note that the dynamic context must be empty for
application of the \textsc{T-App2} rule --- otherwise, it is treated as static
in the evaluation of $f$, and we have no guarantee how it is used.

\section{Recursive and Polymorphic Types}

The type system of \rfunc supports both recursive types as well as rank-1
parametrically polymorphic types. To support both of these, type variables,
which serve as holes that may be plugged by other types, are used.

For recursive types, we employ a standard treatment of iso-recursive types in
which explicit \textbf{roll} and \textbf{unroll} constructs are added to
witness the isomorphism between $\mu X. \tau$ and $\tau[\mu X. \tau/X]$ for a
given type $\tau$ (which, naturally, may contain the type variable $X$). For a
type $\tau$, we let $\text{TV}(\tau)$ denote the set of type variables that
appear free in $\tau$. For example, the type of lists of a given type $\tau$
can be expressed as the recursive type $\mu X. 1 + (\tau \times X)$, and
$\text{TV}(1 + (\tau \times X)) = \{X\}$ when the type $\tau$ contains no free
type variables. We define TV on contexts as $\text{TV}(\Sigma) = \{ v \in
\text{TV}(\tau) \mid x : \tau \in \Sigma \}$.

For polymorphism, we use an approach similar to System F, restricted to rank-1
polymorphism. In a polymorphic type system with rank-1 polymorphism, type
variables themselves cannot be instantiated with polymorphic types, but must be
instantiated with concrete types instead. While this approach is significantly
more restrictive than the full polymorphism of System F, it is expressive
enough that many practical polymorphic functions may be expressed (e.g.  ML and
Haskell both employ a form of rank-1 polymorphism based on the Hindley-Milner
type system~\cite{Milner:1978}), while being simple enough that type inference
is often both decidable and feasible in practice.

\section{Operational Semantics}\label{sec:semantics}

\begin{figure}[t]
\begin{align*}
  c ::= () \mid \inl{c} \mid \inr{c} \mid (c_1, c_2) \mid \roll{\tau}{c} \mid x
\end{align*}
\caption{Canonical forms.}\label{fig:canonical}
\end{figure}

We present a call-by-value big step operational semantics on expressions in
Fig.~\ref{math:semantics}, with canonical forms shown in
Fig.~\ref{fig:canonical}. As is customary with functional languages, we use
substitution (defined as usual by structural induction on expressions) to
associate free variables with values (canonical forms). Since the language does
not allow for values of a function type, we instead use an environment $p$ of
function definitions in order to perform computations in a context (such as a
program) with all defined functions.

\begin{figure}[ht]
\setlength\fboxsep{0.15cm}
\noindent$\boxed{\text{Judgement: } \pcon e \downarrow c}$

\begin{align*}
  \textsc{E-Unit: }
    \dfrac
      {}
      {\pcon () \downarrow ()} &&
  \textsc{E-Inl: }
    \dfrac
      {\pcon e \downarrow c}
      {\pcon \textbf{inl}(e) \downarrow \textbf{inl}(c)} &&
  \textsc{E-Inr: }
    \dfrac
      {\pcon e \downarrow c}
      {\pcon  \textbf{inr}(e) \downarrow \textbf{inr}(c)}
\end{align*}
\alignspace
\begin{align*}
  \textsc{E-Roll: }
    \dfrac
      {\pcon e \downarrow c}
      {\pcon \roll{\tau}{e} \downarrow \roll{\tau} c} &&
  \textsc{E-Unroll: }
    \dfrac
      {\pcon e \downarrow \roll{\tau}{c}}
      {\pcon \unroll{\tau}{e} \downarrow c}
\end{align*}
\alignspace
\begin{align*}
  \textsc{E-Prod: }
    \dfrac
      {\begin{array}{ccc}
       \pcon e_1 \downarrow c_1 & \quad &
       \pcon e_2 \downarrow c_2
       \end{array}}
      {\pcon (e_1, e_2) \downarrow (c_1, c_2)} &&
  \textsc{E-Let: }
    \dfrac
      {\begin{array}{ccc}
       \pcon e_1 \downarrow c_1 & \quad &
       \pcon e_2[c_1/x] \downarrow c
       \end{array}}
      {\pcon \textbf{let $x = e_1$ in $e_2$} \downarrow c}
\end{align*}
\alignspace
\begin{align*}
  \textsc{E-LetP: }
    \dfrac
      {\begin{array}{ccc}
       \pcon e_1 \downarrow (c_1,c_2) & \quad &
       \pcon e_2[c_1 / x, c_2 / y] \downarrow c
       \end{array}}
      {\pcon \textbf{let $(x,y) = e_1$ in $e_2$} \downarrow c}
\end{align*}
\alignspace
\begin{align*}
  \textsc{E-CaseL: }
    \dfrac
      {\begin{array}{ccc}
       \pcon e_1 \downarrow \inl{c_1} & \quad &
       \pcon e_2 [c_1 / x] \downarrow c
       \end{array}}
      {\pcon \caseof{e_1}{\inl{x}}{e_2}{\inr{y}}{e_3} \downarrow c}
\end{align*}
\alignspace
\begin{align*}
  \textsc{E-CaseR: }
    \dfrac
      {\begin{array}{ccccc}
       \pcon e_1 \downarrow \inr{c_1} & \quad &
       \pcon e_3 [c_1 / y] \downarrow c & \quad &
       %\pcon x \downarrow c'
       \end{array}}
      {\pcon \caseof{e_1}{\inl{x}}{e_2}{\inr{y}}{e_3} \downarrow c}c \notin
      \PLVal(e_2)% c \neq c'
\end{align*}
\alignspace
\begin{align*}
  \textsc{E-App: }
    \dfrac
      {\begin{array}{ccc}
       \pcon e_1 \downarrow c_1 \cdots \pcon e_n \downarrow c_n & \quad &
       \pcon e[c_1 / x_1,\ \cdots,\ c_n / x_n] \downarrow c
       \end{array}}
      {\pcon f\ \alpha_1 \cdots \alpha_m~e_1\ \cdots\ e_n \downarrow c}
      p(f) = f\ \alpha_1 \cdots \alpha_m ~ x_1 \cdots x_n = e
\end{align*}
\caption{Big step semantics of \rfunc.}\label{math:semantics}
\end{figure}

A common problem in reversible programming is to ensure that branching of
programs is done in such a way as to uniquely determine in the backward
direction which branch was taken in the forward direction. Since
case-expressions allow for such branching, we will need to define some rather
complicated machinery of \emph{leaf expressions}, \emph{possible leaf values},
and \emph{leaves} (the latter is similar to what is also used
in~\cite{YokoyamaAxelsenGlueck:2012:LNCS}) in order to give their semantics.

Say that an expression $e$ is a \emph{leaf expression} if it does not contain
any subexpression (including itself) of the form $\textbf{let $l = e_1$ in
$e_2$}$ or $\caseof{e_1}{\inl{x}}{e_2}{\inr{y}}{e_3}$; the collection of leaf
expressions form a set, $\text{LExpr}$. As the name suggests, a leaf expression
is an expression that can be considered as a \emph{leaf} of another expression.
The set of leaves of an expression $e$, denoted $\leaves(e)$, is defined in
Fig.~\ref{fig:leaves}.

\begin{figure}[tp]
\begin{align*}
  \leaves( () ) & = \left\{()\right\} \\
  \leaves((e_1, e_2)) & = \{ (e_1',e_2') \mid e_1' \in \leaves(e_1), \\
  & \phantom{= \{ (e_1',e_2') \mid}\ \ e_2' \in \leaves(e_2) \} \\
  \leaves(\inl{e}) & = \left\{ \inl{e'} \mid e' \in \leaves(e) \right\}\\
  \leaves(\inr{e}) & = \left\{ \inr{e'} \mid e' \in \leaves(e) \right\}\\
  \leaves(\roll{\tau}{e}) & = \left\{ \roll{\tau}{e'} \mid e' \in \leaves(e)
  \right\}\\
  \leaves(\textbf{let $l = e_1$ in $e_2$}) & = \leaves(e_2) \\
  \leaves(\caseof{e_1}{\inl{x}}{e_2}{\inr{y}}{e_3}) & = \leaves(e_2) \cup \leaves(e_3) \\
  \leaves(x) & = \{x\} \\
  \leaves(\unroll{\tau}{e}) & = \{\unroll{\tau}{e'} \mid e' \in \leaves(e) \}\\
  \leaves(f~e_1~\dots~e_n) & = \{f~e_1'~\dots~e_n' \mid e_i' \in \leaves(e_i) \}
\end{align*}
\caption{Definition function that computes the leaves of a program.}
\label{fig:leaves}
\end{figure}

The leaves of an expression are, in a sense, an abstract over-approximation of
its possible values, save for the fact that leaves may be leaf expressions
rather than mere canonical forms. We make this somewhat more concrete with the
definition of the \emph{possible leaf values} of an expression $e$, defined as:

\begin{align}
  \PLVal(e) = \{ e' \in \text{LExpr} \mid e'' \in \leaves(e), e' \rhd e'' \}
  \label{def:PLVal}
\end{align}

Where the relation $- \rhd -$ on leaf expressions is defined inductively as
(the symmetric closure\footnote{The symmetric closure of a binary relation on
a set is the smallest symmetric relation on that set which contains the
relation.} of):

\begin{align}
  () & \rhd () \nonumber \\
  (e_1, e_2) & \rhd (e_1', e_2')  & \text{if} \quad e_1 \rhd e_1'
  \text{ and } e_2 \rhd e_2' \nonumber \\
  \inl{e} & \rhd \inl{e'} & \text{if} \quad e \rhd e' \nonumber
  \label{def:rhd_relation} \\
  \inr{e} & \rhd \inr{e'} & \text{if} \quad e \rhd e' \\
  \roll{\tau}{e} & \rhd \roll{\tau}{e'} & \text{if} \quad e \rhd e' \nonumber \\
  e & \rhd x \nonumber \\
  e & \rhd f~e_1~\dots~e_n \nonumber \\
  e & \rhd \unroll{\tau}{e'} \nonumber
\end{align}

As such, the set $\PLVal(e)$ is the set of leaf expressions that can be
unified, in a certain sense, with a leaf of $e$. Since variables, function
applications, and unrolls do nothing to describe the syntactic form of possible
results, we define that these may be unified with \emph{any} expression. As
such, using $\PLVal(e)$ is somewhat conservative in that it may reject
definitions that are in fact reversible. Note also that $\PLVal(e)$
specifically includes all canonical forms that could be produced by $e$, since
all canonical forms are leaf expressions as well.

This way, if we can ensure that a canonical form $c$ produced by a branch in
a case-expression could not possibly have been produced by a \emph{previous}
branch in the case-expression, we know, in the backward direction, that $c$
must have been produced by the current branch. This is precisely the reason for
the side condition of $c \notin \PLVal(e_2)$ on \textsc{E-CaseR}, as this
conservatively guarantees that $c$ could not have been produced by the previous
branch.

It should be noted that for iterated functions\footnote{An iterated function
$f^n$ is a function $f$ composed with itself $n$ times.} this may add a
multiplicative execution overhead that is equal to the size of the data
structure. This effect has previously been shown in~\cite{Thomsen:2012:FDL},
where a \lstinline{plus} function over Peano numbers, which was linear
recursive over its input, actually had quadratic runtime.

It is immediate that should the side condition not hold for an expression $e'$,
no derivation is possible, and the expression does not evaluate to a value
entirely. In Sect.~\ref{sec:staticFMP}, we will look at exactly under which
conditions we can \emph{statically} guarantee that the side condition will hold
for every possible value of a function's domain.

We capture the conservative correctness of our definition of $\PLVal(e)$ with
respect to the operational semantics --- i.e. the property that any
canonical form $c$ arising from the evaluation of an expression $e$ will also
be ``predicted'' by $\PLVal$ in the sense that $c \in \PLVal(e)$ --- in the
following theorem:

\begin{theorem}
  If $p \vdash e \downarrow c$ then $c \in \PLVal(e)$.
\end{theorem}

\begin{proof}

  By induction on the structure of the derivation of $p \vdash e \downarrow c$.
  The proof is mostly straightforward: The case for \textsc{E-Unit} follows
  trivially, as do the cases for \textsc{E-Unroll} and \textsc{E-App} since
  leaves of $\unroll{\tau}{e}$ (respectively $f~e_1~\cdots~e_n$) are all of the
  form $\unroll{\tau}{e'}$ (respectively $f~e_1'~\cdots~e_n'$), and since $e''
  \rhd \unroll{\tau}{e'}$ (respectively $e'' \rhd f~e_1'~\cdots~e_n'$) for
  \emph{any} choice of $e''$, it follows that $\PLVal(\unroll{\tau}{e'}) =
  \PLVal(f~e_1~\cdots~e_n) = \text{LExpr}$. The cases for \textsc{E-Inl},
  \textsc{E-Inr}, \textsc{E-Roll}, and \textsc{E-Prod} all follow
  straightforwardly by induction, noting that $\PLVal(\inl{e}) = \{ \inl{e'}
  \mid e' \in \PLVal(e) \}$, and similarly for $\inr{e}$, $(e_1,e_2)$, and
  $\roll{\tau}{e}$. This leaves only the cases for \textbf{let} and
  \textbf{case} expressions, which follow using Lemma~\ref{thm:plval_subs}.

\end{proof}

\begin{lemma}\label{thm:plval_subs}
  For any expression $e$, variables $x_1, \dots, x_n$, and canonical forms
  $c_1, \dots, c_n$, we have $\PLVal(e[c_1/x_1, \dots, c_n/x_n]) \subseteq
  \PLVal(e)$.
\end{lemma}

\begin{proof}

This lemma follows straightforwardly by structural induction on $e$, noting
that it suffices to consider the case where $e$ is open with free variables
$x_1, \dots, x_n$, as it holds trivially when $e$ is closed (or when its free
variables are disjoint from $x_1, \dots, x_n$). With this lemma, showing the
case for e.g. \textsc{E-Let} is straightforward since $c \in
\PLVal(e_2[c_1/x])$ by induction, and since $\PLVal(e_2[c_1/x]) \subseteq
\PLVal(e_2)$ by this lemma, so $c \in \PLVal(e_2) =
\PLVal(\mathbf{let}~x=e_1~\mathbf{in}~e_2)$ by
$\leaves(\mathbf{let}~x=e_1~\mathbf{in}~e_2) = \leaves(e_2)$.

\end{proof}

\section{Reversibility}

Showing that the operational semantics are reversible amounts to showing that
they exhibit both forward and backward determinism. Showing forward
determinism is standard for any programming language (and holds
straightforwardly in \rfunc as well), but backward determinism is unique to
reversible programming languages. Before we proceed, we recall the usual
terminology of \emph{open} and \emph{closed} expressions: Say that an
expression $e$ is closed if it contains no free (unbound) variables, and open
otherwise.

Unlike imperative languages, where backward determinism is straightforwardly
expressed as a property of the reduction relation $\sigma \vdash c \downarrow
\sigma'$ where $\sigma$ is a store, backward determinism is somewhat more
difficult to express for functional languages, as the obvious analogue --- that
is, if $e \downarrow c$ and $e' \downarrow c$ then $e = e'$ --- is much too
restrictive (specifically, it is obviously \emph{not} satisfied in all but the
most trivial reversible functional languages). A more suitable notion turns out
to be a \emph{contextual} one, where rather than considering the reduction
behaviour of closed expressions in themselves, we consider the reduction
behaviour of canonical forms in a given \emph{context} (in the form of an open
expression) instead.

{
\renewcommand{\thetheorem}{\ref{thm:ctxbackwarddet}}
\begin{theorem}[Contextual Backwards Determinism]

  For all open expressions $e$ with free variables $x_1, \dots, x_n$, and all
  canonical forms $v_1, \dots, v_n$ and $w_1, \dots, w_n$, if $\pcon e[v_1 /
  x_1, \dots, v_n / x_n] \downarrow c$ and $\pcon e[w_1 / x_1, \dots, w_n /
  x_n] \downarrow c$ then $v_i = w_i$  for all $1 \le i \le n$.

\end{theorem}
\addtocounter{theorem}{-1}
}

\begin{proof}

The proof of this theorem follows by induction on the structure of $e$. For
brevity, we sometimes denote the set $x_1, \dots, x_n$ as $X$. The proof is
mostly straightforward:

\begin{itemize}

  \item Case $e = ()$. This follows immediately as there are no free variables
    in $e$.

  \item Case $e = x$. The only possible substitution is for the free variable
    $x$, and the only possible inference rule is \textsc{E-Var}. We have $x[v /
    x] \downarrow v$ and $x[w / x] \downarrow w$, so $v = w = c$.

  \item Case $e = \inl{e'}$. Follows from using the Induction Hypothesis on
    $e'$ and using the \textsc{E-Inl} rule to construct a derivation for $e$.
    The proof is identical for when $e = \inr{e'}$, when $e = \roll{\tau}{e'}$
    and when $e = \unroll{\tau}{e'}$.

  \item Case $e = (e', e'')$. The only possible rule for the derivation is
    \textsc{E-Prod}. First, we must consider which open terms occur in $e'$ and
    $e''$ respectively. We have some subset $Y = y_1, \dots, y_k \in X$, of
    open terms occurring in $e'$, and likewise, we have some subset $Z = z_1,
    \dots, z_l \in X$ occurring $e''$, such that $Y \cup Z = X$. Note that any
    $y_i$ or $z_i$ is simply an alias for some $x_j$, they are not fresh
    variables. By the Induction Hypothesis on $e'$ and $e''$ we prove it for
    both premises. Now, what about any $x_k \in Y \cap Z$? Any value $v_k$ has
    not been proven to be the same as $w_k$, but it certainly has to hold. But
    we have this by the inductive definition of substitution:

    \begin{align*}
      (e', e'') [c / x] = (e' [c / x], e'' [c / x])
    \end{align*}

  \item Case $e = \lett{x}{e'}{e''}$. We forego the argument about the
    partitioning of open terms as in the case before, but take note that it
    holds here as well. We first prove it for $e'$ by the Induction Hypothesis
    on $e'$.

    Now, we have by assumption that the expressions $(\lett{x}{e_1}{e_2})[v_1 /
    x_1, \dots, v_n / x_n]$ and $(\lett{x}{e_1}{e_2})[w_1 / x_1, \dots, w_n /
    x_n]$ evaluate to the same canonical value $c$. Thus the theorem holds for
    the only free substitution (of $x$) in $e''$, and combined with the
    Induction Hypothesis on $x_1, \dots, x_n$ it holds for $e$ as a whole.

  % \item Case $e = f~\alpha_1 \dots \alpha_m~e_1 \dots e_q$. We have that $p(f)
  %   = f~\alpha_1 \dots \alpha_m~y_1 \dots y_q = e'$. Straightforwardly by the
  %   Induction Hypothesis on each expression $e_1, \dots, e_q$, we have that
  %   $c_1' = c_1'', \dots, c_q' = c_q''$ where $e_k[v_1 / x_1, \dots, v_n / x_n]
  %   \downarrow c_k'$ and $e_k[w_1 / x_1, \dots, w_n / x_n] \downarrow c_k''$
  %   for $i \in 1 \dots q$. Now, by the Induction Hypothesis on $e'[c_1' / y_1,
  %   \dots c_q' / y_q]$, we have what we want.

  \item Case $e = f~\alpha_1 \dots \alpha_m~e_1 \dots e_q$. We have that $p(f)
    = f~\alpha_1 \dots \alpha_m~y_1 \dots y_q = e'$. There are $q+1$ premises
    and thus $q+1$ possible partitions of the open terms. Luckily, the theorem
    holds by simply applying the Induction Hypothesis on the $q$ first
    premises, followed by noticing that every other free variable in $e'$
    (variables $y_1, \dots, y_q$) must be the substituted for the same set of
    canonical values by the theorem assumption --- this paired with an
    application of the Induction Hypothesis gives us what we want.

  \item Case $e = \caseof{e'}{\inl{x}}{e''}{\inr{y}}{e'''}$. We may use two
    inference rules, based on the derivation of $e'$. Induction on $e'$ derives
    a value $c$ (where $e'$ contains some subset $Y = y_1, \dots, y_k \in X$ of
    free variables). The possible forms of the value $c$ are:

    \begin{itemize}

      \item $c' = \inl{x}$. By the Induction Hypothesis on $e''$ with the
        \textsc{E-CaseL} rule, we have what we want, with $e''$ evaluating to
        some $c'$, where $e''$ contains some set of open terms $A \in X$.

      \item $c' = \inr{y}$. By the Induction Hypothesis on $e'''$ with the
        \textsc{E-CaseR} rule, we have that the theorem holds for $e'''$ with
        open terms $B = a_1, \dots, a_l \in x_1, \dots, x_n$, such that $Y \cup
        B = X$, where $e'''$ evaluates to some $c''$.

        Both evaluation rules are over the same expression: This means the
        expression contain the same set of open terms $Y$ in the first premise
        and not necessarily the same set of open terms $A$ and $B$ in the
        second premise. By the Induction Hypothesis on $e'''$ we have
        contextual backwards determinism for the premise $e'''$, but there is
        no way of guaranteeing that the substitutions of $A$ and $B$ are
        equivalent, and they can both evaluate to the same closed term where
        $c' = c''$.

        At this point we invoke the side condition of \text{E-CaseR}, which
        specifically states that should a derivation using the \textsc{E-CaseR}
        inference rule be possible so that $c' = c''$, then the derivation of
        $e''$ to $c'$ must not exist.

    \end{itemize}

\end{itemize}

And we have covered all the cases.

\end{proof}

\begin{corollary}[Injectivity]
  For all functions $f$ and canonical forms $v,w$, if $\pcon f~v \downarrow c$
  and $\pcon f~w \downarrow c$ then $v = w$.
\end{corollary}

\begin{proof}
  Let $e$ be the open expression $f~x$ (with free variable $x$). Since
  $(f~x)[v/x] = f~v$ and $(f~x)[w/x] = f~w$, applying
  Theorem~\ref{thm:ctxbackwarddet} on $e$ yields precisely that if $\pcon f~v
  \downarrow c$ and $\pcon f~w \downarrow c$ then $v=w$.
\end{proof}

% \section{Type Soundness}
%
% \begin{itemize}
%
%   \item There are some standard exercises when presenting a type system over a
%     language to show that it is \emph{sound}, meaning that it rejects ill-typed
%     programs. This is often done by showing progress and preservation of a type
%     system. The idea is due to~\cite{Wright:1994}.
%
%   \item Progress says that a derivation cannot be stuck, e.g.~any non-diverging
%     computation will derive to some value eventually (it is safe.) It has to do
%     with the semantics.
%
%   \item Preservation says that if a term evaluates to a value, then the type of
%     of the term and the value are the same.
%
%   \item I could mention these, but with a big step evaluation strategy they are
%     not very interesting properties, as computations are not really modeled
%     (through a small-step semantic), we just expect expressions to evaluate to
%     values and not get stuck.
%
% \end{itemize}
%
% \section{r-Turing Completeness}
%
% \begin{itemize}
%
%   \item Define a r-Turing machine.
%
%   \item Show how an r-Turing machine can be simulated in \rfunc.
%
% \end{itemize}
