%-------------------------------------------------------------------------------
% PACKAGES AND OTHER DOCUMENT CONFIGURATIONS
%-------------------------------------------------------------------------------

\documentclass[a4paper,10p,openright]{memoir}

\usepackage{dikuReport}

\usepackage[utf8]{inputenc}
\usepackage[T1]{fontenc}
\usepackage{amssymb}
\usepackage{amsmath}
\usepackage{amsthm}
\usepackage{latexsym}
\usepackage{caption}
\usepackage{fancyhdr}
\usepackage{lastpage}
\usepackage{extramarks}
\usepackage{color}
\usepackage{graphicx}
\usepackage{listings}
\usepackage{courier}
\usepackage{lipsum}
\usepackage{tabularx}
\usepackage{mathtools}
\usepackage{mathrsfs}
\usepackage{tikz}
\usepackage{lstrfun}
\usepackage{xspace}
\usepackage{cite}
\usepackage{microtype}
\usepackage{caption}
\usepackage{todonotes}
\usepackage{subcaption}
\usepackage{url}
\usepackage{stmaryrd}
\usepackage{ebproof}
\usepackage{xcolor,calc}
\usepackage{float}
\usepackage{morefloats}
\usepackage{latexsym}
\usepackage{fix-cm}
\usepackage{pdfpages}
\usepackage{amsbsy}
\usepackage{alltt}
\usepackage{relsize}
\usepackage{ifthen}
\usepackage[]{units}
\usepackage{hyperref}

% Include some layout setup.
%!TEX root = main.tex

% NO RESETTING OF FOOTNOTES AT CHAPER CHANGE
\usepackage{remreset}
\makeatletter\@removefromreset{footnote}{chapter}\makeatother

% Set the style of chapter heads

%\definecolor{TemplateColor}{rgb}{.75,.116,.60}  % KU Red
\definecolor{TemplateColor}{rgb}{.2941,.4549,.2359} % KU green

\newcommand{\pnumb}[1]{
    \begin{tikzpicture}
      \draw[fill,color=gray] (0,0) rectangle (5mm,5mm);
      \draw[color=white] (2.5mm,2.5mm) node { #1};
    \end{tikzpicture}
}
\newcommand{\printchapternumm}{}
\makechapterstyle{combined}{
  \setlength{\midchapskip}{-60pt}
  \setlength{\afterchapskip}{2.5cm}
  \renewcommand*{\printchaptername}{}
  \renewcommand*{\chapnumfont}{\normalfont\bfseries\fontsize{45}{0}\selectfont}
  \renewcommand*{\printchapternum}{\flushright\chapnumfont\textcolor{TemplateColor}{\thechapter}}
  \renewcommand*{\chaptitlefont}{\normalfont\Huge\bfseries}
  \renewcommand*{\printchaptertitle}[1]{%
    \raggedright\chaptitlefont\parbox[t]{\textwidth-2cm}{\raggedright##1}}
}


%% DEPTH OF TITLE NUMBERING
% Command                            Level        Comment
% ------------------------------------------------------------------------
% \part{part}                        -1       not in letters
% \chapter{chapter}                   0       only books and reports
% \section{section}                   1       not in letters
% \subsection{subsection}             2       not in letters
% \subsubsection{subsubsection}       3       not in letters
% \paragraph{paragraph}               4       not in letters
% \subparagraph{subparagraph}         5       not in letters
\setcounter{secnumdepth}{2}

% Only clearpage (not cleardoublepage) at new chapter.
\renewcommand{\clearforchapter}{\clearpage}


\def\UrlFont{\rmfamily}

\newtheorem{theorem}{Theorem}
\newtheorem{corollary}{Corollary}
\newtheorem{lemma}{Lemma}
\newtheorem{definition}{Definition}

% Margins
% \setlength{\marginparwidth}{2.75cm}
\topmargin=0in
\evensidemargin=0.0in
\oddsidemargin=0.0in
\textwidth=6.2in
\textheight=8.5in
\headsep=0.35in

% \makeatletter
% \renewcommand\chapter{\if@openright\cleardoublepage\else\clearpage\fi
%                     \thispagestyle{fancy}
%                     \global\@topnum\z@
%                     \@afterindentfalse
%                     \secdef\@chapter\@schapter}
% \makeatother

\linespread{1.1} % Line spacing

% Set up the header and footer
% \pagestyle{fancy}
% \lhead{} % Top left header
% \chead{}
% \rhead{} % Top right header
% \lfoot{\lastxmark} % Bottom left footer
% \cfoot{} % Bottom center footer
% \rfoot{Page\ \thepage} % Bottom right footer
% \renewcommand\headrulewidth{0pt} % Size of the header rule
% \renewcommand\footrulewidth{0.4pt} % Size of the footer rule

% \setlength\parindent{0pt} % Removes all indentation from paragraphs

\tikzset{node distance=2cm, auto}

\lstset{language=RFun}

\newcommand*{\anddot}{%
  \mathclose{}%
  \nonscript\mskip.5\thinmuskip
  \boldsymbol{.}%
  \;%
  \mathopen{}%
}

\def\rfunc{\ensuremath{\mathsf{CoreFun}}\xspace}
% \def\rfunfst{\ensuremath{\mathsf{RFun}_{fst}}\xspace}
\def\rfun{\ensuremath{\mathsf{RFun}}\xspace}

%-------------------------------------------------------------------------------
% SEMANTICS COMMANDS
%-------------------------------------------------------------------------------

% Environments
\newcommand{\cod}{\text{cod}}
\newcommand{\dom}{\text{dom}}
\newcommand{\Hom}{\text{Hom}}

\newcommand{\leaves}{\text{leaves}}
\newcommand{\sfmp}{\text{staticFMP}}
\newcommand{\PLVal}{\text{PLVal}}

\def\vdashe{\vdash}
\def\vdashd{\vdash\!\!\!\vdash}
\newcommand{\con}{\Sigma;\Gamma\vdash}
\newcommand{\conp}{\Sigma;\Gamma'\vdash}
\newcommand{\conpp}{\Sigma;\Gamma''\vdash}
\newcommand{\conemp}{\Sigma;\emptyset\vdash}
\newcommand{\concup}{\Sigma;\Gamma\cup\Gamma'\vdash}
\newcommand{\concupp}{\Sigma;\Gamma\cup\Gamma''\vdash}
\newcommand{\pcon}{p \vdash}
\newcommand{\storeinv}{p;\sigma\vdash^{-1}}
\newcommand{\emptyinv}{p;\emptyset\vdash^{-1}}

\newcommand{\dcon}{\Delta\vdash}

% Constructs
\newcommand{\inl}[1]{\textbf{inl}(#1)}
\newcommand{\inr}[1]{\textbf{inr}(#1)}

\newcommand{\lett}[3]{\textbf{let } #1 = #2 \textbf{ in } #3}
\newcommand{\caseof}[5]{\textbf{case } #1 \textbf{ of } #2 \Rightarrow #3, #4 \Rightarrow #5}
\newcommand{\caseofu}[5]{\textbf{case } #1 \textbf{ of } #2 \Rightarrow #3, #4 \Rightarrow #5 \textbf{ unsafe }}
\newcommand{\caseofs}[6]{\textbf{case } #1 \textbf{ of } #2 \Rightarrow #3, #4 \Rightarrow #5 \textbf{ safe } #6}
\newcommand{\smpcase}[2]{\textbf{case } #1 \textbf{ of } #2}

\newcommand{\roll}[2]{\textbf{roll } [#1]\ #2}
\newcommand{\unroll}[2]{\textbf{unroll } [#1]\ #2}

\newcommand{\abort}[2]{\textbf{abort}_#1\ #2}

\newcommand{\class}[2]{\textbf{class } #1 \textbf{ where } #2}
\newcommand{\instance}[2]{\textbf{instance } #1 \textbf{ where } #2}

\newcommand{\within}[2]{\textbf{within } #1 : #2 \textbf{ end}}

\newcommand{\defeq}{\stackrel{\>\text{def}}{=}\>}

\begin{document}

%-------------------------------------------------------------------------------
% TITLE PAGE
%-------------------------------------------------------------------------------

% Basic information
\thesistype{MSc thesis}
\thesiscomment{} % You can leave this blank
\title{Design of a Reversible Functional Programming Language}
\subtitle{And its Type System}
\author{Petur Andrias Højgaard Jacobsen}
\cosupervisor{Robin Kaarsgaard Jensen}
\supervisor{Michael Kirkedal Thomsen}
\date{\today} % Hand-in date
%\subject{The short description that is suitable for a database.} % This is not needed.

% Make the front page, title page, and other required information.
\pagestyle{plain}
\maketitle

% Start at page 3. I do not count the front page in the numbering.
\cleardoublepage
\pagenumbering{roman}
\setcounter{page}{3}

\cleardoublepage
\pagestyle{plain}

%-------------------------------------------------------------------------------

\begin{abstract}

  Reversible programming languages are languages which exhibit both forward and
  backward determinism. The theory of reversible flowchart languages is
  relatively well understood, but studies of reversible functional languages
  are few and far between. In this thesis, we introduce a garbage-free,
  reversible, function language, inspired by \rfun, which we call \rfunc. We
  provide a formalization of its semantics and extend it with a type system
  based on relevance logic. The type system also has support for recursive and
  polymorphic types. With the type system, we are able to add the use of
  ancillae variables through an unrestricted fragment. Backwards determinism of
  of branching is achieved with a First Match Policy, but we investigate the
  possibility of ensuring backwards determinism with exit assertions or static
  guarantees of orthogonality as alternatives. As program inversion of
  non-flowchart languages generally is hard, we present a formalization of its
  inverse semantics instead. Finally, we describe how to lift the core language
  into a syntactically lighter language via a sequence of translation schemes,
  making it more amenable for modern style programming.

\end{abstract}

% Table of contents
\cleardoublepage
\chapterstyle{combined}
\tableofcontents*

% Starting the real text.
\cleardoublepage
\pagenumbering{arabic}
\setcounter{page}{1}

%-------------------------------------------------------------------------------

\chapter{Introduction}

Reversible computing is the study of computational models in which individual
computation steps can be uniquely and unambiguously inverted. For programming
languages, this means languages in which programs can be run \emph{backward}
and get a unique result (the exact input). In this thesis, we restrict ourselves
to \emph{garbage-free} reversible programming languages, which guarantee not
only that all programs are reversible, but also that no hidden duplication of
data is required in order to make this guarantee.

In this thesis, we present a simple, but \emph{r-Turing
complete}~\cite{AxelsenGlueck:2016}, reversible typed functional programming
language, \rfunc. Functional languages and programming constructs are currently
quite successful; this includes both applications in special domains, e.g.
Erlang, and functional constructs introduced in mainstream programming
languages, such as Java and C++. We believe that functional languages also
provide a suitable environment for studying reversible programs and
computations, as recently shown in~\cite{ThomsenAxelsen:2016:IFL}. However, the
lack of a type system exposed the limitations of the original \rfun language,
which has motivated this work. A carefully designed type system can provide
better handling of static information through the introduction of
\emph{ancillae typed} variables, which are guaranteed to be unchanged across
function calls. Further, it can often be used to statically verify the
\emph{first match policy} that is essential to reversibility of partially
defined functions. It should be noted that this type system is not meant to
guarantee reversibility of well-typed programs (rather, guaranteeing
reversibility is a job for the \emph{semantics}). Instead, the type system aids
in the clarity of expression for programs, provides fundamental
well-behavedness guarantees, and is a source of additional static information
which can enable static checking of certain properties, such as the
aforementioned \emph{first-match policy}.

An implementation of the work in this thesis can be found at:

\begin{center}
  \texttt{\url{https://github.com/diku-dk/coreFun/}}
\end{center}

\section{Background}

Initial studies of reversible (or information lossless) computation date back
to the years around 1960. These studies were based on quite different
computation models and motivations: Huffman studied information lossless finite
state machines for their applications in data transmission~\cite{Huffman:1959},
Landauer came to study reversible logic in his quest to determine the sources
of energy dissipation in a computing system~\cite{Landauer:1961}, and Lecerf
studied reversible Turing machines for their theoretical
properties~\cite{Lecerf:1963}.

Although the field is often motivated by a desire for energy and entropy
preservation though the work of Landauer~\cite{Landauer:1961}, we are more
interested in the possibility to use reversibility as a property that can aid
in the execution of a system, an approach which can be credited to
Huffman~\cite{Huffman:1959}. It has since been used in areas like programming
languages for quantum computation~\cite{GreenEtAl:2013:Quip}, parallel
computing~\cite{SchordanEtal:2015:RC}, and even
robotics~\cite{SchultzLEA:2015}. This diversity motivates studying reversible
functional programming (and other paradigms) independently, such that we can
get a better understanding of how to improve reversible programming in these
diverse areas.

The earliest reversible programming language (to the authors' knowledge) is
Janus, an imperative language invented in the 1980's, and later
rediscovered~\cite{LutzDerby:1986,YokoyamaGlueck:2007:Janus} as interest in
reversible computation spread. Janus and languages deriving from it have
since been studied in detail, so that we today have a reasonably good
understanding of these kinds of reversible flowchart
languages~\cite{YokoyamaEtAl:2015,GlueckKaarsgaard:2018}.

Reversible functional programming languages are still at an early stage of
development, and today only a few proof-of-concept languages exist. This work
is founded on the initial work on
\rfun~\cite{YokoyamaAxelsenGlueck:2012:LNCS,ThomsenAxelsen:2016:IFL}, while
another notable example of a reversible functional language is
Theseus~\cite{JamesSabry:2014:RC}, which has recently been further developed
towards a language for quantum computations~\cite{SabryEtal:2018}.

%% Linear types

The type system formulated here is based on relevance logic (originally
introduced in \cite{Anderson:1975}, see also \cite{Dunn:2002}), a substructural
logic similar to linear logic~\cite{Girard:1987,Wadler:1990} which (unlike
linear logic) permits the duplication of data. In reversible functional
programming, linear type systems (see e.g.~\cite{JamesSabry:2014:RC}) have
played an important role in ensuring reversibility, but they also appear in
modern languages like the Rust programming language~\cite{Matsakis:2014}. To
support ancillary variables at the type level, we adapt a type system inspired
by Polakow's combined reasoning system of ordered, linear, and unrestricted
intuitionistic logic~\cite{Polakow:2001}.

The rest of this thesis is organised in the following way: In
Sect.~\ref{sec:formal} we will first introduce \rfunc followed by the type
system and operational semantics. We also discuss type polymorphism and show
that the language is indeed reversible. In Sect.~\ref{sec:staticFMP} we will
show how the type system in some cases can be used to statically verify the
first match policy. In Sect.~\ref{sec:running_backwards} we discuss program
inversion and present inverse operational semantics. Sect.~\ref{sec:prog} we
show how syntactic sugar can be used to design a more modern style functional
language from \rfunc. In Sect.~\ref{sec:implementation} we briefly introduce a
reference implementation. In Sect.~\ref{sec:discussion} we discuss language
design and future work. Finally in Sect.~\ref{sec:conclusion} we conclude.

%-------------------------------------------------------------------------------
% LANGUAGE
%-------------------------------------------------------------------------------

% !TEX root = main.tex

\newcommand{\alignspace}{\vspace{-4mm}}

\chapter{Formalisation of \rfunc}\label{sec:formal}

The following section will present the formalisation of \rfunc. The language is
intended to be minimal, but it will still accommodate future extensions to a
modern style functional language. We first present a core language syntax,
which will work as the base of all formal analysis. Subsequently we present
typing rules and operational semantics over this language. The following is
built on knowledge about implementation of type systems as explained
in~\cite{Pierce:2002:TPL}.

\section{Grammar}

A program is a collection of zero or more function definitions. Each definition
must be defined over some number of input variables as constant functions are
not interesting in a reversible setting. All function definitions will in
interpretation be available though a static context.  A typing of a program is
synonymous with a typing of each function. A function is identified by a name
$f$ and takes 0 or more type parameters, and 1 or more formal parameters as
inputs. Each formal parameter $x$ is associated with a typing term $\tau$ at
the time of definition for each function, which may be one of the type
variables given as type parameter.  The grammar is given in
Fig.~\ref{fig:grammar}.

\begin{figure}[t]
\begin{tabular}{p{.5\textwidth}l}
$q ::= d^*$                                             & Program definition\\
$d ::= f\ \alpha^*~v^+ = e$                             & Function definition\\
$e ::= x$                                               & Variable name\\
$\hspace{1.9em}\mid ()$                                 & Unit term\\
$\hspace{1.9em}\mid \inl{e}$                            & Left of sum term\\
$\hspace{1.9em}\mid \inr{e}$                            & Right of sum term\\
$\hspace{1.9em}\mid (e, e)$                             & Product term\\
$\hspace{1.9em}\mid \lett{l}{e}{e}$                     & Let-in expression \\
$\hspace{1.9em}\mid \caseof{e}{\inl{x}}{e}{\inr{y}}{e}$ & Case-of expression\\
$\hspace{1.9em}\mid f~\alpha^*~e^+$                     & Function application\\
$\hspace{1.9em}\mid \roll{\tau}{e}$                     & Recursive-type construction \\
$\hspace{1.9em}\mid \unroll{\tau}{e}$                   & Recursive-type deconstruction\\
$l ::= x$                                               & Definition of variable\\
$\hspace{1.9em}\mid (x, x)$                             & Definition of product \\
$v ::= x:\tau_a$                                        & Variable declaration
\end{tabular}

\caption{Grammar of \rfunc. Program variables are denoted by $x$, and type
variables by $\alpha$.}\label{fig:grammar}
\end{figure}

\section{Type System}

Linear logic is the foundation for linear type theory. In linear logic, each
hypothesis must be used exactly once. Likewise, values which belong to a linear
type must be used exactly once, and may not be duplicated nor destroyed.
However, if we accept that functions may be partial (a necessity for
\emph{r-Turing completeness}~\cite{AxelsenGlueck:2016}), first-order data may
be duplicated reversibly. For this reason, we may relax the linearity
constraint to relevance, that is that all available variables \emph{must} be
used at least once.

A useful concept in reversible programming is access to ancillae, i.e. values
that remain unchanged across function calls. Such values are often used as a
means to guarantee reversibility in a straightforward manner.  To support such
ancillary variables at the type level, a type system inspired by Polakow’s
combined reasoning system of ordered, linear, and unrestricted intuitionistic
logic~\cite{Polakow:2001} is used. The type system splits the typing contexts
into two parts: a static one (containing ancillary variables and other static
parts of the environment), and a dynamic one (containing variables not
considered ancillary). This gives a typing judgment of $\con e : \tau$, where
$\Sigma$ is the static context and $\Gamma$ is the dynamic context.

\begin{figure}[t]
\begin{align*}
  \tau_f & ::= \tau_f \rightarrow \tau'_f
        \mid \tau \rightarrow \tau'_f
        \mid \tau \leftrightarrow \tau' \mid \forall X. \tau_f \\
  \tau & ::=
  1 \mid \tau \times \tau' \mid \tau + \tau' \mid X \mid
           \mu X.\tau \\
  \tau_a & ::= \tau \mid \tau \leftrightarrow \tau'
\end{align*}
\caption{Typing terms. Note that $X$ in this figure denotes any type
variable.}\label{fig:typing}
\end{figure}

We discern between two sets of typing terms: primitive types and arrow types.
This is motivated by a need to be careful about how we allow manipulation of
functions, as we will treat all functions as statically known.

The grammar for typing terms can be seen in Fig.~\ref{fig:typing}: $\tau_f$
denotes arrow types, $\tau$ primitive types, and $\tau_a$ ancillary types
(i.e., types of data that may be given as ancillary data).

Arrow types are types assigned to functions. For arrow types, we discern
between primitive types and arrow types in the right component of
unidirectional application. We only allow primitive types in bidirectional
application. This is to restrict functions to only being able to be bound to
ancillary parameters. This categorizes \rfunc as a restricted higher-order
language in that functions may be bound to variables, but only if they are
immediately known. It is ill-formed for a type bound in the dynamic context to
be of an arrow type --- if it were well-formed, we would be allowing the
creation of new functions, which would break our assumption that all functions
are statically known. We will detail the motivation for this restriction in
Sect.~\ref{sect:higher_order}.

Primitive types are types assigned to expressions which evaluate to canonical
values by the big step semantics. These are distinctly standard, containing sum
types and product types, as well as (rank-1) parametric polymorphic
types\footnote{A rank-1 polymorphic system may not instantiate type variables
with polymorphic types.} and a fix point operator for recursive data types
(see~\cite{Pierce:2002:TPL} for an introduction to the latter two concepts).

Throughout this thesis, we will write $\tau_1 + \cdots + \tau_n$ for the nested
sum type $\tau_1 + (\tau_2 + (\cdots + (\tau_{n-1} + \tau_n) \cdots))$ and
equivalently for product types $\tau_1 \times \cdots \times \tau_n$. Similarly,
as is usual, we will let arrows associate to the right.

\subsection{Type Rules for Expressions.}

The typing rules for expressions are shown in Fig.~\ref{fig:exprType}. A
combination of features of the typing rules enforces relevant typing:

\begin{enumerate}

  \item [(1)] \emph{Variable Typing Restriction:} The restriction on the
    contents of the dynamic context during certain type rules.

  \item [(2)] \emph{Dynamic Context Union:} A union operator for splitting the
    dynamic contexts in most type rules with more than one premise.

  \item [(3)] \emph{Context Update:} The assignment to the static context with
    new information instead of the dynamic when the dynamic context is empty.

\end{enumerate}

The rules for application are split into three different rules, corresponding
to application of dynamic parameters (\textsc{T-App1}), application of static
parameters (\textsc{T-App2}), and type instantiation for polymorphic functions
(\textsc{T-PApp}). Notice further the somewhat odd \textsc{T-Unit-Elm} rule.
Since relevant type systems treat variables as resources that must be consumed,
specific rules are required when data \emph{can} safely be discarded (such as
the case for data of unit type). What this rule essentially states is that
\emph{any} expression of unit type can be safely discarded; this is dual to the
\textsc{T-Unit} rule, which states that the unique value $()$ of unit type can
be freely produced (i.e.~in the empty context).

\paragraph{Variable Typing Restriction}

When applying certain axiomatic type rules (\textsc{T-Var1} and
\textsc{T-Unit}), we require the dynamic context to be empty. This is necessary
to inhibit unused parts of the dynamic context from being accidentally
``spilled'' through the use of these rules. Simultaneously, we require that
when we do use a variable from the dynamic context, the dynamic context
contains exactly this variable and nothing else. This requirement prohibits
the same sort of spilling.

\paragraph{Dynamic Context Union}

The union of the dynamic context is a method for splitting up the dynamic
context into named parts, which can then be used separately in the premises of
the rule. In logical derivations, splitting the known hypotheses is usually
written as $\Gamma, \Gamma' \vdash \dots$, but we deliberately introduce a
union operator to signify that we allow an overlap in the splitting of the
hypotheses. Were we not to allow overlapping, typing would indeed be linear.
For example, a possible split is:

$$
  \dfrac
    {\begin{array}{ccc}
        \overset{\vdots}{\emptyset; x \mapsto 1, y \mapsto 1 \vdashe \dots} &
        \quad &
        \overset{\vdots}{\emptyset; y \mapsto 1, z \mapsto 1 \vdashe \dots}
     \end{array}}
    {\emptyset; x \mapsto 1, y \mapsto 1, z \mapsto 1 \vdashe \dots}
$$

Here $y$ is part of the dynamic context in both premises.

\paragraph{Context Update}

We overload the rules for let and case-expressions depending on which context
we are going to update with the variable assignments in these rules. This is
motivated by what the form of the expression $e$ we are assigning to the
variable names is. If the dynamic context is empty, $e$ is necessarily one of
two things:

\begin{enumerate}

  \item [(1)] A closed term. A canonical value constructed by a closed term is
    free as no information is consumed in its creation, allowing us to assign
    it to the static context instead.

  \item [(2)] An open term with free variables from the static context. Since
    the static context may grow arbitrarily, and we have not consumed a dynamic
    variable in $e$, no information is lost by assigning the resulting
    canonical value to the static context instead.

\end{enumerate}

Therefore we bind variables introduced in let and case-expressions to the
static context when the dynamic context is empty for the derivation of $e$. The
three instances of overloaded rules can be seen by \textsc{T-Sum} versus
\textsc{T-SumSt}, \textsc{T-Let1} versus \textsc{T-Let1St} and \textsc{T-Let2}
versus \textsc{T-Let2St}.

Alternatively we could have introduced a restricted notion of \emph{weakening}.
Weakening is not regularly supported in linear or relevant logic systems as it
is not resource sensitive. But relevant logic dictates that we must only not
forget information before use, otherwise we may use it freely. With restricted
weakening, we say that if we already know a variable from the static
environment, we may freely forget it in the dynamic environment, as the
information is not lost.

This is more in line with expressing the idea that ancillae are static directly
in a well typed program, as we explicitly require that each ancilla is built
up again after we used it. However, it makes programs more long-winded. The
proposed weakening rule is:

\begin{align*}
  \textsc{T-Weakening: }
    \dfrac
      {\Sigma, x \mapsto \tau; \Gamma, x \mapsto \tau \vdash e : \tau'}
      {\Sigma, x \mapsto \tau; \Gamma \vdash e : \tau'}
\end{align*}

\begin{figure}[t!]

\setlength\fboxsep{0.15cm}
\noindent$\boxed{\text{Judgement: } \con e : \tau}$
\begin{align*}
  \textsc{T-Var1: }
    \dfrac
      {}
      {\conemp x : \tau} \Sigma(x) = \tau &&
  \textsc{T-Var2: }
    \dfrac
      {}
      {\Sigma; (x \mapsto \tau) \vdashe x : \tau}
\end{align*}
\alignspace
\begin{align*}
  \textsc{T-Unit: }
    \dfrac
      {}
      {\conemp () : 1} &&
  \textsc{T-Unit-Elm: }
    \dfrac
      {\begin{array}{ccc}
       \con e : 1 & \quad &
       \conp e' : \tau
       \end{array}}
      {\concup e' : \tau}
\end{align*}
\alignspace
\begin{align*}
  \textsc{T-Inl: }
    \dfrac
      {\con e : \tau}
      {\con \inl{e} : \tau + \tau'} &&
  \textsc{T-Inr: }
    \dfrac
      {\con e : \tau}
      {\con \inr{e} : \tau' + \tau}
\end{align*}
\alignspace
\begin{align*}
  \textsc{T-Prod: }
    \dfrac
      {\begin{array}{ccc}
       \con e_1 : \tau & \quad &
       \conp e_2 : \tau'
       \end{array}}
      {\concup (e_1, e_2) : \tau \times \tau'}
\end{align*}
\alignspace
\begin{align*}
  \textsc{T-App1: }
    \dfrac
      {\begin{array}{ccc}
       \con f : \tau \leftrightarrow \tau' & \quad &
       \conp e : \tau
       \end{array}}
      {\concup f~e : \tau'}
\end{align*}
\alignspace
\begin{align*}
  \textsc{T-App2: }
    \dfrac
      {\begin{array}{ccc}
       \con f : \tau_a \rightarrow \tau_f & \quad &
       \conemp e : \tau_a
       \end{array}}
      {\con f~e : \tau_f}
\end{align*}
\alignspace
\begin{align*}
  \textsc{T-PApp:}
    \dfrac
      {\con f : \forall \alpha.\tau_f}
      {\con f~\tau_a : \tau_f[\tau_a/\alpha]} &&
  \textsc{T-Let1: }
    \dfrac
      {\begin{array}{ccc}
       \con e_1 : \tau' & \quad &
       \Sigma; \Gamma', x : \tau' \vdashe e_2 : \tau
       \end{array}}
      {\concup \lett{x}{e_1}{e_2} : \tau}
\end{align*}
\alignspace
\begin{align*}
  \textsc{T-Let1St: }
    \dfrac
      {\begin{array}{ccc}
       \Sigma; \emptyset \vdash e_1 : \tau' & \quad &
       \Sigma, x : \tau' ; \Gamma, \vdashe e_2 : \tau
       \end{array}}
      {\con \lett{x}{e_1}{e_2} : \tau}
\end{align*}
\alignspace
\begin{align*}
  \textsc{T-Let2: }
    \dfrac
      {\begin{array}{ccc}
       \con e_1 : \tau' \times \tau'' & \quad &
       \Sigma; \Gamma', x : \tau', y : \tau'' \vdashe e_2 : \tau
       \end{array}}
      {\concup \lett{(x, y)}{e_1}{e_2} : \tau}
\end{align*}
\alignspace
\begin{align*}
  \textsc{T-Let2St: }
    \dfrac
      {\begin{array}{ccc}
       \Sigma; \emptyset \vdash e_1 : \tau' \times \tau'' & \quad &
       \Sigma, x : \tau', y : \tau''; \Gamma \vdashe e_2 : \tau
       \end{array}}
      {\con \lett{(x, y)}{e_1}{e_2} : \tau}
\end{align*}
\alignspace
\begin{align*}
  \textsc{T-Sum: }
    \dfrac
      {\begin{array}{ccccc}
       \con e_1 : \tau' + \tau'' & \quad &
       \Sigma; \Gamma', x : \tau' \vdashe e_2 : \tau & \quad &
       \Sigma; \Gamma', y : \tau'' \vdashe e_3 : \tau
       \end{array}}
      {\concup \caseof{e_1}{\inl{x}}{e_2}{\inr{y}}{e_3} : \tau}
\end{align*}
\alignspace
\begin{align*}
  \textsc{T-SumSt: }
    \dfrac
      {\begin{array}{ccccc}
       \Sigma; \emptyset \vdash e_1 : \tau' + \tau'' & \quad &
       \Sigma, x : \tau'; \Gamma \vdashe e_2 : \tau & \quad &
       \Sigma, y : \tau''; \Gamma \vdashe e_3 : \tau
       \end{array}}
      {\con \caseof{e_1}{\inl{x}}{e_2}{\inr{y}}{e_3} : \tau}
\end{align*}
\alignspace
\begin{align*}
  \textsc{T-Roll: }
    \dfrac
      {\con e : \tau' [\mu X . \tau' / X]}
      {\con \roll{\mu X . \tau'}{e} : \mu X . \tau'} &&
  \textsc{T-Unroll: }
    \dfrac
      {\con e : \mu X . \tau}
      {\con \unroll{\mu X . \tau}{e} : \tau' [\mu X . \tau / X]}
\end{align*}
\caption{Expression typing.}\label{fig:exprType}
\end{figure}

\subsection{Type Rules For Function Declarations.}

The type rules for function declarations are shown in Fig.~\ref{fig:funcType}.
Here \textsc{T-PFun} generalizes the type arguments, next \textsc{T-Fun1}
consumes the ancillary variables, and finally \textsc{T-Fun2} handles the last
dynamic variable by applying the expression typing.

\begin{figure}[t]
\setlength\fboxsep{0.15cm}
\noindent$\boxed{\text{Judgement: } \Sigma \vdashd d : \tau}$

\begin{align*}
  \textsc{T-Fun1: }
    \dfrac
      {\Sigma, x : \tau_a \vdashd f~v^+ = e : \tau_f}
      {\Sigma \vdashd f~x{:}\tau_a\ v^+ = e : \tau_a \rightarrow \tau_f} &&
  \textsc{T-Fun2: }
    \dfrac
      {\Sigma; (x \mapsto \tau) \vdashe e : \tau'}
      {\Sigma \vdashd f~x{:}\tau = e : \tau \leftrightarrow \tau'}
\end{align*}
\alignspace
\begin{align*}
  \textsc{T-PFun:}
    \dfrac
      {\Sigma \vdashd f~\beta^*~v^+ = e : \tau_f}
      {\Sigma \vdashd f~\alpha~\beta^*~v^+ = e : \forall \alpha.\tau_f} \alpha
      \notin \text{TV}(\Sigma)
\end{align*}

\caption{Function typing.}\label{fig:funcType}
\end{figure}

We implicitly assume that pointers to all defined functions are placed in the
static context $\Sigma$ as an initial step. For example, when typing an
expression $e$ in a program where a function $f~x = e$ is defined, and we have
been able to establish that $\Sigma \vdashd f~x = e : \tau \leftrightarrow
\tau'$ for some types $\tau, \tau'$, we assume that a variable $f : \tau
\leftrightarrow \tau'$ is placed in the static context in which we will type
$e$ ahead of time. This initial step amounts to a typing rule for the full
program.

Note that we write two very similar application rules \textsc{T-App1} and
\textsc{T-App2}. This discerns between function application of ancillary and
dynamic data, corresponding to the two different arrow types. In particular, as
shown in \textsc{T-App1}, application in the dynamic variable of a function is
only possible when that function is of type $\tau \leftrightarrow \tau'$, where
$\tau$ and $\tau'$ are non-arrow types: This specifically disallows
higher-order functions. Also note that the dynamic context must be empty for
application of the \textsc{T-App2} rule --- otherwise, it is treated as static
in the evaluation of $f$, and we have no guarantee how it is used.

\section{Recursive and Polymorphic Types}

The type system of \rfunc supports both recursive types as well as rank-1
parametrically polymorphic types. To support both of these, type variables,
which serve as holes that may be plugged by other types, are used.

For recursive types, we employ a standard treatment of iso-recursive types in
which explicit \textbf{roll} and \textbf{unroll} constructs are added to
witness the isomorphism between $\mu X. \tau$ and $\tau[\mu X. \tau/X]$ for a
given type $\tau$ (which, naturally, may contain the type variable $X$). For a
type $\tau$, we let $\text{TV}(\tau)$ denote the set of type variables that
appear free in $\tau$. For example, the type of lists of a given type $\tau$
can be expressed as the recursive type $\mu X. 1 + (\tau \times X)$, and
$\text{TV}(1 + (\tau \times X)) = \{X\}$ when the type $\tau$ contains no free
type variables. We define TV on contexts as $\text{TV}(\Sigma) = \{ v \in
\text{TV}(\tau) \mid x : \tau \in \Sigma \}$.

For polymorphism, we use an approach similar to System F, restricted to rank-1
polymorphism. In a polymorphic type system with rank-1 polymorphism, type
variables themselves cannot be instantiated with polymorphic types, but must be
instantiated with concrete types instead. While this approach is significantly
more restrictive than the full polymorphism of System F, it is expressive
enough that many practical polymorphic functions may be expressed (e.g.  ML and
Haskell both employ a form of rank-1 polymorphism based on the Hindley-Milner
type system~\cite{Milner:1978}), while being simple enough that type inference
is often both decidable and feasible in practice.

\section{Operational Semantics}\label{sec:semantics}

\begin{figure}[t]
\begin{align*}
  c ::= () \mid \inl{c} \mid \inr{c} \mid (c_1, c_2) \mid \roll{\tau}{c} \mid x
\end{align*}
\caption{Canonical forms.}\label{fig:canonical}
\end{figure}

We present a call-by-value big step operational semantics on expressions in
Fig.~\ref{math:semantics}, with canonical forms shown in
Fig.~\ref{fig:canonical}. As is customary with functional languages, we use
substitution (defined as usual by structural induction on expressions) to
associate free variables with values (canonical forms). Since the language does
not allow for values of a function type, we instead use an environment $p$ of
function definitions in order to perform computations in a context (such as a
program) with all defined functions.

\begin{figure}[ht]
\setlength\fboxsep{0.15cm}
\noindent$\boxed{\text{Judgement: } \pcon e \downarrow c}$

\begin{align*}
  \textsc{E-Unit: }
    \dfrac
      {}
      {\pcon () \downarrow ()} &&
  \textsc{E-Inl: }
    \dfrac
      {\pcon e \downarrow c}
      {\pcon \textbf{inl}(e) \downarrow \textbf{inl}(c)} &&
  \textsc{E-Inr: }
    \dfrac
      {\pcon e \downarrow c}
      {\pcon  \textbf{inr}(e) \downarrow \textbf{inr}(c)}
\end{align*}
\alignspace
\begin{align*}
  \textsc{E-Roll: }
    \dfrac
      {\pcon e \downarrow c}
      {\pcon \roll{\tau}{e} \downarrow \roll{\tau} c} &&
  \textsc{E-Unroll: }
    \dfrac
      {\pcon e \downarrow \roll{\tau}{c}}
      {\pcon \unroll{\tau}{e} \downarrow c}
\end{align*}
\alignspace
\begin{align*}
  \textsc{E-Prod: }
    \dfrac
      {\begin{array}{ccc}
       \pcon e_1 \downarrow c_1 & \quad &
       \pcon e_2 \downarrow c_2
       \end{array}}
      {\pcon (e_1, e_2) \downarrow (c_1, c_2)} &&
  \textsc{E-Let: }
    \dfrac
      {\begin{array}{ccc}
       \pcon e_1 \downarrow c_1 & \quad &
       \pcon e_2[c_1/x] \downarrow c
       \end{array}}
      {\pcon \textbf{let $x = e_1$ in $e_2$} \downarrow c}
\end{align*}
\alignspace
\begin{align*}
  \textsc{E-LetP: }
    \dfrac
      {\begin{array}{ccc}
       \pcon e_1 \downarrow (c_1,c_2) & \quad &
       \pcon e_2[c_1 / x, c_2 / y] \downarrow c
       \end{array}}
      {\pcon \textbf{let $(x,y) = e_1$ in $e_2$} \downarrow c}
\end{align*}
\alignspace
\begin{align*}
  \textsc{E-CaseL: }
    \dfrac
      {\begin{array}{ccc}
       \pcon e_1 \downarrow \inl{c_1} & \quad &
       \pcon e_2 [c_1 / x] \downarrow c
       \end{array}}
      {\pcon \caseof{e_1}{\inl{x}}{e_2}{\inr{y}}{e_3} \downarrow c}
\end{align*}
\alignspace
\begin{align*}
  \textsc{E-CaseR: }
    \dfrac
      {\begin{array}{ccccc}
       \pcon e_1 \downarrow \inr{c_1} & \quad &
       \pcon e_3 [c_1 / y] \downarrow c & \quad &
       %\pcon x \downarrow c'
       \end{array}}
      {\pcon \caseof{e_1}{\inl{x}}{e_2}{\inr{y}}{e_3} \downarrow c}c \notin
      \PLVal(e_2)% c \neq c'
\end{align*}
\alignspace
\begin{align*}
  \textsc{E-App: }
    \dfrac
      {\begin{array}{ccc}
       \pcon e_1 \downarrow c_1 \cdots \pcon e_n \downarrow c_n & \quad &
       \pcon e[c_1 / x_1,\ \cdots,\ c_n / x_n] \downarrow c
       \end{array}}
      {\pcon f\ \alpha_1 \cdots \alpha_m~e_1\ \cdots\ e_n \downarrow c}
      p(f) = f\ \alpha_1 \cdots \alpha_m ~ x_1 \cdots x_n = e
\end{align*}
\caption{Big step semantics of \rfunc.}\label{math:semantics}
\end{figure}

A common problem in reversible programming is to ensure that branching of
programs is done in such a way as to uniquely determine in the backward
direction which branch was taken in the forward direction. Since
case-expressions allow for such branching, we will need to define some rather
complicated machinery of \emph{leaf expressions}, \emph{possible leaf values},
and \emph{leaves} (the latter is similar to what is also used
in~\cite{YokoyamaAxelsenGlueck:2012:LNCS}) in order to give their semantics.

Say that an expression $e$ is a \emph{leaf expression} if it does not contain
any subexpression (including itself) of the form $\textbf{let $l = e_1$ in
$e_2$}$ or $\caseof{e_1}{\inl{x}}{e_2}{\inr{y}}{e_3}$; the collection of leaf
expressions form a set, $\text{LExpr}$. As the name suggests, a leaf expression
is an expression that can be considered as a \emph{leaf} of another expression.
The set of leaves of an expression $e$, denoted $\leaves(e)$, is defined in
Fig.~\ref{fig:leaves}.

\begin{figure}[tp]
\begin{align*}
  \leaves( () ) & = \left\{()\right\} \\
  \leaves((e_1, e_2)) & = \{ (e_1',e_2') \mid e_1' \in \leaves(e_1), \\
  & \phantom{= \{ (e_1',e_2') \mid}\ \ e_2' \in \leaves(e_2) \} \\
  \leaves(\inl{e}) & = \left\{ \inl{e'} \mid e' \in \leaves(e) \right\}\\
  \leaves(\inr{e}) & = \left\{ \inr{e'} \mid e' \in \leaves(e) \right\}\\
  \leaves(\roll{\tau}{e}) & = \left\{ \roll{\tau}{e'} \mid e' \in \leaves(e)
  \right\}\\
  \leaves(\textbf{let $l = e_1$ in $e_2$}) & = \leaves(e_2) \\
  \leaves(\caseof{e_1}{\inl{x}}{e_2}{\inr{y}}{e_3}) & = \leaves(e_2) \cup \leaves(e_3) \\
  \leaves(x) & = \{x\} \\
  \leaves(\unroll{\tau}{e}) & = \{\unroll{\tau}{e'} \mid e' \in \leaves(e) \}\\
  \leaves(f~e_1~\dots~e_n) & = \{f~e_1'~\dots~e_n' \mid e_i' \in \leaves(e_i) \}
\end{align*}
\caption{Definition function that computes the leaves of a program.}
\label{fig:leaves}
\end{figure}

The leaves of an expression are, in a sense, an abstract over-approximation of
its possible values, save for the fact that leaves may be leaf expressions
rather than mere canonical forms. We make this somewhat more concrete with the
definition of the \emph{possible leaf values} of an expression $e$, defined as:

\begin{align}
  \PLVal(e) = \{ e' \in \text{LExpr} \mid e'' \in \leaves(e), e' \rhd e'' \}
  \label{def:PLVal}
\end{align}

Where the relation $- \rhd -$ on leaf expressions is defined inductively as
(the symmetric closure\footnote{The symmetric closure of a binary relation on
a set is the smallest symmetric relation on that set which contains the
relation.} of):

\begin{align}
  () & \rhd () \nonumber \\
  (e_1, e_2) & \rhd (e_1', e_2')  & \text{if} \quad e_1 \rhd e_1'
  \text{ and } e_2 \rhd e_2' \nonumber \\
  \inl{e} & \rhd \inl{e'} & \text{if} \quad e \rhd e' \nonumber
  \label{def:rhd_relation} \\
  \inr{e} & \rhd \inr{e'} & \text{if} \quad e \rhd e' \\
  \roll{\tau}{e} & \rhd \roll{\tau}{e'} & \text{if} \quad e \rhd e' \nonumber \\
  e & \rhd x \nonumber \\
  e & \rhd f~e_1~\dots~e_n \nonumber \\
  e & \rhd \unroll{\tau}{e'} \nonumber
\end{align}

As such, the set $\PLVal(e)$ is the set of leaf expressions that can be
unified, in a certain sense, with a leaf of $e$. Since variables, function
applications, and unrolls do nothing to describe the syntactic form of possible
results, we define that these may be unified with \emph{any} expression. As
such, using $\PLVal(e)$ is somewhat conservative in that it may reject
definitions that are in fact reversible. Note also that $\PLVal(e)$
specifically includes all canonical forms that could be produced by $e$, since
all canonical forms are leaf expressions as well.

This way, if we can ensure that a canonical form $c$ produced by a branch in
a case-expression could not possibly have been produced by a \emph{previous}
branch in the case-expression, we know, in the backward direction, that $c$
must have been produced by the current branch. This is precisely the reason for
the side condition of $c \notin \PLVal(e_2)$ on \textsc{E-CaseR}, as this
conservatively guarantees that $c$ could not have been produced by the previous
branch.

It should be noted that for iterated functions\footnote{An iterated function
$f^n$ is a function $f$ composed with itself $n$ times.} this may add a
multiplicative execution overhead that is equal to the size of the data
structure. This effect has previously been shown in~\cite{Thomsen:2012:FDL},
where a \lstinline{plus} function over Peano numbers, which was linear
recursive over its input, actually had quadratic runtime.

It is immediate that should the side condition not hold for an expression $e'$,
no derivation is possible, and the expression does not evaluate to a value
entirely. In Sect.~\ref{sec:staticFMP}, we will look at exactly under which
conditions we can \emph{statically} guarantee that the side condition will hold
for every possible value of a function's domain.

We capture the conservative correctness of our definition of $\PLVal(e)$ with
respect to the operational semantics --- i.e. the property that any
canonical form $c$ arising from the evaluation of an expression $e$ will also
be ``predicted'' by $\PLVal$ in the sense that $c \in \PLVal(e)$ --- in the
following theorem:

\begin{theorem}
  If $p \vdash e \downarrow c$ then $c \in \PLVal(e)$.
\end{theorem}

\begin{proof}

  By induction on the structure of the derivation of $p \vdash e \downarrow c$.
  The proof is mostly straightforward: The case for \textsc{E-Unit} follows
  trivially, as do the cases for \textsc{E-Unroll} and \textsc{E-App} since
  leaves of $\unroll{\tau}{e}$ (respectively $f~e_1~\cdots~e_n$) are all of the
  form $\unroll{\tau}{e'}$ (respectively $f~e_1'~\cdots~e_n'$), and since $e''
  \rhd \unroll{\tau}{e'}$ (respectively $e'' \rhd f~e_1'~\cdots~e_n'$) for
  \emph{any} choice of $e''$, it follows that $\PLVal(\unroll{\tau}{e'}) =
  \PLVal(f~e_1~\cdots~e_n) = \text{LExpr}$. The cases for \textsc{E-Inl},
  \textsc{E-Inr}, \textsc{E-Roll}, and \textsc{E-Prod} all follow
  straightforwardly by induction, noting that $\PLVal(\inl{e}) = \{ \inl{e'}
  \mid e' \in \PLVal(e) \}$, and similarly for $\inr{e}$, $(e_1,e_2)$, and
  $\roll{\tau}{e}$. This leaves only the cases for \textbf{let} and
  \textbf{case} expressions, which follow using Lemma~\ref{thm:plval_subs}.

\end{proof}

\begin{lemma}\label{thm:plval_subs}
  For any expression $e$, variables $x_1, \dots, x_n$, and canonical forms
  $c_1, \dots, c_n$, we have $\PLVal(e[c_1/x_1, \dots, c_n/x_n]) \subseteq
  \PLVal(e)$.
\end{lemma}

\begin{proof}

This lemma follows straightforwardly by structural induction on $e$, noting
that it suffices to consider the case where $e$ is open with free variables
$x_1, \dots, x_n$, as it holds trivially when $e$ is closed (or when its free
variables are disjoint from $x_1, \dots, x_n$). With this lemma, showing the
case for e.g. \textsc{E-Let} is straightforward since $c \in
\PLVal(e_2[c_1/x])$ by induction, and since $\PLVal(e_2[c_1/x]) \subseteq
\PLVal(e_2)$ by this lemma, so $c \in \PLVal(e_2) =
\PLVal(\mathbf{let}~x=e_1~\mathbf{in}~e_2)$ by
$\leaves(\mathbf{let}~x=e_1~\mathbf{in}~e_2) = \leaves(e_2)$.

\end{proof}

\section{Reversibility}

Showing that the operational semantics are reversible amounts to showing that
they exhibit both forward and backward determinism. Showing forward
determinism is standard for any programming language (and holds
straightforwardly in \rfunc as well), but backward determinism is unique to
reversible programming languages. Before we proceed, we recall the usual
terminology of \emph{open} and \emph{closed} expressions: Say that an
expression $e$ is closed if it contains no free (unbound) variables, and open
otherwise.

Unlike imperative languages, where backward determinism is straightforwardly
expressed as a property of the reduction relation $\sigma \vdash c \downarrow
\sigma'$ where $\sigma$ is a store, backward determinism is somewhat more
difficult to express for functional languages, as the obvious analogue --- that
is, if $e \downarrow c$ and $e' \downarrow c$ then $e = e'$ --- is much too
restrictive (specifically, it is obviously \emph{not} satisfied in all but the
most trivial reversible functional languages). A more suitable notion turns out
to be a \emph{contextual} one, where rather than considering the reduction
behaviour of closed expressions in themselves, we consider the reduction
behaviour of canonical forms in a given \emph{context} (in the form of an open
expression) instead.

{
\renewcommand{\thetheorem}{\ref{thm:ctxbackwarddet}}
\begin{theorem}[Contextual Backwards Determinism]

  For all open expressions $e$ with free variables $x_1, \dots, x_n$, and all
  canonical forms $v_1, \dots, v_n$ and $w_1, \dots, w_n$, if $\pcon e[v_1 /
  x_1, \dots, v_n / x_n] \downarrow c$ and $\pcon e[w_1 / x_1, \dots, w_n /
  x_n] \downarrow c$ then $v_i = w_i$  for all $1 \le i \le n$.

\end{theorem}
\addtocounter{theorem}{-1}
}

\begin{proof}

The proof of this theorem follows by induction on the structure of $e$. For
brevity, we sometimes denote the set $x_1, \dots, x_n$ as $X$. The proof is
mostly straightforward:

\begin{itemize}

  \item Case $e = ()$. This follows immediately as there are no free variables
    in $e$.

  \item Case $e = x$. The only possible substitution is for the free variable
    $x$, and the only possible inference rule is \textsc{E-Var}. We have $x[v /
    x] \downarrow v$ and $x[w / x] \downarrow w$, so $v = w = c$.

  \item Case $e = \inl{e'}$. Follows from using the Induction Hypothesis on
    $e'$ and using the \textsc{E-Inl} rule to construct a derivation for $e$.
    The proof is identical for when $e = \inr{e'}$, when $e = \roll{\tau}{e'}$
    and when $e = \unroll{\tau}{e'}$.

  \item Case $e = (e', e'')$. The only possible rule for the derivation is
    \textsc{E-Prod}. First, we must consider which open terms occur in $e'$ and
    $e''$ respectively. We have some subset $Y = y_1, \dots, y_k \in X$, of
    open terms occurring in $e'$, and likewise, we have some subset $Z = z_1,
    \dots, z_l \in X$ occurring $e''$, such that $Y \cup Z = X$. Note that any
    $y_i$ or $z_i$ is simply an alias for some $x_j$, they are not fresh
    variables. By the Induction Hypothesis on $e'$ and $e''$ we prove it for
    both premises. Now, what about any $x_k \in Y \cap Z$? Any value $v_k$ has
    not been proven to be the same as $w_k$, but it certainly has to hold. But
    we have this by the inductive definition of substitution:

    \begin{align*}
      (e', e'') [c / x] = (e' [c / x], e'' [c / x])
    \end{align*}

  \item Case $e = \lett{x}{e'}{e''}$. We forego the argument about the
    partitioning of open terms as in the case before, but take note that it
    holds here as well. We first prove it for $e'$ by the Induction Hypothesis
    on $e'$.

    Now, we have by assumption that the expressions $(\lett{x}{e_1}{e_2})[v_1 /
    x_1, \dots, v_n / x_n]$ and $(\lett{x}{e_1}{e_2})[w_1 / x_1, \dots, w_n /
    x_n]$ evaluate to the same canonical value $c$. Thus the theorem holds for
    the only free substitution (of $x$) in $e''$, and combined with the
    Induction Hypothesis on $x_1, \dots, x_n$ it holds for $e$ as a whole.

  % \item Case $e = f~\alpha_1 \dots \alpha_m~e_1 \dots e_q$. We have that $p(f)
  %   = f~\alpha_1 \dots \alpha_m~y_1 \dots y_q = e'$. Straightforwardly by the
  %   Induction Hypothesis on each expression $e_1, \dots, e_q$, we have that
  %   $c_1' = c_1'', \dots, c_q' = c_q''$ where $e_k[v_1 / x_1, \dots, v_n / x_n]
  %   \downarrow c_k'$ and $e_k[w_1 / x_1, \dots, w_n / x_n] \downarrow c_k''$
  %   for $i \in 1 \dots q$. Now, by the Induction Hypothesis on $e'[c_1' / y_1,
  %   \dots c_q' / y_q]$, we have what we want.

  \item Case $e = f~\alpha_1 \dots \alpha_m~e_1 \dots e_q$. We have that $p(f)
    = f~\alpha_1 \dots \alpha_m~y_1 \dots y_q = e'$. There are $q+1$ premises
    and thus $q+1$ possible partitions of the open terms. Luckily, the theorem
    holds by simply applying the Induction Hypothesis on the $q$ first
    premises, followed by noticing that every other free variable in $e'$
    (variables $y_1, \dots, y_q$) must be the substituted for the same set of
    canonical values by the theorem assumption --- this paired with an
    application of the Induction Hypothesis gives us what we want.

  \item Case $e = \caseof{e'}{\inl{x}}{e''}{\inr{y}}{e'''}$. We may use two
    inference rules, based on the derivation of $e'$. Induction on $e'$ derives
    a value $c$ (where $e'$ contains some subset $Y = y_1, \dots, y_k \in X$ of
    free variables). The possible forms of the value $c$ are:

    \begin{itemize}

      \item $c' = \inl{x}$. By the Induction Hypothesis on $e''$ with the
        \textsc{E-CaseL} rule, we have what we want, with $e''$ evaluating to
        some $c'$, where $e''$ contains some set of open terms $A \in X$.

      \item $c' = \inr{y}$. By the Induction Hypothesis on $e'''$ with the
        \textsc{E-CaseR} rule, we have that the theorem holds for $e'''$ with
        open terms $B = a_1, \dots, a_l \in x_1, \dots, x_n$, such that $Y \cup
        B = X$, where $e'''$ evaluates to some $c''$.

        Both evaluation rules are over the same expression: This means the
        expression contain the same set of open terms $Y$ in the first premise
        and not necessarily the same set of open terms $A$ and $B$ in the
        second premise. By the Induction Hypothesis on $e'''$ we have
        contextual backwards determinism for the premise $e'''$, but there is
        no way of guaranteeing that the substitutions of $A$ and $B$ are
        equivalent, and they can both evaluate to the same closed term where
        $c' = c''$.

        At this point we invoke the side condition of \text{E-CaseR}, which
        specifically states that should a derivation using the \textsc{E-CaseR}
        inference rule be possible so that $c' = c''$, then the derivation of
        $e''$ to $c'$ must not exist.

    \end{itemize}

\end{itemize}

And we have covered all the cases.

\end{proof}

\begin{corollary}[Injectivity]
  For all functions $f$ and canonical forms $v,w$, if $\pcon f~v \downarrow c$
  and $\pcon f~w \downarrow c$ then $v = w$.
\end{corollary}

\begin{proof}
  Let $e$ be the open expression $f~x$ (with free variable $x$). Since
  $(f~x)[v/x] = f~v$ and $(f~x)[w/x] = f~w$, applying
  Theorem~\ref{thm:ctxbackwarddet} on $e$ yields precisely that if $\pcon f~v
  \downarrow c$ and $\pcon f~w \downarrow c$ then $v=w$.
\end{proof}

% \section{Type Soundness}
%
% \begin{itemize}
%
%   \item There are some standard exercises when presenting a type system over a
%     language to show that it is \emph{sound}, meaning that it rejects ill-typed
%     programs. This is often done by showing progress and preservation of a type
%     system. The idea is due to~\cite{Wright:1994}.
%
%   \item Progress says that a derivation cannot be stuck, e.g.~any non-diverging
%     computation will derive to some value eventually (it is safe.) It has to do
%     with the semantics.
%
%   \item Preservation says that if a term evaluates to a value, then the type of
%     of the term and the value are the same.
%
%   \item I could mention these, but with a big step evaluation strategy they are
%     not very interesting properties, as computations are not really modeled
%     (through a small-step semantic), we just expect expressions to evaluate to
%     values and not get stuck.
%
% \end{itemize}
%
% \section{r-Turing Completeness}
%
% \begin{itemize}
%
%   \item Define a r-Turing machine.
%
%   \item Show how an r-Turing machine can be simulated in \rfunc.
%
% \end{itemize}


%-------------------------------------------------------------------------------
% FIRST MATCH GUARANTEE
%-------------------------------------------------------------------------------

% !TEX root = main.tex

\chapter{Statically Checking the First Match Policy}\label{sec:staticFMP}

The side condition in the \textsc{E-CaseR} rule is essential when ensuring
reversibility of partial functions. The authors of the original \rfun paper
dubbed their very similar mechanism the \emph{First Match
Policy}~\cite{YokoyamaAxelsenGlueck:2012:LNCS}, a convention we will follow. It
is, unfortunately, a property that can only be fully guaranteed at run-time;
from Rice's theorem we know that all non-trivial semantic properties of
programs are undecidable~\cite{Rice:1953}. However, with the type system, we
can now in many cases resolve the first match policy statically.

Overall we may differentiate between two notions of divergence:

\begin{enumerate}

  \item A function may have inputs that do not result in program termination.

  \item A function may have inputs for which it does not have a defined
    behaviour; this could be the result of missing clauses.

\end{enumerate}

Note that the semantics of \rfunc dictate that if a computation does not
terminate in the forward direction, no input in the backwards direction may
evaluate to this input. Dual for backwards computations.

\begin{proof}

  Assume that termination is not a symmetric property. Consider some function
  $f~x_1 \dots x_n$ which diverges in the forward direction with applied values
  $c_1 \dots c_n$. Now, there exists an $f$ such that we should be able to find
  some input $i$ to the inverse function such that $f^{-1}~c_1 \dots i
  \downarrow c_n$ --- that is, the diverging input in the forward direction.
  Since the inverse direction converges, we have determined a result in the
  forward direction, which is a contradiction. A dual argument is valid in the
  backward direction.

\end{proof}

Termination analysis (1.) is not what we will detail here, but rather inputs
for which the function is not defined (2.). Because the first match policy is
enforced by the operational semantics, it follows that whenever an application
of the \textsc{Case-R} rule fails its side condition, the expression cannot be
derived, which by extension means the function is not defined for this input.

\begin{definition}

The domain of a function $f$ is the subset of the $n$-ary Cartesian product of
canonical values which are inhabitants of the type of each parameter, for which
$f$ evaluates to a closed term:

\begin{align}
  \dom (f~(x_1 : \tau_1) \dots (x_n : \tau_n)) =
  \{ (c_1, \dots, c_n) \mid c_1 \in \tau_1, \dots, c_2 \in \tau_n,
     f~c_1 \dots c_n \downarrow c \}
\end{align}

\end{definition}

Intuitively, some combination of inhabitants of the types $\tau_1 \dots \tau_n$
applied to $f$ might fail to evaluate to a closed term because of either (1.)
or (2.) (they are not in the domain of $f$.) We wish to investigate exactly
when we can or when we cannot \emph{guarantee} that a derivation of a
case-expression in $f$ is going to uphold the first match policy for \emph{all}
elements in the domain of $f$. To a certain degree this is reminiscent of
arguing for the \emph{totality} of the function, up to termination. Contrary to
the argument for termination, this property of a first match policy guarantee
is not symmetric: More specifically, a function $f$ and its inverse function
$f^{-1}$ might not accommodate this restricted notion of totality on their
respective domains, although it is certainly possible.

This analysis is only possible due to the type system, as we define the domain
based directly on the type of the function --- it formally hints us at the
underlying sets of values which occur in case-expressions, something which is
only possible when terms are given types.

\section{Benefits of First Match Policy Assertions}

A pitfall with a language using a first-match policy to guarantee reversibility
is the added cost of making sure the first match policy is being met at
runtime. It establishes a need to be attentive when writing programs to not
write case-expressions whose side condition check adds a multiplicative
overhead to the function's run time, potentially increasing the asymptotic
complexity of the algorithm. This may for example occur when the side condition
invocation involves traversing the full structure of the input, making each
iteration linear. With a first match policy assertion provided, the side
condition check may be substituted with a (hopefully) simpler assertion,
increasing computational efficiency.

As an additional benefit, an assertion which is expressive can be seen as an
enhancement of clarity of the behaviour of the case-expression to which it
belongs. It also simplifies the process of inverse evaluation of that case
expression by adding additional structure to the form of the left and right
arm, as will be discussed in Sect.~\ref{sec:running_backwards}.

\section{Adding First Match Policy Assertions to the Formal Language}

The first match policy is ultimately enforced with the addition of the side
condition of the \textsc{Case-R} type rule. The issue is that it compels a
computation of the PLVal set on $e_1$ at every application of this rule. The
present ambition is to express an alternative method of ensuring reversibility
of branching. We introduce two new syntactic constructs to \rfunc: \emph{Safe}
and \emph{unsafe} case-expressions. These are presented in the grammar in
Fig.~\ref{fig:new_cases}.

\begin{figure}[t]
\begin{tabular}{p{.6\textwidth}l}
  $e ::=$ \\
  $\qquad \quad \vdots$ \\
  $\hspace{1.7em}\mid \caseofs{e}{\inl{x}}{e}{\inr{y}}{e}{e}$ & \text{Safe case-expression} \\
  $\hspace{1.7em}\mid \caseofu{e}{\inl{x}}{e}{\inr{y}}{e}$ & \text{Unsafe case-expression} \\
\end{tabular}
\caption{New case-expression syntax}\label{fig:new_cases}
\end{figure}

\paragraph{Safe case-expressions}

A safe case-expression is augmented with a \emph{safe exit assertion}. Consider
an expression $\caseofs{e_1}{x}{e_2}{y}{e_3}{e_{out}}$. It omits the first
match policy check and instead checks the resulting value $c$ of the forward
evaluation of the case-expression against a static predicate $e_{out}$ at run
time (that is, an expression which evaluates to a Boolean type $1 + 1$.) Since
it is determined by the syntax of a case-expression that the expression $e_1$
cased over is of a binary sum type, it suffices to define a single predicate
which should always hold after $e_2$ has been evaluated and by implication
always \emph{not} hold when $e_3$ has been evaluated. The safe exit assertions
we propose here preserve reversibility much like Janus --- Janus includes
conditionals and loops augmented with an additional exit assertion, making
its control flow constructs completely symmetric. Inverse evaluation of
conditionals and loops in Janus is then as simple as exchanging the entry and
exit assertions in the backwards direction (while also reversing the statement
blocks contained in these.)

We require a new type rule for the case-expression to support safe case
expressions. It is highly similar to the preexisting \textsc{T-Sum} rule, but
with the added criteria that the type of the safe exit assertion is $1 + 1$. We
introduce a variable $out$ in the assertion, which assumes the value of the
result of the case-expression --- granting the programmer the possibility to
make the exit assertion depend on the value produced by the case-expression.
The new type rule can be seen in Fig.~\ref{math:case_typing}.

\begin{figure}[t]
\begin{align*}
  \textsc{T-SumU: }
    \dfrac
      {\begin{array}{ccccc}
       \con e_1 : \tau' + \tau'' & \quad &
       \Sigma; \Gamma', x : \tau' \vdashe e_2 : \tau \\
       \Sigma; \Gamma', y : \tau'' \vdashe e_3 : \tau & &
       \Sigma; out : \tau \vdash e_a : 1 + 1
       \end{array}}
       {\concup \caseofs{e_1}{\inl{x}}{e_2}{\inr{y}}{e_3}{e_a} : \tau}
\end{align*}
\caption{New typing rule for a case-expression with an exit
assertion.}\label{math:case_typing}
\end{figure}

We also need two new evaluation rules for the big step semantic. These differ
from the conventional \textsc{E-CaseR} and \text{E-CaseL} rules in that we
enforce the added assertion through a premise which demands to be evaluated to
either \textbf{inr}(()) or \textbf{inl}(()), depending on which branch was
taken. Meanwhile, the side condition is removed. Note that with an exit
assertion, matching on a left branch is stricter than the conventional
behaviour exhibited by \textsc{E-CaseL} as it might fail.

\paragraph{Unsafe case-expressions}

An \emph{unsafe case-expression} is augmented with the \textbf{unsafe} keyword.
The result of the forward evaluation of a case-expression is not subjected to
any validation check when the right branch is taken. This means it allows more
programs, including programs which are not injective. It is thus only
reversible if the correct branch can be discerned in the backwards direction
unanimously by its syntactic form. Therefore we do not allow unsafe
case-expressions into the syntax directly. Rather, we only allow statically
generated unsafe case-expressions. No new type rule is needed to eliminate
unsafe case-expressions as typechecking occurs before static analysis.
Operationally, the only difference in behaviour from the \textsc{E-CaseR} rule
is that the side condition is omitted. Unsafe case-expression are inserted into
the program by the static analysis we will present in the next
Sect.~\ref{subsec:fmp_guarantee}.

The operational rules for safe and unsafe case-expressions are shown in
Fig.~\ref{math:assertion_semantics}. As already hinted at, the unsafe
case-expression strictly speaking is not actually reversible, so we cannot
prove Theorem~\ref{thm:ctxbackwarddet} if we adopt them as an expression form
directly --- but we will informally say that in any circumstance where we
\emph{do} insert them, they are safe enough to maintain reversibility.

We still want to prove reversibility for case-expressions with safe exit
assertions.  Below, we make sure that they actually maintain the reversibility
of \rfunc by extending the proof for Theorem~\ref{thm:ctxbackwarddet}.

\begin{figure}[ht]
\begin{align*}
  \textsc{E-CaseSL: }
    \dfrac
      {\begin{array}{cccccc}
       \pcon e_1 \downarrow \inl{c_1} & \quad &
       \pcon e_2 [c_1 / x] \downarrow c & \quad &
       \pcon e_a [c / out] \downarrow \inl{()}
       \end{array}}
      {\pcon \caseofs{e_1}{\inl{x}}{e_2}{\inr{y}}{e_3}{e_a} \downarrow c}
\end{align*}
\alignspace
\begin{align*}
  \textsc{E-CaseSR: }
    \dfrac
      {\begin{array}{cccccc}
       \pcon e_1 \downarrow \inr{c_1} & \quad &
       \pcon e_3 [c_1 / y] \downarrow c & \quad &
       \pcon e_a [c / out] \downarrow \inr{()}
       \end{array}}
      {\pcon \caseofs{e_1}{\inl{x}}{e_2}{\inr{y}}{e_3}{e_a} \downarrow c}
\end{align*}
\alignspace
\begin{align*}
  \textsc{E-CaseUL: }
    \dfrac
      {\begin{array}{ccccc}
       \pcon e_1 \downarrow \inl{c_1} & \quad &
       \pcon e_2 [c_1 / x] \downarrow c & \quad &
       \end{array}}
      {\pcon \caseofu{e_1}{\inl{x}}{e_2}{\inr{y}}{e_3} \downarrow c}
\end{align*}
\alignspace
\begin{align*}
  \textsc{E-CaseUR: }
    \dfrac
      {\begin{array}{ccccc}
       \pcon e_1 \downarrow \inr{c_1} & \quad &
       \pcon e_3 [c_1 / y] \downarrow c & \quad &
       \end{array}}
      {\pcon \caseofu{e_1}{\inl{x}}{e_2}{\inr{y}}{e_3} \downarrow c}
\end{align*}
\caption{New big step semantic rules for case-expressions with
assertions.}\label{math:assertion_semantics}
\end{figure}

\begin{theorem}[Continued]\label{thm:ctxbackwarddet}

\end{theorem}

\begin{proof}

  We first reassure ourselves that the proof for the previously shown forms of
  $e$ remain valid, which is given to us by Induction Hypothesis directly. We
  now consider the newly added case for $e$, which is missing to complete the
  proof:

  \begin{itemize}

    \item Case $e = \caseofs{e_1}{\inl{x'}}{e_2}{\inr{y'}}{e_3}{e_a}$

      There are two possible derivations of $e$:

  \begin{itemize}

      \item By \textsc{E-CaseSL}. By Induction Hypothesis on $e_1$ we get a
        contextually deterministic derivation for $e_1$, evaluating to $c_1$.
        By application of the Induction Hypothesis on $e_2 [c_1 / x]$ we derive
        a value $c' = c'' = c$. Now, as the semantics are deterministic, by the
        Induction Hypothesis on $e_a [c / out]$, we can only ever derive the
        same value $c_a$ by the Theorem assumption --- specifically $\inl{()}$
        if we are going to infer the conclusion by the \textsc{E-CaseSL} rule.

      \item By \textsc{E-CaseSR}. The proof is analogous to the previous case.

    \end{itemize}

  \end{itemize}

  And we are done.

\end{proof}

\section{First Match Policy over Open Terms}\label{subsec:fmp_guarantee}

In the following two sections we discuss the possibility of asserting that an
unsafe static assertion may be defined for a case-expression.

Intuitively, when the range of a function call is well-defined (typed), and all
the leaves are disjoint, it is clear that any evaluated term will not match any
other leaf. For example, the following function performs a transformation on a
sum term. It is immediately obvious that its leaves are disjoint; it either
evaluates to $\inl{\cdot}$ or to $\inr{\cdot}$:

\begin{rfuncode}
f $x : 1 + \tau$ = case $x$ of
                inl(()) => inr(())
                inr($y$) => inl($y$)
\end{rfuncode}

In Sect.~\ref{sec:semantics}, when we defined the operational semantics (cf.
Fig.~\ref{math:semantics}), the first match policy was upheld for a case
expression by checking that the closed term $c$ of the evaluation of the second
branch ($\inr{\cdot}$) could not be a possible leaf value of the first branch
($\inl{\cdot}$). However, the above example includes an open term that is
defined over $y$.

Given the existing definition of the unification relation $- \rhd -$
(Def.~\ref{def:rhd_relation}), this is actually easy to alleviate with regards
to the static analysis. $- \rhd -$ has already been defined to take \emph{any}
term, both open and closed terms. Thus, in the general case, all we have to
ensure is that all leaves of either branch do not have a possible value in the
other branch.

Said otherwise, we wish to cross compare the sets of possible leaf expressions
of each branch. For a case-expression $\caseof{e_1}{x}{e_2}{y}{e_3}$, $e_2$ and
$e_3$ each form a set of leaves, and by taking the Cartesian product of these
sets, we can see if any leaf expressions unify pair-wise.  This can all be
described as:

\begin{align}
  \{ (l_2, l_3) \mid l_2 \in \leaves(e_2)
                   , l_3 \in \leaves(e_3)
                   , l_2 \rhd l_3 \}
  = \emptyset
  \label{eq:leaves_ortho}
\end{align}

Because $- \rhd -$ is a symmetric relation, we may interchange the operands and
still describe the same thing. Obviously, this static analysis is quite
restricted because most terms will be open, and open terms unify very broadly.

\section{First Match Policy over Closed Terms}

Yet, we have not incorporated the type system into our analysis. We now
investigate an alternative method --- concretely examining the domain of a
function we are trying to write a first match policy guarantee for. This has a
couple of benefits: Critically, we wish to avoid how the static leaves sets of
each branch in a case-expression often will contain open terms (as these unify
with anything, \emph{even though} the term might always have a simple,
predictable form which unifies well.)

The principle behind exhausting the domain is comparing the unions of possible
leaf values of each branch for a case-expression for the complete domain of the
function. Thus, the leaves sets being compared will potentially be much more
populous, \emph{but} they will exclusively consist of closed terms, which in
the end will more truthfully reflect how the function may be applied. In the
following, let the case-expression $e_c = \caseof{e_1}{x}{e_2}{y}{e_3}$ be a
subexpression of $e$ for the function $f~(x_1: \tau_1) \dots (x_n: \tau_n) =
e$.

We respectively define the union of the left and right leaves sets for $e_c$ as
Eq.~\ref{def:leavesl} and Eq.~\ref{def:leavesr}. $e_2'$ is derived from $e_2$
with substitutions occurring as for the application of $f$ with the values
$c_1, \dots, c_n$ up until a case-rule is to be applied to $e_c$ (by the
operational semantics.) Equivalently for $e_3'$ with $e_3$. We then adopt a
notion of comparability between leaf sets similar to how it was defined over
open terms in Eq.~\ref{def:closed_comparing}.

\begin{align}
  \leaves_l = \bigcup_{\substack{(c_1, \dots, c_n) \\ \in \dom{(f)}}} \leaves(e_2')
  \label{def:leavesl} \\
  \leaves_r = \bigcup_{\substack{(c_1, \dots, c_n) \\ \in \dom{(f)}}} \leaves(e_3')
  \label{def:leavesr}
\end{align}

\begin{align}
  \{ (l_2, l_3) \mid l_2 \in \leaves_l
                   , l_3 \in \leaves_r
                   , l_2 \rhd l_3 \}
  = \emptyset
  \label{def:closed_comparing}
\end{align}

The method is obviously only tangible when it converges, which requires two
characteristics of the function: for the domain of the function to be finite
and for the function to terminate on any inhabitant of its domain. We require
the domain to be finite as we actually want to compute the full set of possible
leaves, and with an infinite domain, we have to try infinitely many inputs. We
require that any computation terminates as otherwise there is a possibility
that we get stuck computing the leaves for the same instance of a function
application indefinitely. These two requirements are not mutually inclusive, as
can be shown by an example:

\begin{align*}
  f~(x:1) = \lett{y}{f~x}{y}
\end{align*}

It is immediate that the domain of a function $f$ is finite iff.~none of $f$'s
parameters comprise a recursive type. Note that a parameter of a polymorphic
type is general enough to be instantiated as a recursive type and is also
prohibited. This property is immediate from the fact that we can construct a
recursive value of an infinitely long chain of nested $\roll{\mu X .
\tau}{\cdot}$ terms of type $\mu X . \tau$. Any other type must necessarily be
constructed by values of strictly decreasing constituent types with $1$ as the
bottom type. Even function applications in $f$ of functions which return forms
of recursive types can only take on a finite set of values if the domain of $f$
is finite due to referential transparency.

Deciding if a function will terminate on any given input requires
\emph{termination analysis}. In general this property is described as the
halting problem, a famous undecidable problem~\cite{Turing:1937}. A lot of
effort has been spent on statically guaranteeing termination for an increasing
class of functions in all paradigms of programming. These often exploit a
static hint of guaranteed decreases of input size, using term
rewriting~\cite{Giesl:2006} or calling context graphs~\cite{Manolios:2006}.
Since we forbid infinite domains, we do not expect functions not to terminate,
so we may get away with a much simpler analysis.

A method that is conservative but easily proven to be correct is constructing a
directed \emph{computation graph}. It is defined as follows: Each vertex is a
function name $f,g, \dots$ and each edge $f \rightarrow g, \dots$ between
vertices $f$ and $g$ implies there is a function application of $g$ in the body
of $f$. For each function $m$ we wish to analyze termination for, we look at
the reachability relation of every function that $m$ is related to to see if
they are reflexive. This is equivalent to checking for cycles in the
computation graph. If no cycles exist, termination is guaranteed.

Finally, both of the aforementioned analyses can be performed statically,
making them a relatively simple but attainable practice.

%\paragraph{Notes:}
%
%Because the PLVal set is overly conservative it restricts the possibility for a
%First match policy guarantee. One cause of conservativeness is due to the
%unification relation on variables and function applications, which does not
%inspect the types of these constructs before concluding that any expression may
%be unified with them. To remedy this we introduce unification of types as the
%following:
%
%\begin{align*}
%  1 & \rhd 1 \\
%  \tau_1 \times \tau_2 & \rhd \tau_1' \times \tau_2
%    & \text{if} \quad \tau_1 \rhd \tau_1' \vee \tau_2 \rhd \tau_2' \\
%  \tau & \rhd \tau_1 + \cdots + \tau_n
%    & \text{if} \quad \exists 0 \le i \le n .~\tau \rhd \tau_i \\
%  \mu X . \tau & \rhd \mu X . \tau' & \text{if} \quad \tau \rhd \tau' \\
%  \tau & \rhd X \\
%\end{align*}
%
%Then we can extend the variable and function application clauses in the
%definition of $\PLVal(\cdot)$:
%
%\begin{align*}
%  e & \rhd x & \text{if} \quad e : \tau_e,~x : \tau_x,~\tau_e \rhd \tau_x \\
%  e & \rhd f~e_1~\dots~e_n &
%    \text{if} \quad e : \tau_e,~f : \tau \leftrightarrow \tau_f,~\tau_e \rhd \tau_f \\
%\end{align*}
%
%Knowing the types is literally zero helpful though, as every leaf expression
%\emph{must} have the same type for the function to be well types.


%-------------------------------------------------------------------------------
% INVERSE
%-------------------------------------------------------------------------------

\chapter{Evaluating Backwards}\label{sec:running_backwards}

For any computable function $f : X \rightarrow Y$ and $x$ such that $f(x) = y$,
we can derive the inverse function $f' : Y \rightarrow X$ such that $f'(y) =
x$, by simply trying every possible $x \in X$ for $f$ until we find one for
which $f$ computes $y$. This works as long as $f$ is decidable and is known as
the Generate-and-test method~\cite{Mccarthy:1956}. It is inherently inefficient
though, as we need to enumerate all possible values of $x \in X$.
Alternatively, any program can be made reversible by embedding a trace of
information that would otherwise be destroyed, restoring enough information so
as to make any computation injective. A Landauer embedding turns a
deterministic TM into a two-tape deterministic RTM and the Bennett method turns
a TM into a three-tape RTM, both describing the same
problem~\cite{AxelsenGlueck:2011:FoSSaCS, Bennett:1973}. Predictably however,
multi-tape RTMs converted to 1-tape RTM, are less efficient~\cite{Axelsen:2011}
than their original counterparts.

The strength of reversible languages lies in their design which enforces
injectivity of their computable functions. In the context of reversible
languages, a \emph{program inverter} writes the inverse program $p^{-1}$ from a
program $p$. Program inversion is \emph{local} if it requires no global
analysis to perform (syntactic constructs can be inverted locally) and this is
usually shown by a set of inverters $\mathcal{I}_x, \dots, \mathcal{I}_z$, one
for each syntactic domain $x, \dots, z$ of the language's grammar. As an
example, if we were to present program inversion for \rfunc, it would be
natural to present a program inverter $\mathcal{I}_p$, a function inverter
$\mathcal{I}_f$ and an expression inverter $\mathcal{I}_e$. A pleasant property
of program inversion is maintaining the program size and asymptotic complexity.

Unlike some structural reversible languages like Janus, the semantics of \rfunc
are not trivially locally reversed --- the central issue when presenting a
program inverter for a language like \rfunc is figuring out how to write
deterministic control flow for the inverse program. Writing it requires
``unification'' analysis of the leaves of the whole function structure, which
is difficult.

General methods have been proposed (some of which were later used to define a
program inverter for \rfun) for writing program inverters for reversible
functional languages~\cite{Gluck:2003, Kawabe:2005}, but the existence of both
an equality/duplication operator and constructor terms as atoms is assumed,
both of which are omitted from \rfunc.

Instead of presenting a program inverter, inverse evaluation for \rfunc is
going to rely on dedicated big step inverse semantics, defined over the
original program's syntax. Although evaluating inverse function calls in Janus
most often is achieved by calling the inverse function produced by the Janus
program inverter directly, big step inverse semantics exist for Janus as
well~\cite{Paolini:2018}.

An interesting observation is that program inversion as a method for
reversibility is less expressive than the inverse semantics we will present in
Sect.~\ref{sec:inverse_semantics}. Specifically, we can accommodate duplication
of data (in the forward direction) in the inverse semantics by the
\textsc{I-Bind2} rule. Meanwhile, the type system prohibits us from writing the
inverse program of a function which duplicates data, as we are lacking native
support for the equality/duplication operator. See:

\begin{align*}
  \mathcal{I}_f \llbracket (x:\tau) = (x, x) \rrbracket \defeq
  f^{-1}~(x': \tau \times \tau) = \mathcal{I}_e \llbracket x' \rrbracket
\end{align*}

How would we translate $\mathcal{I}_e \llbracket x' \rrbracket$? The forward
program is legal --- we adopted a relevant type system precisely because we
embraced the partiality of functions, so the forward function may be defined
while the inverse function may not. Now, the closest program we can produce is:

\begin{align*}
  \mathcal{I}_f \llbracket (x:\tau) = (x, x) \rrbracket \defeq
  f^{-1}~(x': \tau \times \tau) = \lett{(x, y)}{x'}{x}
\end{align*}

Wherein we attempt to destroy $y$ after decomposing the product, as they should
be equal anyway --- however we may never implicitly assume that the two values
in the decomposed product should be equal, so the program is not well-formed.
We say that \rfunc is not \emph{closed under program
inversion}~\cite{Kaarsgaard:2017} when we cannot produce the inverse program
from the language's grammar directly.

\section{Supporting Inverse Function Application}

Naturally, the inverse application of a function requires the full extent of
information which was returned in the forward direction, and since ancillae
parameters remain constant across application, the same set of values is
supplied for ancillae parameters in the forward and backward direction. The
relationship between a function $f$ and its inverse can be described as:

\begin{align}
  f~a_1 \dots a_n~x = y
  \Longleftrightarrow f^{-1}~a_1 \dots a_n~y = x
\end{align}

For typing of a program when inverse function applications are supported, we
may deduce the type of the inverse function from the static context $\Sigma$,
which contains the usual set of statically declared functions. The inverse
function $f^{-1}$ mirrors the type of $f$ but with the type of the dynamic
parameter $\tau_n$ and the return type $\tau_{out}$ swapped around the
bidirectional arrow. For example, if a program contains the following function
$f$, we can deduce the type of $f^{-1}$:

\begin{align}
  f : \tau_1
    \rightarrow \dots
    \rightarrow \tau_{n-1}
    \rightarrow \tau_n
    \leftrightarrow \tau_{out}
  \Longleftrightarrow~f^{-1}: \tau_1
    \rightarrow \dots
    \rightarrow \tau_{n-1}
    \rightarrow \tau_{out}
    \leftrightarrow \tau_n
  \label{def:inv_relationship}
\end{align}

We introduce a new piece of syntax to support inverse applications. We say that
to apply the inverse of a function, the programmer should append the original
function name with an exclamation mark. So in the grammar an application
$f^{-1}~a_1 \dots a_n$ is denoted as $f!~a_1 \dots a_n$. The addition to the
grammar is highlighted in Fig.~\ref{fig:inverse_call_grammar}.

\begin{figure}[t]
\begin{tabular}{p{.6\textwidth}l}
  $e ::=$ \\
  $\qquad \quad \vdots$ \\
  $\hspace{1.7em}\mid f!~\alpha^*~e^+$ & \text{Inverse function call} \\
\end{tabular}
\caption{Added inverse function call to expressions in the
grammar.}\label{fig:inverse_call_grammar}
\end{figure}

\subsection{Typing and Evaluation of Inverse Function
Applications}\label{sec:inverse_typing_evaluation}

Typing of an inverse application is identical to a regular application, with
the signature of the inverse function stored in the static context as seen in
Eq.~\ref{def:inv_relationship}.

The big step rule for  an inverse function application uses the inverse
semantics defined in Sec.~\ref{sec:inverse_semantics}, with the original
function stored in $p$ to produce a canonical term, and is presented in
Fig.~\ref{fig:inverse_call_evaluation}.

\begin{figure}[htp]
\begin{align*}
\textsc{E-InvApp: }
  \dfrac
    {\begin{array}{ll}
        p(f) = \alpha_1 \dots \alpha_m~x_1 \dots x_n = e &
        \pcon e_1 \downarrow c_1 \cdots \pcon e_n \downarrow c_n \\
        p; \emptyset \vdash^{-1} e[c_1 / x_1, \dots, c_{n-1} / x_{n-1}], c_n \leadsto \sigma &
        \sigma(x_n) = c
     \end{array}}
     {\pcon f!~\alpha_1 \dots \alpha_m~e_1 \dots e_n \downarrow c}
\end{align*}
\caption{New big step rules related to inverse application.}\label{fig:inverse_call_evaluation}
\end{figure}

\section{Inverse Semantics}\label{sec:inverse_semantics}

We now present the inverse big step semantics of \rfunc. The semantics should
satisfy that, given a function $f$ with body $e$, which is to be run in
inverse, and a value $c_{out}$, which is the dynamic input to $f^{-1}$, we may
derive a value $c_{in}$ which was supplied as the dynamic input to $f$ so
that $f~e_1 \dots e_n~c_{in} = c_{out}$.

Using substitution to assign values to variables in the backwards direction can
not be achieved directly as for the forward direction, as the dynamic input in
the backward direction does not have a variable name we can substitute with.

Instead, for the inverse semantics, we keep a store $\sigma$ with mappings of
the variables to values we have thus far inferred. Continuously throughout
inverse evaluation we keep a free value $c$ which we want to \emph{bind} over
an expression. The loose idea is that whenever we happen upon a variable name $x$
while attempting to bind $c$, we know that $x$ must have been bound to $c$ in
the forward direction. We define a binding form as:

\begin{align*}
  \langle e, c \rangle \sigma =
  \begin{cases}
    \sigma[x \mapsto c] &\text{If $e \equiv x$} \\
    \sigma              &\text{If $e \equiv c \equiv ()$} \\
    \sigma \cup \sigma' &\text{If $e \equiv x$ and $\sigma(x) = e'$ and
                               $\langle e', c \rangle \sigma = \sigma'$} \\
    \sigma \cup \sigma' &\text{If $e \equiv \inl{e'}$ and $c \equiv \inl{c'}$ and
                               $\langle e', c' \rangle = \sigma'$} \\
                        &\vdots \\
    \bot                &\text{Undefined otherwise}
  \end{cases}
\end{align*}

Thus, the form of $e$ should match the canonical value $c$, which ``retracts''
$c$ until a variable name is bound. The variable name may have been encountered
before during inverse evaluation due to relevance. We require that if an
attempt is made to map a variable which already exists in $\sigma$, the value
of the previous binding and the free value must be the same. The end result is
that amongst the values stored in $\sigma$, the input to the forward
application is assuredly mapped to the name of the dynamic forward parameter.

The first free value we attempt to bind is the value $c_{out}$ applied as the
dynamic input to the inverse application, and we bind it against the function
body $e$. We informally review a motivating example of inverse evaluation:

\begin{align}
  f~(x:\tau) = \lett{y}{()}{(y, x)}
  \label{prog:inverse_example}
\end{align}

The input to $f^{-1}$ must be a product of type $1 \times \tau$ (assume $\tau$
is generic). Say we evaluate $f^{-1}~((), ())$. This gives us an initial free
value $c = ((), ())$. We will look to bind $c$ to something. The result of
evaluating a let-expression in the forward direction is an evaluation of $e_2$
with $e_1$ substituted for $y$, so we first want to inspect $e_2$ in the
backward direction. Here we see that $e_2 = (x, y)$. By decomposing the
products against the free value, We attain two binding forms: $\langle x, ()
\rangle \sigma = \sigma'$ and $\langle y, () \rangle \sigma' = \sigma''$. Thus
far, we have that $\sigma[x \mapsto (), y \mapsto ()] \Leftrightarrow
\sigma''$.

Now we move to inverse evaluate $y = ()$. We know that the let-expression
resulted in an assignment to $y$ in the forward direction and therefore look
at how $y$ was constructed. Remember that $y$ is in $\sigma$. We indicate that
$y$ is the new free value we wish to bind (which will immediately be fetched
from the store, giving us that () actually is the free value) and $e_1$ is the
expression we attempt to bind $y$ against. In this case, we will obtain the
binding form $\langle (), () \rangle \sigma$ which does not change $\sigma$.
Note that the type system ensure that the value for $y$ in most cases will
follow the syntactic form of what is given as the inverse application input.
Since we are done, we should have obtained a value for the dynamic parameter
$x$ which we can directly read from the $\sigma$.

Most of the inverse semantics are presented in
Fig.~\ref{fig:inverse_semantics}. Because we have a wide array of possible case
expression forms, we present the inverse semantics of case-expressions on their
own in Fig.~\ref{fig:inverse_semantics_cases}. In the following we flesh out
the meaning behind some of the more convoluted rules.

\begin{figure}[ht!]
\setlength\fboxsep{0.15cm}
\noindent$\boxed{\text{Judgement: } \storeinv e, c \leadsto \sigma'}$

\begin{align*}
\textsc{I-Bind: }
  \dfrac
    {}
    {\storeinv x, c \leadsto \sigma, x \mapsto c} &&
\textsc{I-Bind2: }
  \dfrac
    {}
    {p; \sigma, x \mapsto c_2 \vdash^{-1} x, c_1 \leadsto \sigma, x \mapsto c}
    c_1 = c_2
\end{align*}
\begin{align*}
\textsc{I-Unit: }
  \dfrac
    {}
    {\storeinv (), () \leadsto \sigma} &&
\textsc{I-Prod: }
  \dfrac
    {\begin{array}{cc}
       \storeinv e_1, c_1 \leadsto \sigma' &
       p; \sigma' \vdash^{-1} e_2, c_2 \leadsto \sigma''
     \end{array}}
    {\storeinv (e_1, e_2), (c_1, c_2) \leadsto \sigma''}
\end{align*}
\begin{align*}
\textsc{I-Inl: }
  \dfrac
    {\storeinv e, c \leadsto \sigma'}
    {\storeinv \inl{e}, \inl{c} \leadsto \sigma'} &&
\textsc{I-Inr: }
  \dfrac
    {\storeinv e, c \leadsto \sigma'}
    {\storeinv \inr{e}, \inr{c} \leadsto \sigma'}
\end{align*}
\begin{align*}
\textsc{I-Var1: }
  \dfrac
    {\storeinv e, c \leadsto \sigma'}
    {p; \sigma, x \mapsto c \vdash^{-1} e, x \leadsto \sigma'}
    \sigma(x) = c &&
\textsc{I-Var2: }
  \dfrac
    {}
    {\storeinv e, x \leadsto \sigma}
    x \not\in \sigma
\end{align*}
\begin{align*}
\textsc{I-Roll: }
  \dfrac
    {\storeinv e, c \leadsto \sigma'}
    {\storeinv \roll{\tau}{e}, \roll{\tau}{c} \leadsto \sigma'} &&
\textsc{I-Unroll: }
  \dfrac
    {\storeinv e, \roll{\tau}{c} \leadsto \sigma'}
    {\storeinv \unroll{\tau}{e}, c \leadsto \sigma'}
\end{align*}
\begin{align*}
\textsc{I-Let: }
  \dfrac
    {\begin{array}{cc}
       \storeinv e_2, c \leadsto \sigma' &
       p; \sigma' \vdash^{-1} e_1, x \leadsto \sigma''
     \end{array}}
    {\storeinv \lett{x}{e_1}{e_2}, c \leadsto \sigma''}
\end{align*}
\begin{align*}
\textsc{I-LetC: }
  \dfrac
    {\begin{array}{cc}
       \pcon e_1 \downarrow c' &
       \storeinv e_2[c' / x], c \leadsto \sigma'
     \end{array}}
    {\storeinv \lett{x}{e_1}{e_2}, c \leadsto \sigma'}
    \text{$e_1$ is closed}
\end{align*}
\begin{align*}
\textsc{I-LetP: }
  \dfrac
    {\begin{array}{ccc}
       \storeinv e_2, c \leadsto \sigma' &
       \sigma' \vdash^{-1} e_1, (x, y) \leadsto \sigma''
     \end{array}}
    {\storeinv \lett{(x,y)}{e_1}{e_2}, c \leadsto \sigma''}
\end{align*}
\begin{align*}
\textsc{I-App: }
  \dfrac
    {\begin{array}{ll}
        \pcon f!~\tau_1 \dots \tau_m~e_1 \dots e_{n-1}~c \downarrow c' &
        \storeinv e_n, c' \leadsto \sigma'
    \end{array}}
    {\storeinv f~\tau_1 \dots \tau_m~e_1 \dots e_n, c \leadsto \sigma'}
    \text{$e_1, \dots, e_{n-1}$ are closed.}
\end{align*}
\begin{align*}
\textsc{I-InvApp: }
  \dfrac
    {\begin{array}{ll}
        \pcon f~\tau_1 \dots \tau_m~e_1 \dots e_{n-1}~c \downarrow c' &
        \storeinv e_n, c' \leadsto \sigma'
    \end{array}}
    {\storeinv f!~\tau_1 \dots \tau_m~e_1 \dots e_n, c \leadsto \sigma'}
    \text{$e_1, \dots, e_{n-1}$ are closed.}
\end{align*}
\caption{The inverse semantics of \rfunc, missing case rules.}\label{fig:inverse_semantics}
\end{figure}

\begin{figure}[ht!]
\begin{align*}
\textsc{I-CaseL: }
  \dfrac
    {\begin{array}{cc}
       \storeinv e_2, c \leadsto \sigma' &
       p; \sigma' \vdash^{-1} e_1, \inl{x} \leadsto \sigma''
     \end{array}}
    {\storeinv \caseof{e_1}{\inl{x}}{e_2}{\inr{y}}{e_3}, c \leadsto \sigma''}
\end{align*}
\begin{align*}
\textsc{I-CaseR: }
  \dfrac
    {\begin{array}{cc}
       \storeinv e_3, c \leadsto \sigma' &
       p; \sigma' \vdash^{-1} e_1, \inr{y} \leadsto \sigma''
     \end{array}}
    {\storeinv \caseof{e_1}{\inl{x}}{e_2}{\inr{y}}{e_3}, c \leadsto \sigma'}
    c \not\in \PLVal (e_2)
\end{align*}
\begin{align*}
\textsc{I-CaseSL: }
  \dfrac
    {\begin{array}{ccc}
        \pcon e_a[c / out] \downarrow \inl{()} &
        \storeinv e_2, c \leadsto \sigma' &
        p; \sigma' \vdash^{-1} e_1, \inr{x} \leadsto \sigma''
     \end{array}}
    {\storeinv \caseofs{e_1}{\inl{x}}{e_2}{\inr{y}}{e_3}{e_a}, c \leadsto \sigma'}
\end{align*}
\begin{align*}
\textsc{I-CaseSR: }
  \dfrac
    {\begin{array}{ccc}
        \pcon e_a[c / out] \downarrow \inr{()} &
        \storeinv e_3, c \leadsto \sigma' &
        p; \sigma' \vdash^{-1} e_1, \inr{y} \leadsto \sigma''
     \end{array}}
    {\storeinv \caseofs{e_1}{\inl{x}}{e_2}{\inr{y}}{e_3}{e_a}, c \leadsto \sigma'}
\end{align*}
\begin{align*}
\textsc{I-CaseUL: }
  \dfrac
    {\begin{array}{cc}
       % c \in \leaves (e_2) &
       \storeinv e_2, c \leadsto \sigma' &
       p; \sigma' \vdash^{-1} e_1, \inr{x} \leadsto \sigma''
     \end{array}}
    {\storeinv \caseofu{e_1}{\inl{x}}{e_2}{\inr{y}}{e_3}, c \leadsto \sigma'}
\end{align*}
\begin{align*}
\textsc{I-CaseUR: }
  \dfrac
    {\begin{array}{cc}
       % c \in \leaves (e_3) &
       \storeinv e_3, c \leadsto \sigma' &
       p; \sigma' \vdash^{-1} e_1, \inr{y} \leadsto \sigma''
     \end{array}}
    {\storeinv \caseofu{e_1}{\inl{x}}{e_2}{\inr{y}}{e_3}, c \leadsto \sigma'}
\end{align*}
\caption{The inverse semantics for case-expressions.}\label{fig:inverse_semantics_cases}
\end{figure}

\paragraph{Binding Rules}

The binding rules ensure two distinct properties: Predicting a value for a
variable from the forward direction (by \textsc{I-Bind1}) and maintaining the
relevance of variables (by \textsc{I-Bind2}). Specifically, if the same
variable name is attempted to be bound twice, we conclude that they must occur
twice in the program. This is legal by relevance, but we must assure ourselves
that the new value we try to bind is the same as the old one, otherwise the
derivation fails.

\paragraph{Function Application}

A function application does not retract any syntactic construct atomically like
other inference rules which match on the free value --- we instead invoke a
function application which is designed to revert the result of the forward
function application. As expected, the ancillae parameters are invariant (and
closed). During inverse evaluation, a regular function application dictates an
inverse function application and vice versa. We want to express that the result
of the forward application must have been the current free value. Hence, the
inverse application designates it as the dynamic parameter --- and the result
of the application is matched against the dynamic parameter expression.

\paragraph{Let-expressions}

The free value must necessarily match on $e_2$ as that is evaluated last, with
$x$ substituted for $e_1$. Therefore we attempt to bind $c$ to $e_2$ first.
This likely gives us a binding for $x$. How $x$ was computed is witnessed in
$e_1$. So we either completely uncompute $x$ (if $e_1$ was closed, which
completely retracts $e_1$) or we figure out bindings for the variables making
up $x$ by treating $x$ as a free value against $e_1$. The two inference rules
\textsc{I-Let} and \textsc{I-LetP} for let-expressions function identically,
but are written for each syntactic form of let-expressions.

The final inference rules for let-expression, \textsc{I-LetC}, is legal as when
$e_1$ is closed, then is constant, and we are not missing any information
regarding what went into deriving it in the forward direction. This rule is
important for one purpose --- consider the following function:

\begin{align*}
  f~(y: \tau) = \lett{x}{()}{f~x~y}
\end{align*}

$x$ is constant (and static) and may be used an ancilla in the application of
$f$, but the inverse inference rules cannot discern it before it is needed,
\emph{unless} we by the \textsc{I-LetC} rule recognize that we do already know
it.

\paragraph{Case-expressions}

We expect branching to be deterministic. This is achieved in three different
ways, based on the form of the case-expression. We present particular inference
rules for each form:

\begin{enumerate}

  \item Determinism is achieved by the first match principle. In this case, an
    identical side condition, as seen in the big step forward evaluation, is
    added to the \textsc{I-CaseR} rule.

  \item Determinism is achieved by an exit assertion: This is straightforward
    --- we simply check the exit assertion against the free value, as the free
    value should be regarded as the result of whichever branch was taken.

  \item Determinism is achieved by knowing that the free value can only
    uniquely match on one of the branch expressions. In this case we can freely
    expect the branch expression which has a binding form with $c$ to have been
    taken in the forward direction without any further validity checks (but
    again, only if the uniqueness of the branches has been guaranteed).

\end{enumerate}

With formal inference rules for inverse evaluation in place, we can derive
$\sigma$ for Prog.~\ref{prog:inverse_example}. A partial view of the derivation
is presented in Fig.~\ref{fig:inverse_derivation}.

\begin{figure}[t]
  \centering
  \begin{prooftree}
    \infer0[\textsc{I-Bind}]{p; [] \vdash^{-1} x, () \leadsto [x \mapsto ()]}
    \infer0[\textsc{I-Bind}]{p; [x \mapsto ()] \vdash^{-1} y, () \leadsto [x \mapsto (), y \mapsto ()]}
    \infer2[\textsc{I-Prod}]{p; [] \vdash^{-1} (y, x), ((), ()) \leadsto [x \mapsto (), y \mapsto ()]}
    \hypo{\dots}
    \infer2[\textsc{I-LetP}]{p; [] \vdash^{-1} \lett{y}{()}{(y ,x)}, ((), ()) \leadsto [x \mapsto ()]}
  \end{prooftree}
  \caption{Partial view of derivation of program
  Def.~\ref{prog:inverse_example}, by the inverse semantics with inverse input
  $((), ())$.}\label{fig:inverse_derivation}
\end{figure}

\section{Correctness of Inverse Semantics}

We now move on to prove the correctness of the inverse semantics as seen in
Fig.~\ref{fig:inverse_semantics}. The proof reflects that we can recover the
form of the substitution $c$ of $x$ by the inverse semantics from an expression
$e$ --- where  we have that $e[c / x] \downarrow c'$ --- by knowing $c'$. Note
that it is only relevant precisely when $x$ is the only open term in $e$ ---
but this fits well with reality: We will exploit that $x$ is the variable name
of the dynamic variable after every ancillary parameter has been substituted
already, noticing that we may substitute the ancillary variable directly as
they are invariant.

\begin{lemma}\label{thm:exactinv}

  If $\pcon e[c / x] \downarrow c'$ (by $\mathcal{E}$) where $x$ is the only
  free variable in $e$, then $p; \emptyset \vdash^{-1} e, c' \leadsto \{x
  \mapsto c\}$ (by $\mathcal{I}$).

\end{lemma}

\begin{proof}

  By induction over the derivation $\mathcal{E}$. We have the cases:

  \begin{itemize}
    \setlength\itemsep{1em}

    \item Case $\mathcal{E} =
      \dfrac
        {}
        {\pcon ()[c / x] \downarrow ()}$

        This case is not applicable as $()$ obviously cannot contain a free
        occurrence of $x$, so it is vacuously true.

    \item Case $\mathcal{E} =
      \dfrac
        {}
        {\pcon x[c / x] \downarrow c'}$

      Then $c = c'$ and $e = x$, and we can construct a derivation directly by
      \textsc{I-Bind}:

      \begin{align*}
        \mathcal{I} = \dfrac
          {}
          {\emptyinv x, c \leadsto x \mapsto c}
      \end{align*}

    \item Case $\mathcal{E} =
      \dfrac
        {\overset
           {\mathcal{E}_0}
           {\pcon e [c / x] \downarrow c'}
        }
        {\pcon \inl{e}[c / x] \downarrow c'}$

      We immediately use the Induction Hypothesis on $\mathcal{E}_0$ and get a
      derivation $\mathcal{I}_0 = \emptyinv e, c' \leadsto x \mapsto c$, and by
      the \textsc{I-Inl} rule on $\mathcal{I}_0$ we have:

      \begin{align*}
        \mathcal{I} = \dfrac
          {\overset
            {\mathcal{I}_0}
            {\emptyinv e, c' \leadsto x \mapsto c}
          }
          {\emptyinv \inl{e}, \inl{c'} \leadsto x \mapsto c}
      \end{align*}

      An almost identical proof is constructed for when $\mathcal{E}$ is a
      derivation of $\pcon \inr{e}[c / x] \downarrow c'$ and $\pcon
      (\roll{\tau}{e}) [c / x] \downarrow c'$.

    \item Case $\mathcal{E} =
      \dfrac
        {\begin{array}{cc}
            \overset
              {\mathcal{E}_0}
              {e_1 [c / x] \downarrow c'} &
            \overset
              {\mathcal{E}_1}
              {e_2 [c / x] \downarrow c''}
         \end{array}
        }
        {\pcon (e_1, e_2) [c / x] \downarrow (c', c'')}$

        $x$ must occur in either $e_1$ or $e_2$ or both. If $x$ occurs in $e_1$
        only, the Induction Hypothesis on $\mathcal{E}_0$ gives us a derivation
        $I_0 = \emptyinv e_1, c' \leadsto x \mapsto c$. Further, since $e_2$ by
        implication contains no free variables, by Lemma~\ref{thm:emptyinv} we
        have a derivation $\mathcal{I}_1 = p; x \mapsto c \vdash^{-1} e_2, c''
        \leadsto x \mapsto c$. By applying \textsc{I-Prod} with $\mathcal{I}_0$
        and $\mathcal{I}_1$ we have:

        \begin{align*}
          \mathcal{I} = \dfrac
            {\begin{array}{cc}
                \overset
                  {\mathcal{I}_0}
                  {\emptyinv e_1, c' \leadsto x \mapsto c} &
                \overset
                  {\mathcal{I}_1}
                  {p; x \mapsto c \vdash^{-1} e_2, c'' \leadsto x \mapsto c}
             \end{array}
            }
            {\emptyinv (e_1, e_2), (c', c'') \leadsto x \mapsto c}
        \end{align*}

        An analogous proof is valid for if only $e_2$ contains $x$. If both
        $e_1$ and $e_2$ contain $x$, the derivation of $\mathcal{I}_1$ must
        include an application of \textsc{I-Bind2} to show that the resulting
        store is in fact $x \mapsto c$.

    % \item Case $\mathcal{E} =
    %   \dfrac
    %     {\overset
    %       {\mathcal{E}_0}
    %       {\pcon e [c / x] \downarrow c'}
    %     }
    %     {\pcon (\roll{\tau}{e}) [c / x] \downarrow \roll{\tau}{c'}}$.

    \item Case $\mathcal{E} =
      \dfrac
        {\overset
          {\mathcal{E}_0}
          {\pcon e [c / x] \downarrow \roll{\tau}{c''}}
        }
        {\pcon (\unroll{\tau}{e}) [c / x] \downarrow c''}$

        By the Induction Hypothesis on $\mathcal{E}_0$ (with $c' =
        \roll{\tau}{c''}$) we get a derivation $\mathcal{I}_0 = \emptyinv e [c
        / x], \roll{\tau}{c''} \leadsto x \mapsto c$, and we can use it
        directly to construct $\mathcal{I}$ by applying \textsc{I-Unroll}:

        \begin{align*}
          \mathcal{I} = \dfrac
            {\overset
              {\mathcal{I}_0}
              {\emptyinv e, \roll{\tau}{c''} \leadsto x \mapsto c}
            }
            {\emptyinv \unroll{\tau}{e}, c'' \leadsto x \mapsto c}
        \end{align*}

    \item Case $\mathcal{E} =
      \dfrac
        {\begin{array}{cc}
            \overset
              {\mathcal{E}_0}
              {\pcon e_1 [c / x] \downarrow c''} &
            \overset
              {\mathcal{E}_1}
              {\con e_2 [x' / c'', c / x] \downarrow c'}
         \end{array}
        }
        {\pcon (\lett{x'}{e_1}{e_2}) [c / x] \downarrow c'}$

        By Lemma~\ref{thm:generalinv}, we may consider multiple substitutions.
        There are a number of cases here: $x'$ might or might not be an open
        term in $e_2$ and $x$ might be in $e_1$, $e_2$ or both.

     \begin{itemize}

       \item If $x'$ is not an open term in $e_2$, then $x$ must occur in
         $e_2$.  This is evident as a derivation $\mathcal{I}_0$ by the
         Induction Hypothesis on $\mathcal{E}_1$ would result in a store not
         containing a mapping for $x'$ --- and a derivation $\mathcal{I}_1$ of
         $\langle e_1, x' \rangle$ would immediately apply \textsc{I-Var2} as
         there is no mapping for $x'$ by the derivation $\mathcal{I}_0$. Thus
         the Induction Hypothesis on $\mathcal{E}_1$ gives us $\mathcal{I}_0 =
         \emptyinv e_2, c \leadsto x \mapsto c$ and we can construct
         $\mathcal{I}$ as:

         \begin{align*}
           \mathcal{I} = \cfrac
             {\begin{array}{cc}
                 \overset
                   {\mathcal{I}_0}
                   {\emptyinv e_2, c' \leadsto x \mapsto c} &
                   {\cfrac
                     {}
                     {p; x \mapsto c \vdash^{-1} e_1, x' \leadsto x \mapsto c}
                     x' \not\in \{ x \mapsto c \}
                   }
              \end{array}
            }
            {\emptyinv \lett{x'}{e_1}{e_2}, c' \leadsto x \mapsto y}
         \end{align*}

       \item If $x'$ is an open term in $e_2$, we have three possible cases for
         where $x$ occurs: in only $e_1$, in only $e_2$ or in both. If it only
         occurs in $e_2$, then by Lemma~\ref{thm:generalinv} we have a
         derivation $\mathcal{I}_1 = \emptyinv e_2, c' \leadsto x' \mapsto c'',
         x \mapsto c$. This means $e_1$ is necessarily a closed term, and by
         Lemma~\ref{thm:emptyinv} on $\mathcal{E}_0$ we finally have:

         \begin{align*}
           \mathcal{I} = \dfrac
           {\begin{array}{cc}
               \overset
                 {\mathcal{I}_0}
                 {\emptyinv e_2, c' \leadsto x \mapsto c, y \mapsto c''} &
               \overset
                 {\mathcal{I}_1}
                 {p; x \mapsto c, y \mapsto c'' \vdash^{-1} e_1, y \leadsto x \mapsto c}
            \end{array}
           }
           {\emptyinv \lett{x'}{e_1}{e_2}, c' \leadsto x \mapsto y}
         \end{align*}

         If $x$ is an open term in both, a similar reasoning is used, but $e_1$
         is not closed --- rather, we know the derivation of $\langle e_1, x'
         \rangle$ must use an application of \textsc{I-Bind2} which ensures
         that the only binding is $x \mapsto c$ is consistent, and we are done.

         If $x$ only occurs in $e_1$, then by applying the Induction Hypothesis
         directly on $\mathcal{E}_1$ we get a derivation $\mathcal{I}_0 =
         \emptyinv e_2, c' \leadsto x' \mapsto c''$. The only inference rule
         possible for $\mathcal{I}$ is \textsc{I-Let}, and we need to construct
         a derivation for $\mathcal{I}_1$. By applying \textsc{I-Var1} for the
         second premise on $\langle e_1, x' \rangle$ we get some derivation:

         \begin{align*}
           \mathcal{I}_1 = \dfrac
            {\overset
              {\mathcal{I}_1'}
              {\emptyinv e_1, c'' \leadsto \sigma}
            }
            {p; x' \mapsto c'' \vdash^{-1} e_1, x' \leadsto \sigma}
         \end{align*}

         For some $\sigma$. Then we may use the Induction Hypothesis on
         $\mathcal{E}_0$ with $\mathcal{I}_1'$, and we finalize the derivation
         $\mathcal{I}_1$ so that $\sigma = x \mapsto c$ and we can finally
         construct:

         \begin{align*}
           \mathcal{I} = \dfrac
            {\begin{array}{cc}
                \overset
                  {\mathcal{I}_0}
                  {\emptyinv e_2, c' \leadsto x' \mapsto c''} &
                \overset
                  {\mathcal{I}_1}
                  {\cfrac
                    {\emptyinv e_1, c'' \leadsto x \mapsto c}
                    {p; x' \mapsto c'' \vdash^{-1} e_1, x' \leadsto x \mapsto c}
                  }
             \end{array}}
            {\emptyinv \lett{x'}{e_1}{e_2}, c' \leadsto x \mapsto c}
         \end{align*}

         An analogous proof works for when $e = \lett{(x, y)}{e_1}{e_2}$.

     \end{itemize}

    \item Case $\mathcal{E} =
      \dfrac
        {\begin{array}{cc}
            \overset
              {\mathcal{E}'_0, \dots, \mathcal{E}'_n}
              {\pcon e_1 [c / x] \downarrow c_1' \dots \pcon e_n [c / x] \downarrow c_n'} &
            \overset
              {\mathcal{E}''}
              {\pcon e[c_1' / x_1, c_n' / x_n, c / x] \downarrow c'}
         \end{array}
        }
        {\pcon (f~\alpha_1 \dots \alpha_m~e_1 \dots e_n \downarrow) [c / x] \downarrow c'}$

        Where $p(f) = f~\alpha_1 \dots \alpha_m~x_1 \dots x_n = e$. The only
        applicative rule for the derivation is \textsc{I-App}, so each
        expression $e_1, \dots, e_{n-1}$ must be closed and $x$ must occur in
        $e_n$. $c'$ is also closed, as it is a canonical value. We by
        Theorem~\ref{thm:inversemain} (the correctness of inverse application,)
        have that $\mathcal{E}''' = f!~\alpha_1 \dots \alpha_m~e_1 \dots
        e_{n-1}~c' \downarrow c''$, where $\emptyinv e[c_1' / x_1, \dots,
        c_{n-1}' / x_{n-1}], c' \leadsto \sigma$ with $\sigma(x_n) = c''$.

        But that means that $e_n [c / x] \downarrow c''$, as a substitution of
        $c_n'$ occurs for $x_n $ in $\mathcal{E}''$, and we have $e_n [c / x]
        \downarrow c_n'$ by $\mathcal{E}_n'$. By the Induction Hypothesis on
        $\mathcal{E}_n'$ we have $\mathcal{I}_0 = \emptyinv e_n, c'' \leadsto x
        \mapsto c$, and we construct the full derivation:

        \begin{align*}
          \mathcal{I} = \dfrac
            {\begin{array}{cc}
                \overset
                  {\mathcal{E}''''}
                  {f!~\alpha_1 \dots \alpha_m~e_1 \dots e_{n-1}~c' \downarrow c''} &
                \overset
                  {\mathcal{I}_0}
                  {\emptyinv e_n, c'' \leadsto x \mapsto c}
             \end{array}
            }
            {\emptyinv f~\alpha_1 \dots \alpha_m~e_1 \dots e_n, c' \downarrow c''}
        \end{align*}

        An analogous proof occurs when $\mathcal{E}$ is a derivation of an
        inverse function call.

    \item Case $\mathcal{E} =
      \dfrac
        {\begin{array}{cc}
            \overset
              {\mathcal{E}_0}
              {\pcon e_1 [c / x] \downarrow \inl{c''}} &
            \overset
              {\mathcal{E}_1}
              {\pcon e_2 [c'' / x', c / x]}
         \end{array}
        }
        {\pcon (\caseof{e_1}{x'}{e_2}{x''}{e_3}) [c / x] \downarrow c'}$.

        Proving this case is very similar to the proof for let-expressions,
        with completely similar sub-cases for which expression contains which
        open term. The only difference is we cannot directly apply
        \textsc{I-Var} on $e_1$ but must start with matching the sum-term by
        \textsc{I-Inl}. The same is true for \textsc{E-CaseR},
        \textsc{E-CaseSL} and \textsc{E-CaseSR}.

  \end{itemize}

And we are done.

\end{proof}

\begin{lemma}\label{thm:emptyinv}

  If $\pcon e \downarrow c$ (by $\mathcal{E}$) where there are no free
  variable in $e$, then $\storeinv e, c \leadsto \sigma$ (by $\mathcal{I}$).

\end{lemma}

\begin{proof}

  By induction over the derivation $\mathcal{E}$. The proof is highly similar
  to the one for Lemma~\ref{thm:exactinv}, so we only show exemplary cases,
  especially ones which are relevant:

  \begin{itemize}

    \item Case $\mathcal{E} = \dfrac
      {}
      {\pcon () \downarrow ()}$

      We can construct a derivation directly using the \textsc{I-Unit} rule as:

      \begin{align*}
        \mathcal{I} = \dfrac
          {}
          {\emptyinv (), () \leadsto \emptyset}
      \end{align*}

    \item Cases for when $e$ is $\inl{e'}, \inr{e'}, \roll{\tau}{e'}, (e',
      e'')$ and $\unroll{\tau}{e'}$ follow almost immediately by the Induction
      Hypothesis on the premises of $\mathcal{E}$. The precise method was
      exercised in the proof of Lemma~\ref{thm:exactinv}.

    \item Case $\mathcal{E} = \dfrac
      {\begin{array}{cc}
        \overset
          {\mathcal{E}_0}
          {\pcon e_1 \downarrow c'} &
        \overset
          {\mathcal{E}_1}
          {\pcon e_2 [c' / x] \downarrow c}
       \end{array}
      }
      {\pcon \lett{x}{e_1}{e_2} \downarrow c}$

      There are two cases: $e_2$ contains the open term $x$ or it does not. If
      it does not, then by the Induction Hypothesis on $\mathcal{E}_1$ we get a
      derivation $\mathcal{I}_0 = \emptyinv e_2, c \leadsto \emptyset$.
      $\mathcal{I}$ must necessarily use the \textsc{I-Let} rule, and a
      derivation of $\langle e_1, x \rangle$ must immediately use the
      \textsc{I-Var2} rule as $\sigma = \emptyset$, and we have:

      \begin{align*}
        \mathcal{I} = \dfrac
          {\begin{array}{cc}
              \overset
                {\mathcal{I}_0}
                {\emptyinv e_2, c \leadsto \emptyset} &
              \cfrac
                {}
                {\emptyinv e_1, x \leadsto \emptyset}
              x \not\in \emptyset
           \end{array}}
          {\emptyinv \lett{x}{e_1}{e_2} \leadsto \emptyset}
      \end{align*}

      If $x$ is an open term in $e_2$, then by the induction hypothesis on
      $\mathcal{E}_0$ we get a derivation $\mathcal{I}_0 = \emptyinv e_2, c
      \leadsto x \mapsto c'$ for some $c'$. Then we construct $\mathcal{I}_1$
      by applying the \textsc{I-Var1} if we can prove $\mathcal{I}_1' =
      \emptyinv e_1, c' \leadsto \sigma$ for some $\sigma$. But we have this by
      the Induction Hypothesis on $\mathcal{E}_0$ with $\mathcal{I}_1'$
      concluding $\sigma = \emptyset$, allowing us to construct:

      \begin{align*}
        \mathcal{I} = \dfrac
          {\begin{array}{cc}
              \overset
                {\mathcal{I}_0}
                {\emptyinv e_2, c \leadsto x \mapsto c'} &
              \cfrac
                {\overset
                  {\mathcal{I}_1}
                  {\emptyinv e_1, c' \leadsto \emptyset}
                }
                {p; x \mapsto c \vdash^{-1} e_1, x \leadsto \emptyset}
           \end{array}}
          {\emptyinv \lett{x}{e_1}{e_2} \leadsto \emptyset}
      \end{align*}

  \end{itemize}

  The remaining cases have been omitted, but are straightforward to include.

\end{proof}

\begin{lemma}\label{thm:generalinv}

  If $\pcon e[c_1 / x_1, \dots, c_n / x_n] \downarrow c'$ (by $\mathcal{E}$)
  where exactly the set $x_1, \dots, x_n$ are the free variables occurring in
  $e$, then $p; \emptyset \vdash^{-1} e, c' \leadsto x_1 \mapsto c_1, \dots,
  x_n \mapsto c_n$ (by $\mathcal{I}$).

\end{lemma}

\begin{proof}

  The proof of this is a straightforward generalization of
  Lemma~\ref{thm:exactinv} and~\ref{thm:emptyinv}.

\end{proof}

\begin{theorem}[Correctness of Inverse Semantics]\label{thm:inversemain}

  If $p \vdash f~c_1 \dots c_n \downarrow c$ with $p(f) = \alpha_1 \dots
  \alpha_m~x_1 \dots x_n = e$, then $\sigma(x_n) = c_n$ where $\emptyinv e[c_1
  / x_1, \dots, c_{n-1} / x_{n-1}], c \leadsto \sigma$.

\end{theorem}

\begin{proof}

  Follows immediately by an application of Lemma~\ref{thm:exactinv} by
  considering $e[c_1 / x_1, \dots, c_{n-1} / x_{n-1}]$ as the expression $e$
  with exactly one open term $x_n$, which we substitute with $c_n$. Then
  $\emptyinv e[c_1 / x_1, \dots, c_{n-1} / x_{n-1}], c \leadsto x_n \mapsto
  c_n$, and specifically, $\{x_n \mapsto c_n\} (x_n) = c_n$.

\end{proof}


%-------------------------------------------------------------------------------
% PROGRAMMING
%-------------------------------------------------------------------------------

% !TEX root = main.tex

\chapter{Programming in \rfunc}\label{sec:prog}

Although \rfunc is a r-Turing complete language, it lacks many of the
convenient features of most modern functional languages. Luckily, we can encode
many language constructs directly as syntactic transformations from a
notationally light language to the formal core language.

The principle encompasses that for each category of syntactic abstraction, we
can show that there exists a systematic translation from the notationally light
language to \rfunc. This allows us to introduce a number of practical
improvements without the necessity to corroborate their correctness by either
the type system or operational semantics, instead opting to present a
\emph{translation scheme}.

A translation scheme $e \defeq e'$ replaces the program text $e$ with $e'$,
with $e$ being a valid expression in the light language and $e'$ being a valid
\rfunc expression. A successful transformation therefore does not entail that a
well-typed program is generated, only it is grammatically correct. Often, a
translation will depend on recursive descent over the expression. In these
cases, we express a translation as $\langle e \rangle$ (so an inner translation
is notated as $\langle e' \rangle$ with $e'$ being some subexpression of $e$.
Sometimes, we require some auxiliary information to be present for a
translation. In these cases, we express a translation as $\langle e, s
\rangle$, where the translation for $e$ depends the value $s$. We will make
clear for each translation what is going on.

\section{Variants}\label{sec:variants}

Variants, also known as tagged unions, are related to algebraic data types.
They provide a method of declaring new data types as a fixed number of named
alternatives. In \rfunc they generalize recursive types, sum types, and
case-expressions over these. A variant declaration is of the form:

\begin{align*}
  \beta = \texttt{v}_1 \mid \dots \mid \texttt{v}_n
\end{align*}

This defines a new data type (a variant type) identified by $\beta$.
Constructing a value of a variant type entails choosing exactly one of the
possible \emph{variant constructors} $\texttt{v}_1, \dots, \texttt{v}_n$ and
potentially grant it data to carry. Then, given a variable of a variant type,
we pattern match over its possible constructor forms to unveil the data and to
control program flow.

We have seen that we can generalize binary sums $\tau_1 + \tau_2$ to $n$-ary
sums by repeated sum types in the right component $\tau_2$, and that we can
chain together case-expressions to match the correct arm of such a sum. We
choose an encoding of variants which exploits this pattern. This works because
the variant constructors are ordered in the declaration and will
deterministically match with the respective position in the $n$-ary sum.

For a variant which carries no data, the translation corresponds to stripping
away the variant constructor tags, leaving us with the underlying $n$-sum type
of all unit types:

\begin{align*}
  \beta = \texttt{v}_1 \mid \dots \mid \texttt{v}_n \defeq \beta = 1 + \dots + 1
\end{align*}

We can further extend variants to carry data by adding parameters. We allow
generic type parameters by adding a type parameter to the variant declaration.
The syntax for variant declarations becomes:

\begin{align*}
  \beta~\alpha^* = \texttt{v}_1~[\tau\alpha]^* \mid \dots \mid \texttt{v}_n~[\tau\alpha]^*
\end{align*}

Where $[\tau\alpha]^*$ signifies zero or more constructor parameters of either
a type (allowing for inner variants) \emph{or} type variables supplied to
$\beta$.

If exactly one parameter $\tau$ is present for a constructor \texttt{v}$_i$, the
type at position $i$ in the $n$-ary sum is changed from the unit type to the
type $\tau$.

\begin{align*}
  \beta~\alpha = \texttt{v}_1 \mid \texttt{v}_2~\tau \mid \texttt{v}_3~\alpha
  \defeq \beta~\alpha = 1 + \tau + \alpha
\end{align*}

Notice that we may generalize any parameter-free variant constructor to one
with a single parameter of type unit, which we omit from the syntax.

If a variant constructor \texttt{v}$_i$ has $m \ge 2$ parameters $p_1, \dots,
p_m$ , the type in the position of $i$ in the $n$-ary sum is changed into a
product type $p_1 \times \dots \times p_m$.

\begin{align*}
  \beta = \texttt{v}_1 \mid \dots \mid \texttt{v}_i~\tau_1 \dots \tau_m \mid \dots
  \defeq \beta =  1 + \dots + \tau_1 \times \dots \times \tau_m + \dots
\end{align*}

There is one more case we need to take into consideration for transformation:
Recursively defined variants. In the following we use the subscript $i$ to
indicate that the variant we are declaring has been indexed by $i$ in the
family of possible variant names. The principle goes that if any of the variant
constructors for a variant type $\beta_i$ have a self-referencing parameter (a
parameter of type $\beta_i$), the translated type of $\beta_i$ is recursive and
a fresh variable is designated as the recursion variable.

\begin{align*}
  &\beta_i = \texttt{v}_1 \mid \texttt{v}_2~\beta_i
  \defeq \beta_i = \mu X.1 + X & \text{$X$ is a fresh type variable.}
\end{align*}

The above actually corresponds to an encoding of the natural numbers.

When all variant declarations have been translated, we have generated a set
VarDecs of named translations of the form $\beta = \tau$ where $\tau$ is a core
type. We can define a translation of occurrences of variant types inductively
over the core syntax of \rfunc, expanding variant types as we go. The
interesting cases are:

\begin{align*}
  f \dots (x:\beta) \dots &\defeq f \dots (x:\tau) \dots
  & \text{When $x$ is any parameter for $f$ and VarDecs$(\beta) = \tau$} \\
  \roll{\beta}{e} &\defeq \roll{\tau}{e} &\text{When VarDecs$(\beta) = \tau$} \\
  \unroll{\beta}{e} &\defeq \unroll{\tau}{e} &\text{When VarDecs$(\beta) = \tau$}
\end{align*}

The simplest variant declaration corresponds to the type \texttt{Bool} of
Boolean values:

\begin{align*}
  \texttt{Bool} = \texttt{True} \mid \texttt{False} \defeq
  \texttt{Bool} = 1 + 1
\end{align*}

The \texttt{Maybe} datatype is encoded as:

\begin{align*}
  \texttt{Maybe}~\alpha = \texttt{Nothing} \mid \texttt{Just}~\alpha \defeq
  \texttt{Maybe} = 1 + \alpha
\end{align*}

While the encoding for generic lists exemplifies most of the translation rules
for variant declarations simultaneously:

\begin{align*}
  \texttt{List}~\alpha = \texttt{Nil} \mid \texttt{Cons}~\alpha~(\texttt{List}~\alpha)
  \defeq \texttt{List} = \mu X . 1 + \alpha \times X
\end{align*}

As mentioned earlier a variant value is an encoding of a nested number of sum
terms which correspond to the ordering of the variant. We can define a general
translation for variant values --- in the following we denote $|\beta|$ as the
number of variant constructors a variant type has. We have:

\begin{align*}
  \texttt{v}_i~e_1 \dots e_n &\defeq
    \textbf{inr}_1 (\dots (
    \textbf{inr}_{i-1} (
    \textbf{inl}_i ((e_1, \dots, e_n))) \dots )
    &\text{When $i < |\beta|$} \\
  \texttt{v}_i~e_1 \dots e_n &\defeq
    \textbf{inr}_1 (\dots (
    \textbf{inr}_{i} (e_1, \dots, e_n)) \dots )
    &\text{When $i = |\beta|$} \\
\end{align*}

A handy result to keep in mind is that if two variant definitions have the same
number of alternatives and carry data of identical types, they are isomorphic
and may be encoded the same, which simplifies the translation scheme.

The final critical transformation is obtaining a classical case-expression from
a case over a variant value. In the light language we wish to write a case over
a variant type as a single possible choice between all the variant constructors
of the type of the variant value.

\begin{rfuncode}
  case $v$ of
    $\texttt{v}_1~e_{1_1} \dots e_{1_j}$ => $e_1$
           $\vdots$
    $\texttt{v}_m~e_{m_1} \dots e_{m_k}$ => $e_m$
\end{rfuncode}

Where $v$ has type $\beta$ and $\texttt{v}_1, \dots, \texttt{v}_m$ are the
variant constructors of $\beta$. We observe that we can exploit the encoding of
variant values to unpack the data for each constructor. The construction is a
bit more complex as we need to case over a chain of fresh variables. We
therefore introduce the notation $\langle e, w \rangle$ to signify that $e$ is
to be translated and $w$ is a fresh variable which should be cased over. Then the
translation is defined recursively as a nested structure of binary
case-expressions:

\begin{align*}
  &\langle \texttt{v}_i~e_{i_1}, \dots, e_{i_j} \Rightarrow e_i, w \rangle \defeq
  \caseof{w}{\inl{(e_{i_1}, \dots, e_{i_j})}}{e_i}{\inr{w'}}{e'} \\
  &\qquad\text{When $i < m$ and $w'$ is a fresh variable and $e' = \langle
          w', v_{i+1}~e_{i_1}, \dots, e_{i_k} \Rightarrow e_{i + 1} \rangle$} \\
  &\langle \texttt{v}_i~e_{i_1}, \dots, e_{i_j} \Rightarrow e_i, w \rangle \defeq
  \lett{(e_{i_1}, \dots, e_{i_j})}{w}{e_i} \\
  &\qquad\text{When $i = m$}
\end{align*}

A fully-fledged exampled of how variants are translated is presented in
Fig.~\ref{fig:variant_translation}.

\begin{figure}[ht!]
  \centering
  \begin{subfigure}[b]{0.90\textwidth}
    \begin{rfuncodenum}
Choice = Rock | Paper | Scissors

rpsAI ($c$:Choice) =
  case $c$ of
    Rock => Paper
    Paper => Scissors
    Scissors => Rock
    \end{rfuncodenum}
    \caption{Light program.}
  \end{subfigure}
  ~
  \begin{subfigure}[b]{0.90\textwidth}
    \begin{rfuncodenum}
rpsAI ($c$:1 + (1 + 1)) =
  case $c$ of
    inl(()) => inr(inl(()))
    inr($w$) => case $w$ of
      inl(()) => inr(inr(()))
      inr(()) => inl(())
    \end{rfuncodenum}
    \caption{Core program.}
  \end{subfigure}
\caption{A full example of variant translations}\label{fig:variant_translation}
\end{figure}

\section{Top-level Function Cases}\label{subsec:top_level}

A top-level function case is a generalization of inspecting the immediate form
of the inputs to a function. This entails, contrary to the core language, that
arguments are not necessarily named in the definition but are pattern matched
on directly as expressions. A function definition is then given as a finite,
strictly ordered set of $m$ \emph{function clauses}:

\begin{align*}
\{ f~e_{1_i} \dots e_{n_i} = e \mid 1 \le i \le m \}
\end{align*}

Where we denote the parameter expressions $e_{1_i} \dots e_{n_i}$ for each
clause a \emph{clause pattern}. There are some restrictions on a function
pattern:

\begin{enumerate}

  \item We require that no clause pattern matches another clause pattern
    perfectly (we require the clause patterns to be orthogonal).

  \item We require that the form of each expression in a clause pattern is
    either a sum term, a variable name or a variant constructor. This is
    necessary as case-expressions in the core language are only defined for sum
    terms (and we saw in Sec.~\ref{sec:variants} that variants transform into
    sum terms).

  \item We require that there are no omitted clauses. This is to enforce
    totality of the branches when the program has been transformed to the core
    rendition, which does not support missing branches for sum terms.

\end{enumerate}

We enforce a particular ordering on the clauses. This is due to how the
top-level case will be unfolded in the next transformation step. This ordering
can be inferred, so strictly speaking we do not have to restrict the programmer
from writing them in any arbitrary order (enforcing the proper ordering might
be a good design choice nonetheless, as implicit reordering of clauses might
cause confusion). We define the ordering on a parameter expression as:

\begin{align*}
  \inl{e} &\le \inr{e} & \\
  \texttt{v}_i &\le \texttt{v}_j
    & \text{When $\texttt{v}_i$ and $\texttt{v}_j$ are variant constructors of
            the same type and $i \le j$.} \end{align*}

And we define the ordering of clause patterns as the ordering parameter-wise,
associated to the left.

\begin{align*}
    & f~e_{1_1} \dots e_{n_1} = \cdot
  \le f~e_{1_2} \dots f~e_{n_2} = \cdot &
  \text{if $e_{1_1} \le e_{1_2}, \dots, e_{n_1} \le e_{n_2}$}
\end{align*}

With an ordering in place, we may define function application of $f$ as
evaluating the clause body of the first clause in the ordered set of clauses of
$f$ for which each parameter expression matches the values supplied to the
application. In the following, $f_s$ denotes a chain of increasing order of
function clauses. We define an \emph{ordered traversal} as:

\begin{align*}
  (f~e_1 \dots e_n = e \le f_s)~c_1 \dots c_n
  \begin{cases}
    e & \text{if $c_1 \rhd e_1, \dots, c_n \rhd c_n$}  \\
    (f_s)~c_1 \dots c_n & \text{otherwise}
  \end{cases}
\end{align*}

That is, if the values supplied to the function application are applicable for
a clause, evaluate the clause body. Otherwise, try the next clause. Since we
assumed that clause coverage is total, this will always evaluate \emph{some}
clause body.

The translation should unfold a series of case-expressions which respect the
order of clauses as described. An observation is that we can make a distinction
between the parameters which vary and those that do not. Parameters that do not
vary are necessarily variable names in every clauses and can be ignored.

Another observation is that we can group the clauses by how each parameter
expression is repeated to match on the subsequent parameter expression. This
gives us a nested structure of sets akin a tree, where the clause bodies are
the leafs. We have:

\begin{align*}
  \{ e_{1_i}~\{ e_{2_j}~\{ \dots \{ e_{i + j + \dots} \} \dots \}_j \}_i \mid 1 \le j \le q \} \mid 1 \le i \le p \}
\end{align*}

Where we have truncated the $m$ occurrences of pattern expressions in the first
parameter into $k$ distinct expressions. The same characteristic holds for each
parameter, and the process is then repeated recursively. The inner-most set is
a singleton set containing the clause body for exactly the combination of
pattern expression (which is reflected in its additive subscript). This
structure tells us how to ``flatten'' the clauses into individual
case-expressions.  Each instance of a group is recursively asked to flatten. In
the following, $x$ is a fresh variable name and $a$ and $b$ are placeholders
for the ordering of parameters. We have:

\begin{align}
  &\langle \{ e_{a_i}~\{ e_{b_j}~\{ \dots \} \} \mid 1 \le j \le q \}_i \mid 1 \le i \le p \} \rangle \defeq   \label{def:clause_translation} \\
  &\smpcase{x}{e_{a_1} \Rightarrow \langle \{ e_{b_j}~\{ \dots \} \mid 1 \le j \le q \}_1 \rangle \dots e_{a_p} \Rightarrow \langle \{ e_j~\{ \dots \} \mid 1 \le j \le q \}_p} \rangle \nonumber
\end{align}

And the bottom case:

\begin{align}
  &\langle \{ e_{a_i}~\{ e_i \} \mid 1 \le i \le p \} \rangle \defeq
  \smpcase{x}{e_{a_1} \Rightarrow e_1, \dots, e_{a_k} \Rightarrow e_p}
\label{def:clause_translation_bottom}
\end{align}

A clear issue with translating the current translation scheme is the apparent
loss of type information for parameters. In the core language we require that
parameters are given in the form $(x:\tau)$. Transforming name-less expressions
into fresh variable names is trivial, but omitting type information requires
that we provide full type inference. Therefore we also require a top-clause
whose only purpose is to pair up with the parameters in the core language. A
complete function definition becomes

\begin{align*}
  &f :: \alpha_1 \dots \alpha_m~.~\tau_1 \rightarrow \dots \rightarrow \tau_n \\
  &\{ f~e_{1_i} \dots e_{n_i} = e \mid 1 \le i \le m \}
\end{align*}

The clauses are translated with regards to~(\ref{def:clause_translation})
and~(\ref{def:clause_translation_bottom}), giving $e_{body}$, while the
function as a whole is translated as:

\begin{align*}
  \begin{array}{l}
    f :: \alpha_1 \dots \alpha_m~.~\tau_1 \rightarrow \dots \rightarrow \tau_n \\
    \{ f~e_{1_i} \dots e_{n_i} = e \mid 1 \le i \le m \}
  \end{array} \defeq
  f~\alpha_1 \dots \alpha_m~(x_1: \tau_1) \dots (x_n: \tau_n) = e_{body}
\end{align*}

We show the translation of a top-level cases with a translation of a map
function. The translation is shown in Fig.~\ref{fig:top_level_translation}.

\begin{figure}[ht!]
  \centering
  \begin{subfigure}[b]{0.90\textwidth}
    \begin{rfuncodenum}
map :: $\alpha~\beta~(\alpha \leftrightarrow \beta) \rightarrow \mu X . 1 + \alpha \times X$
map $f$ inl(()) = roll [$\mu X . 1 + \beta \times X$] inl(())
map $f$ inr($(x, xs')$) = let $x' = f~\alpha~x$
                           $xs''$ = map $f~\alpha~\beta~xs'$
                       in roll [$\mu X . 1 + \beta \times X$] inr(($x', xs''$))
    \end{rfuncodenum}
  \caption{Light program.}
  \end{subfigure}
  ~
  \begin{subfigure}[b]{0.90\textwidth}
    \begin{rfuncodenum}
map $\alpha~\beta~(f: \alpha \leftrightarrow \beta))~(xs: \mu X . 1 + \alpha \times X)$ = case unroll [$\mu X . 1 + \alpha \times X$] $xs$ of
  inl(()) => roll [$\mu X . 1 + \beta \times X$] inl(())
  inr($(x, xs')$) => let $x' = f~\alpha~x$
                       $xs''$ = map $f~\alpha~\beta~xs'$
                   in roll [$\mu X . 1 + \beta \times X$] inr(($x', xs''$))
    \end{rfuncodenum}
  \caption{Core program.}
  \end{subfigure}
\caption{An example of a top-level case translation.}\label{fig:top_level_translation}
\end{figure}

\section{Guards}

When we have added support for a generalized top-level function case as seen in
Sect.~\ref{subsec:top_level}, it is natural to consider \emph{guards} as well.
Guards are additional requirements on the form of the clause pattern $e_1 \dots
e_n$ of any particular clause of a function $f$. Each guard is associated with
one and only one clause in the form of an additional expression $g$ where $g$
always evaluates to $1 + 1$ (a Boolean value). A clause becomes:

\begin{align*}
f~e_1 \dots e_n \mid g = e
\end{align*}

During a function application $f$ with canonical values $c_1 \cdots c_n$
substituted for the parameters of $f$, $g$ has the values $c_1 \dots c_n$ in
scope.

Each top-level pattern may now be repeated multiple times in the function
declaration, so we lax the requirement of orthogonality of clauses, but with a
new guard for each repetition. Each match of a clause pattern then in actuality
matches a set of clauses, which we call a \emph{cluster}. Again, we impose an
order on a cluster of clauses. We order them by position of definition, since
there is no intrinsic order of the guards, with on exception: The default guard
is always the top element. We then perform an ordered traversal to discern
which clause-body should be taken. Denote a pattern by $p$. We have:

\begin{align*}
  \begin{array}{c}
    f~p \mid g_1 = e_1 \\
      \vdots \\
    f~p \mid g_n = e_n
  \end{array}
  \Leftrightarrow g_1 = e_1 \le \dots \le g_n = e_n \\
\end{align*}

Notice that the same pattern $p$ occurs in every clause above. We define the
ordered traversal $- \le -$ for guards similarly as to top level function-cases
by

\begin{align*}
  g = e \le g_s
  \begin{cases}
    e & \text{if } g \downarrow \inl{()} \\
    g_s & \text{if } g \downarrow \inr{()}
  \end{cases}
\end{align*}

To make sure that the ordered traversal is total, we require that a default
guard in the form of a keyword \textbf{otherwise} is supplied for each pattern
which has guards. This is simply translated directly to an $\inl{()}$
expression. As for the restrictions on top-level cases, this ensures that
\emph{some} clause body always will be evaluated. We require totality to
guarantee that we always can generate every arm in a case-expression in the
core language. Note that an omitted guard for a clause $f~p = e$ can be
generalized to a clause containing a guard which is always passed (so $f~p = e
\Leftrightarrow f~p \mid \inl{()} = e$).

We show a translation scheme from a top-level case function which has guards
into a top-level case function which does not have guards. By this we impose an
order of translations so that a translation of guards must occur before a
translation of top-level case functions.

Assume we have partitioned the function clauses into clusters. We first show
the translation of a particular cluster. The translation does not depend on the
structure of the clause besides the guard and clause body itself, so we omit
them.

\begin{align*}
  \bot = e_1 &\defeq e_1 \\
  g_1 = e_1 \le g_2 = e_2 \le \dots \le g_n &\defeq \caseof{g_1}{\inl{()}}{e_1}{\inr{()}}{g'} \\
    &\text{When $g'$ is the translation of $g_2 \le \dots \le g_n$}
\end{align*}

\begin{align*}
  \{ f~e_{i_1}~\dots~e_{i_m} \mid g_i = e_i \mid 1 \le i \le n \} \defeq
  \{ f~e_{i'_1}~\dots~e_{i'_m} = g_i' \mid 1 \le i' \le m' \} \\
    \text{When $g'$ is the guard-translation for the cluster $i'$.}
\end{align*}

An example of translation of guards is presented in
Fig.~\ref{fig:guards_translation}.

\begin{figure}[ht!]
  \centering
  \begin{subfigure}[b]{0.90\textwidth}
  \begin{rfuncodenum}
tryPred :: $\mu X . 1 + X$
tryPred x | case unroll [$\mu X . 1 + X$] $x$ of
              inl(()) => inl(()),
              inr($x'$) => inr(()) = inl(())
tryPred x | otherwise = let $x'$ = unroll [$\mu X . 1 + X$] $x$ in inr($x'$)
  \end{rfuncodenum}
  \caption{Light program.}
  \end{subfigure}
  ~
  \begin{subfigure}[b]{0.90\textwidth}
    \begin{rfuncodenum}
tryPred :: $\mu X . 1 + X$
tryPred x = case unroll [$\mu X . 1 + X$] $x$ of
              inl(()) => inl(())
              inr(()) => let $x'$ = unroll [$\mu X . 1 + X$] $x$ in inr($x'$)
    \end{rfuncodenum}
  \caption{Core program.}
  \end{subfigure}
  \caption{A translation of guards.}\label{fig:guards_translation}
\end{figure}

\section{Type Classes}

Type classes, introduced in~\cite{Wadler:1989} and later popularised in
Haskell~\cite{Hall:1996}, are aimed at solving overloading of operations by
allowing types to implement or infer a type class. A type class $\kappa$ is a
collection of function names (called \emph{operations}) with accompanying type
signatures $\{ f \Rightarrow \tau_f \mid f \in \kappa \}$, which are the
operations to be inferred when the type class is instantiated for a type
$\alpha$ (where $\alpha$ is a type variable). $\tau_f$ may naturally contain
$\alpha$. The syntax for type classes is as follows

\begin{align*}
  \textbf{class } \kappa~\alpha \textbf{ where } [f \Rightarrow \tau_f]^+
\end{align*}

Where $[\cdot]^+$ denotes one or more instances of $\cdot$ (in this case
functions and function signatures which belong to the class). Functions with
generic types which apply types on operation which belong to a type class can
be provided a \emph{context} which specifies that a type class instance is
written for that type.

\begin{align*}
  f~\kappa~\alpha \Rightarrow \alpha~.~(x_1:\tau_1) \dots (x_n:\tau_n) = e
\end{align*}

Where the types $\tau_1 \dots \tau_n$ may include $\alpha$ and $e$ may contain
applications of operations from $\kappa$ on terms of type $\alpha$. An instance
of a particular type class $\kappa$ for a type $\tau$ then instantiates the
operations for $\tau$, allowing us to call $f$ with $\tau$ substituted for
$\alpha$.

\begin{align*}
  \instance{\kappa~\tau}{[f~x^+ \Rightarrow e]^+}
\end{align*}

Where for each $f \in \kappa$ we have that there is a correlation between the
number of parameters $x$ in the operation implementation and the number of
types in the signature for $f$. For simplicity we require that operation names
are unique. That is, operation names are accessible through the global function
scope as any other function.

A function application with an overloaded function name still requires that the
type variable be manually applied, as type classes do not give any notion of
type inference. This is clear from the following example, as a function's
context may include the same type class for two different type variables. We
show the example in the syntax given by a top level function case:

\begin{align*}
  &\begin{array}{l}
  f :: \kappa~\alpha,~\kappa~\beta \Rightarrow \alpha~\beta~.~\alpha \rightarrow \beta \\
  f~x~y = g~y
  \end{array}
  &\text{Where $g \in \kappa$}
\end{align*}

Which instance for $g$ between $\alpha$ and $\beta$ should be dispatched to?
Although it is obvious that the correct instantiation is $\beta$ in this case,
it requires type inference for $y$, which is not considered in this work.

In the original introduction of type classes, instances of type classes may
declare an \emph{instance context} $\theta$ for the type $\tau$ which the
instance is being declared for. An instance context dictates which instances
must exist for the types which comprise $\tau$ (including $\tau$ itself),
before the instance can be written. For example, this allows us to describe
that we may write an instance for a type class $\kappa$ of \texttt{List}
$\alpha$ iff. $\alpha$ is already a member of $\kappa$ --- denoted
$\kappa~\alpha \Rightarrow \kappa~(\texttt{List}~\alpha)$.  This has two
functions:

\begin{enumerate}

  \item We can describe a hierarchy of type classes although we only require
    one type class to be instantiated in the context of a function.

  \item We can write type classes for types which have free type variables,
    when the type class' operations require that the type variable also is an
    instance of the same type class.

\end{enumerate}

We will omit instance contexts in this work for simplicity. This severely
hinders the usefulness of type classes as we cannot write type class instances
for types which have free type parameters (like \texttt{List}), but this
limitation can easily be remedied in a more extensive implementation.

We will now show a type class translation scheme. The principle is to strip
away all definitions and instances of classes and treat each instance member
as a top level function definition. An obvious method is to create unique
functions for each instance which specialize for that instance type. We can do
this as the class provides us with the function type and the instance provides
us with the function implementation. We have:

\begin{align*}
  &\begin{array}{l}
    \textbf{class } \kappa~\alpha \textbf{ where } \Rightarrow \alpha \leftrightarrow \tau' \\
    \instance{\kappa~\tau}{f~x \Rightarrow e}
  \end{array} \defeq
  f_\tau~(x:\tau) = e &
  \text{$f_\tau$ is a fresh function name}
\end{align*}

Additionally, any function $g$ in which an application of a member function $f$
from a class $\kappa$ occurs, we need to change the application to the newly
generated $f_\tau$. There are two distinct transformations we must consider here:

\begin{enumerate}

  \item $\kappa$ is in the context of $g$. This means That $g$ takes a type
    variable which is restricted by $\kappa$, and some class member function of
    $\kappa$ occurs in the body of $g$. As we wish to remove the context, the
    type variable shall be removed as well, and any application of it shall
    instead call the member function of $\kappa$. This means $g$ must also be
    made into a fresh function.

    \begin{align*}
      &\begin{array}{l}
        g :: (\kappa~\alpha) \Rightarrow \alpha~.~\alpha \\
        g~x = f~\alpha~x
      \end{array}
      \defeq
      \begin{array}{l}
        g :: \tau \\
        g_\tau~x = f_\tau~x
      \end{array} \\
      &\qquad\text{When $\tau$ is an instance of $\kappa$ and $f$ is a member of $\kappa$}
    \end{align*}

  \item An application of a class member function $f$ of $\kappa$ occurs in
    $g$, but with a valid concrete type (meaning, a concrete type for which an
    instance exists). In this case the type variable application should just be
    removed and the correct member function be applied instead:

    \begin{align*}
      &g~(x:1) = f~1~x \defeq g~(x:1) = f_1~x \\
      &\qquad\text{When $1$ is an instance of $\kappa$ and $f$ is a member of $\kappa$}
    \end{align*}

\end{enumerate}

A strong benefactor of type classes is equality testing. Equality as a
type class can, with variants, replace the duplication/equality operator
through the following definition of equality:

\begin{rfuncode}
  Eq 'a = Eq | Neq 'a
\end{rfuncode}

Which should be interpreted as: Either we have that $x = y$ and thus
\lstinline{eq $x$ $y$ => Eq}, or we have that $x \neq y$ and
\lstinline{eq $x$ $y$ => Neq $y$}. Recall that since $x$ is an ancilla, it
is tacitly returned from an equality check.  Thus when $x = y$, $y$ can be
destroyed as its original value is preserved in $x$ --- otherwise, we have to
remember $y$. In the example in Fig.~\ref{fig:type_classes_translation} we
present a simpler version of an Equality type class which states equality of
values as a form of attribute of two values which consumes neither value. An
attribute in this fashion can always be extracted by transforming a unit value,
which explains the signature of \texttt{eq}.

\begin{figure}[ht!]
  \centering
  \begin{subfigure}[b]{0.90\textwidth}
    \begin{rfuncodenum}
class Eq $\alpha$ where
  eq => $\alpha$ -> $\alpha$ -> 1 <-> (1 + 1)

instance Eq ($\mu X . 1 + X$) where
  eq $n_0$ $n_1$ () => case unroll [$\mu X . 1 + X$] $n_0$ of
                   inl() = case unroll [$\mu X . 1 + X$] $n_1$ of
                     inl() => inl(())
                     inr($n_1'$) => inr(())
                   inr($n_0'$) = case unroll [$\mu X . 1 + X$] $n_1$ of
                     inl() => inr(())
                     inr($n_1'$) => eqInt $n_0'$ $n_1'$ ()

compare (Eq $\alpha$) => $\alpha$ ($x_1$: $\alpha$) ($x_2$: $\alpha$) = eq $\alpha$ $x_1$ $x_2$ ()

eqNil ($x$: $\mu X . 1 + X$) = compare ($\mu X . 1 + X$) $x$ (roll [$\mu X . 1 + X$] inl(()))
    \end{rfuncodenum}
    \caption{Light program.}
  \end{subfigure}
  ~
  \begin{subfigure}[b]{0.90\textwidth}
    \begin{rfuncodenum}
eqNat ($n_0$: $\mu X . 1 + X$) ($n_1$: $\mu X . 1 + X$) (): 1 = case unroll [$\mu X . 1 + X$] $n_0$ of
  inl() = case unroll [$\mu X . 1 + X$] $n_1$ of
    inl() => inl(())
    inr($n_1'$) => inr(())
  inr($n_0'$) = case unroll [$\mu X . 1 + X$] $n_1$ of
    inl() => inr(())
    inr($n_1'$) => eqNat $n_0'$ $n_1'$ ()

compareNat ($x1$: $\mu X . 1 + X$) ($x2$: $\mu X . 1 + X$) = eqNat $x1$ $x2$ ()

eqNil ($x$: $\mu X . 1 + X$) = compareNat $x$ (roll [$\mu X . 1 + X$] inl(()))
    \end{rfuncodenum}
    \caption{Core program.}
  \end{subfigure}
  \caption{Translation of type classes.}\label{fig:type_classes_translation}
\end{figure}

\section{Records}

Records are products of labeled elements of arbitrary types. They generalize
products and product indexing by supporting component-wise projections and
local updates.~\cite{Cardelli:1992} considers \emph{extensible records} for
System F with proper subtyping and equality for records which may be extended
(by a label which is a type variable) as well as \emph{simple records}
(records of a fixed label cardinality). We will adopt a notion similar to a
simple record in \rfunc.

The first major deviation from the simple records of~\cite{Cardelli:1992} we
will enforce is totality of a record instantiation. Simple records as described
in the literature allow records to be partially populated, using \emph{row
variables} to postpone instantiation of any number of record components. We
enforce totality because we look to guarantee that we can directly translate a
record into the core language by writing an ordered $n$-ary product, where $n$
is the record's cardinality. Note that totality exempts us from having to define a
subtyping relation between records of different ``definitiveness''. Combining
simplicity and totality of records means we can completely omit subtyping as a
necessary property of records.

Second, we disallow arbitrary projections, for the obvious reason that
projections destroy information. Notice that projections were omitted in
\rfunc's grammar as well, a pivotal difference from regular functional
programs, necessary to maintain reversibility. What we instead want to do to
support projections is to introduce a special type of scope, instantiated for a
specific record $\gamma$ individually, in which we may fetch and change any
number of labels for $\gamma$. Any remaining labels are then automatically
assumed to be invariant.

A record is defined as

\begin{align*}
  \gamma = \{ l_1 :: \tau_1, \dots, l_n :: \tau_n  \}
\end{align*}

Where $\gamma$ ranges over record names. A record value is constructed by
assigning a variable name to a record constructor amongst the records which
have been defined. Each label of the record needs to be supplied a value (or
expression which evaluates to a value) of the type declared for that label in
the record definition.

We denote $\pi$ as a permuting function over a set of assignments to labels, so
that labels may be assigned in any order. The full set of permuting functions
$l_n$ for any given label cardinality $n$ is the product of possible assignment
orders. An exemplary permuting function $\pi \in l_4$ is:

\begin{align*}
  \begin{array}{llll}
    \pi(1) = 3, & \pi(2) = 1, & \pi(4) = 2, & \pi(1) = 4
  \end{array}
\end{align*}

A permutation is necessarily bijective. The identity permutation
$\text{id}_\pi$ sends every label to itself, so $\text{id}_\pi (a) = a$. We
have

\begin{align*}
  \pi(\{ l_1 = c_1, \dots, l_n = c_n \})
    = \{ \pi(l_1 = c_1), \dots, \pi(l_n = c_n) \}
\end{align*}

Because a permutation is bijective, any particular permutation has an inverse
so that $\pi^{-1} \pi = \text{id}_\pi$, meaning we can always order the label
assignments in a record declaration by applying an inverse permutation. A full
construction of a record $\gamma_i = \{ l_1 :: \tau_1, \dots, l_n :: \tau_n \}$
can now be described as

\begin{align*}
  &\lett{r}{\gamma_i~\pi(\{ l_1 = c_1, \dots, l_n = c_n \})}{e}
  & \text{where } c_1 : \tau_1, \dots, c_n : \tau_n \text{ and } \pi \in l_n
\end{align*}

We now present a translation scheme for record construction. Recall that we
denote the $n$-ary product $\tau_1 \times \cdots \times \tau_n$ as nested
application of binary products. Now, we first need to find the $\pi^{-1}$ for
whatever permutation was induced for the construction of $\gamma_i$. This can
easily be done as we want to invert the position of each label into the
ordering of the labels in the record definition, which is statically decidable.
After the fact that the product is ordered, this turns projection into a
deterministic traversal into the $n$-ary product.

Now we introduce the \emph{record scope}. In the record scope we may freely
extract and update values within the record for which we open a record scope.
The syntax is

\begin{align*}
  \within{\gamma}{e^\rho}
\end{align*}

$e^\rho$ is an extended grammar for expressions where we add the following
expression constructs:

\begin{itemize}

  \item $e ::= \rho(\gamma, l_i)$

  \item $e ::= \lett{\rho(\gamma, l_i)}{e}{e}$

\end{itemize}

Where $\rho(r, l_i)$ is a projection which returns the cell within the record
$\gamma$ for the label $l_i$. We denote $\rho(\gamma, l_i)$ as $\gamma.l_i$ in
the syntax. The two expression constructs thus respectively read the value
associated with a label in a record and write to a label associated with a
record. We expect that the value returned from a record scope is the record
itself. Because we still allow arbitrary expressions in the record scope, we
want the inner expression to result in one of two things: 1) The variant
itself, 2) A variant assignment.  Any other expression is ill-formed.

In the transformation of a record scope we initially expose the full structure
of the record at the entry point to the scope, which in itself allows
projections as the variable names are now uncovered. The translation of an
expression $\rho(\gamma, l_i)$ is substituted with the $i$'th component of the
record product:

\begin{align*}
  &\within{\gamma}{\gamma} \defeq \lett{(x_1, \dots, x_n)}{\gamma}{(x_1, \dots, x_n)} \\
  &\qquad\text{When $\gamma$ is a variable which a record is bound to and $|\gamma| = n$}
\end{align*}

An assignment to a record cell is translated into a regular let-expression with
an assignment to a fresh variable. The exit of the record scope then
reconstructs the record product with the new assignments in place of the old
cells.

\begin{align*}
  &\within{\gamma}{\lett{\gamma . l_1}{\gamma . l_1}{\gamma}} \defeq
  \lett{(x_1, \dots, x_n)}{\gamma}{\lett{x_1'}{x_1}{(x_1', \dots, x_n)}} \\
  &\qquad\text{When $\gamma$ is a variable which a record is bound to and $|\gamma| = n$}
\end{align*}

We show the transformation of a declaration of three-dimensional
\texttt{Vector} record and a function which uses it.

\begin{figure}[htp]
  \centering
  \begin{subfigure}[b]{0.90\textwidth}
    \begin{rfuncodenum}
Vector = { x: $\mu X . 1 + X$, y: $\mu X . 1 + X$, z: $\mu X . 1 + X$ }

$f~(x: \mu X . 1 + X)~(y: \mu X . 1 + X)~(z: \mu X . 1 + X)$ =
  let $v$ = Vector { x = $x$, y = $y$, z = $z$ }
  in within $v$:
       $v.x$ = roll [$\mu X . 1 + X$] inr($v.x$)
     end
    \end{rfuncodenum}
    \caption{Light program.}
  \end{subfigure}
  \begin{subfigure}[b]{0.90\textwidth}
    \begin{rfuncodenum}
$f~(x: \mu X . 1 + X)~(y: \mu X . 1 + X)~(z: \mu X . 1 + X)$ =
  let $v$ = $(x, (y, z))$
  in let $x'$ = roll [$\mu X . 1 + X$] inr($x$)
    in $(x', (y, z))$
    \end{rfuncodenum}
    \caption{Core program.}
  \end{subfigure}
  \caption{Translation of records.}\label{fig:record_translation}
\end{figure}

\section{Arbitrarily Sized Products}

\rfunc only supports binary products in product construction and when pattern
matching on the left-hand side of let and case-expressions, as can be noted in
the language grammar. Here, we will detail a method of pattern matching on
arbitrarily sized products in a single expression. There are essentially three
locations we may encounter $n$-ary products:

\begin{enumerate}

  \item While constructing product values. Example:

    \begin{align*}
      f~(x: \tau) = (x, e_2, e_3)
    \end{align*}

    Translating a constructed product with $n$ elements is as simple as
    repeatedly wrapping the right component into nested products where any
    inner product is solely binary. We define a recursive translation operation
    $\langle \cdot \rangle$ as:

    \begin{align*}
      \langle (e_1, e_2) \rangle &\defeq (e_1, e_2) \\
      \langle (e_1, e_2 \dots, e_n) \rangle &\defeq (e_1, (\langle e_2, \dots e_n \rangle))
    \end{align*}

  \item When pattern matching on the left-hand side of a let-expression. Example:

    \begin{align*}
      g~(x: \tau) = \lett{(a, b, c)}{f~x}{(a, b, c)}
    \end{align*}

    Here we need to exploit the syntax of the let-expression of the core
    language, which binds expressions of a product type on the left hand side
    of the assignment. We can unfold the full product by introducing a
    let-expression for each element of the product with a fresh variable name
    in the right component and the values of the product in the left component.
    We have:

    \begin{align*}
      \langle \lett{(x_1, x_2)}{e_1}{e_2} \rangle &\defeq \lett{(x, y)}{e_1}{e_2} \\
      \langle \lett{(x_1, x_2, \dots, x_n)}{e_1}{e_2} \rangle &\defeq \textbf{let }(x_1, x') = \langle \lett{(x_2, \dots x_n)}{x'}{e_2} \rangle \\
        &\text{Where $x'$ is a fresh variable}
    \end{align*}

  \item When pattern matching on the left-hand side of a case-expression.
    Example:

    \begin{align*}
      \caseof{h~x}{\inl{(a, b, c)}}{(a, b, c)}{\inr{x}}{x}
    \end{align*}

    In this example we expect $x$ to be of some type $\tau_1 \times \tau_2
    \times \tau_3 + \alpha$. An analogous condition exists for a pattern match
    for the right branch.

    Case-expressions actually cannot pattern match over values at all, so we
    introduce a pattern matching let-expression immediately in the body $e$
    when we unfold a product. This allows us to use the translation from (2.)
    immediately after. We introduce two rules, one for each branch. Should we
    need to pattern match in both branches the rules can be used in unison:

    \begin{align*}
      &\langle \caseof{x}{\inl{(e_1, \dots, e_n)}}{e_1}{\inr{e'}}{e_2} \rangle \\
      &\defeq \caseof{x}{\inl{x'}}{\langle \lett{(e_1, \dots, e_n)}{x'}{e_1} \rangle}{\inr{e'}}{e_2} \\
      &\langle \caseof{x}{\inl{e'}}{e_1}{\inr{(e_1, \dots, e_n)}}{e_2} \rangle \\
      &\defeq \caseof{x}{\inl{e'}}{e_1}{\inr{x'}}{\langle \lett{(e_1, \dots, e_n)}{x'}{e_2} \rangle} \\
    \end{align*}

\end{enumerate}

Where $x'$ is a fresh variable. An example of arbitrarily sized products is
presented in Fig.~\ref{fig:products_translation}.

\begin{figure}[ht!]
  \centering
  \begin{subfigure}[b]{0.90\textwidth}
    \begin{rfuncodenum}
tripDup $(x: 1)$ = $(x, x, x)$

$f$ $(x: 1)$ = let $(y, z, w)$ = tripDup $x$
           in case inl(($y, z$)) of
              inl(($y', z'$)) => ($y', z', w$)
              inr(()) => ((), (), ())
    \end{rfuncodenum}
    \caption{Light program.}
  \end{subfigure}
  ~
  \begin{subfigure}[b]{0.90\textwidth}
    \begin{rfuncodenum}
tripDup $(x: 1)$ = $(x, (x, x))$

$f$ $(x: 1)$ = let $(y, t_1)$ = tripDup $x$
            in let $(z, w)$ = $t_1$
               in case inl(($y, z$)) of
                 inl($t_2$) => let ($y', z'$)) = $t_2$ in ($y', z', w$)
                 inr(()) => ((), ((), ()))
    \end{rfuncodenum}
    \caption{Core program.}
  \end{subfigure}
  \caption{Translation of arbitrarily sized products.}\label{fig:products_translation}
\end{figure}

\section{Multiple Let Bindings}

Multiple variable bindings in a row are currently achieved using a nested chain
of let-expressions. This can without further ado be reduced to a single
let-expression in which with every binding occurs first, followed by an
evaluation of the final expression. We have:

\begin{align*}
  \textbf{let } x_1 = e_1, \dots, x_n = e_n \textbf{ in } e
\end{align*}

And we define a translation as:

\begin{align*}
  \langle \lett{x_1}{e_1}{e} \rangle &\defeq \lett{x_1}{e_1}{e} \\
  \langle \textbf{let } x_1 = e_1, x_2 = e_2 \dots, x_n = e_n \textbf{ in } e \rangle &\defeq
  \lett{x_1}{e_1}{\langle \textbf{let } x_2 = e_2 \dots, x_n = e_n \textbf{ in } e} \rangle
\end{align*}

We show a translation of multiple nested let-expressions as a continuation of
the map function, which was shown before as an example of a translation of
top-level function clauses, in Fig.~\ref{fig:let_translation}.

\begin{figure}[htp]
  \centering
  \begin{subfigure}[b]{0.90\textwidth}
    \begin{rfuncodenum}
map $\alpha$ $\beta$ ($f: \alpha \leftrightarrow \beta$) ($xs: \mu X . 1 + \alpha \times X$) = case unroll [$\mu X . 1 + \alpha \times X$] of
  inl(()) => roll [$\mu X . 1 + \alpha \times X$] inl(()),
  inr($xs'$) => let ($(x, xs'')$) = $xs'$
                    $x'$ = $f$ $x$
                    $xs'''$ = map $f$ $xs''$
               in cons $x'$ $x'''$
      \end{rfuncodenum}
    \caption{Light program.}
  \end{subfigure}
  \begin{subfigure}[b]{0.90\textwidth}
    \begin{rfuncodenum}
map $\alpha$ $\beta$ ($f: \alpha \leftrightarrow \beta$) ($xs: \mu X . 1 + \alpha \times X$) = case unroll [$\mu X . 1 + \alpha \times X$] of
  inl(()) => roll [$\mu X . 1 + \alpha \times X$] inl(()),
  inr($xs'$) => let ($(x, xs'')$) = $xs'$
               in let $x'$ = $f$ $x$
                   in let $xs'''$ = map $f$ $xs''$
                       in cons $x'$ $x'''$
      \end{rfuncodenum}
    \caption{Core program.}
  \end{subfigure}
  \caption{Translation of nested let-expressions.}\label{fig:let_translation}
\end{figure}


%-------------------------------------------------------------------------------
% IMPLEMENTATION
%-------------------------------------------------------------------------------

\chapter{Implementation}\label{sec:implementation}

The language described in the work has been implemented in Haskell as a
reference implementation. It includes parsers for both the core and light
grammars, a typechecker implementing the base type system from
Sect.~\ref{sec:formal} including every extension showed later, an interpreter
implementing the big step operational semantics, an inverse interpreter
implementing the inverse semantics, a first match policy analyzer and a
transpiler transforming a light program into the core program.

The implementation involves a number of monads, specifically a Reader, Writer,
State and Error monad. Details on these can be found in~\cite{Jones:1995}. We
also employ a monadic parser combinatoric approach via the \texttt{megaparsec}
library. An introduction to monadic parsing in Haskell can be found
in~\cite{Hutton:1998}.

In the following we will discuss various details about the implementation,
including ways in which the implementation diverges from the theoretic
description.

\section{Parsing}

We implement two parsers and maintain two abstract syntax trees: one for the
light language and one for the core language. Both languages are indentation
oblivious but white space sensitive to some extent. Specifically, we allow line
breaks only under certain conditions. Canonical terms and applications must be
retained on the same line while case expressions recognize branches on separate
lines and let-expressions allow the binding $l = e_1$ be separated from the
body $e_2$, so we can write:

\begin{rfuncode}
case $x$ of
  inl($y$) -> y,
  inr($z$) -> z

let $x$ = $e$
in $e$
\end{rfuncode}

In the core language, parsing case-expressions is deterministic without white
space sensitivity or explicit delimitation of the expression. This is because
we know we have to account for exactly two branches and each left hand side is
uniquely given. In the light language, we do not have this insurance as a
case-expression may involve any number of branches and their left hand side is
unrestricted. Consider the program in Fig.~\ref{fig:ambiguous_parsing}.

\begin{figure}[ht!]
  \centering
  \begin{subfigure}[b]{0.49\textwidth}
    \begin{rfuncode}
           case $x$ of
             v1 -> case v1 of
                     v11 -> v11,
                     v12 -> v12,
                     v13 -> v13,
             v2 -> case v2 of
                     v21 -> v21,
                     v22 -> v22
    \end{rfuncode}
  \caption{}
  \end{subfigure}
~
  \begin{subfigure}[b]{0.49\textwidth}
    \begin{rfuncode}
        case $x$ of
          v1 -> case v1 of
                  v11 -> v11,
                  v12 -> v12,
                  v13 -> v13,
                  v2  -> case v2 of
                           v21 -> v21,
                           v22 -> v22
    \end{rfuncode}
  \caption{}
  \end{subfigure}
  \caption{Ambiguous program.}\label{fig:ambiguous_parsing}
\end{figure}

Note the comma of $\texttt{v13} \rightarrow \texttt{v13,}$ on line 5. In
subfigure (a) the indentation levels seem to indicate that the first inner
case-expression only has three alternatives --- so the comma delimits the two
outer branches, and we can begin to parse the second arm $\texttt{v2}
\rightarrow \dots$. But the meaning is ambiguous: It can also be parsed to mean
that the case-expression in the first inner branch has another alternative,
which is the natural interpretation in subfigure (b).

Capturing the correct meaning requires indentation sensitiveness, which we do
not implement. Therefore we delimit a light case expression with an additional
\textbf{esac} to erase this ambiguousness. We rewrite the case-expression from
Fig.~\ref{fig:ambiguous_parsing} as:

\begin{rfuncode}
case $x$ of
  v1 -> case v1 of
          v11 -> v11,
          v12 -> v12,
          v13 -> v13
        esac,
  v2 -> case v2 of
          v21 -> v21,
          v22 -> v22
        esac
  esac
\end{rfuncode}

\section{Typechecking}

The typechecker employs the State monad to maintain the set of hypotheses,
mapping types to variables. We partition the hypotheses into a static and
dynamic fragment to reflect $\Gamma$ and $\Sigma$. Without loss of correctness
we deviate slightly from the typing judgement by storing function signatures in
its own environment in the form of a Reader monad. The Error monad ensures that
when any error has been encountered, it will propagate to the top typechecker
entry point with a designated error message. This choice of error handling
promises that the first and only the first error encountered will be brought to
attention --- there is no tally of type errors.

Contrary to what is assumed in the type rules --- where the type signature
$\tau_f \rightarrow \dots \rightarrow \tau \leftrightarrow \tau$ of a function
is assumed to be fully known at the time of typing --- the return type of
functions is not known initially in the implementation. Obligatory annotations
of types of parameters ensure that we know every parameter type, but we will
have to infer return types. For this we heed to \emph{constraint solving}. We
define a special kind of type variable called a \emph{unification variable},
denoted $\{i\}$, where $i$ is the identity of a particular unification
variable. The informal idea is to find the most general form of each
unification variable by inspection of how they are used in each function body
expression. A fresh unification variable is introduced for each expression we
do not immediately know the full type of. In the present language, unknown
types occur solely by function applications and by sum term. Specifically,
$\inl{e}$ is typed as $\tau + \{i'\}$ where $i'$ is a fresh unification
variable and $\con e : \tau$.  Symmetrically we have $\inr{e}$ as $\{i'\} +
\tau$. Further, each function name $f$ also omits a unique unification variable
$\{f\}$, meaning every function application types as $\{f\}$.

A constraint on a unification variable $\{i\}$ is produced whenever we learn
anything about $\{i\}$ by how an expression $e : \{i\}$ is subsequently
treated, which might generate new \emph{unification variables}. Exemplary cases
are:

\begin{align*}
  &\lett{(x, y)}{e : \{i\}}{e'} & \text{Cons}(i = i' \times i'') \text{ with $x : \{i'\}$, $y : \{i''\}$} \\
  &\caseof{e : \{i\}}{\inl{x}}{e'}{\inr{y}}{e''} & \text{Cons}(i = i' + i'') \text{ with $x : \{i'\}$, $y : \{i''\}$} \\
\end{align*}

Each constraint is remarked in a global record of constraints. After the type
of each function $f \in p$ has been deduced, we know that every constraint has
been produced for $p$. For constraint solving to succeed, no two constraints
may be in direct contradiction. In the reference implementation we use the
Writer monad to transcribe constraints during typing, and solve them as a
subsequent operation to typing.

\section{Interpretation and Static Analysis}

The implementation of forward interpretation is rather straightforward --- it
is implemented basically as presented in the big step semantics. It employs the
Reader monad to represent $p$, the storage of function definitions. No state is
needed as mappings between variables and values is achieved through
substitution. We expect every program to be well typed. This specifically means
that the only run-time error which \emph{should} occur is a failure of the
\textsc{Case-R} side condition --- in this case, an error is evoked via the
Error monad.

Inverse interpretation introduces a state monad in addition to $p$,
representing the store $\sigma$ as seen in the inverse big step judgement. A
simple first match policy analyser implementation is also made available.

\section{Transformation}

Transformations are performed serially in a pipeline fashion. In theory,
transformations in Sect.~\ref{sec:prog} are (in almost all cases) defined so
that each transformation is layered directly on top of the core language,
something which is not feasible when they are all combined into one grammar.
The exception to the rule is the transformation of guards, which are defined on
top of a top-level function case. The reference implementation takes some
liberties on this front and constructs the transformations on top of each
other to simplify the implementation, while keeping true to the semantics of
each. The order of transformations is:

\begin{align*}
  tProducts \circ tLetIns \circ tVariants \circ tTopFunc \circ tGuards \circ tTypeClasses
\end{align*}

We do not implement the translation of records, and thus do not support them in
the reference implementation. The reason is that the translation scheme for
records exploits that we know the full type signatures of functions to unpack
record values. But we have seen how we need to infer return types via
typechecking in the reference implementation. Ultimately, we cannot typecheck a
program until after transformations have been applied but we apply a
transformation for records until we know the return type of each function. A
workaround would be to allow the programmer to to add return type annotations
to each function signature.

\subsection{Typechecking of Light Programs}

There is a balance to be struck regarding the extent to which the
implementation should validate the non-transformed program. A good rule of
thumb is to attempt to perform the translation as long as the transformation
rules are possible to apply --- in the most rudimentary interpretation, we at
least have to take the precautions addressed in the main text on
transformations into account.

An open issue regarding typechecking a light program is the tackling the
presentation of cryptic error messages which will be encountered when there is
an error during typechecking of a transformation program. If the transformation
is not stateful in a way such that the original program text is preserved and
accessed when an error happens, the error will not reflect what the programmer
wrote at all. This has been left as future work.

\subsection{Transformation Example}

As an example, we present a reversible zip function $\texttt{zip} ::
\alpha~\beta .~[\alpha] * [\beta] \rightarrow [\alpha * \beta]$ (with $[\cdot]$
being the usual shorthand for lists) actually translated by the reference
implementation. \texttt{zip} is a non-trivial example of an \rfunc function, as
we have to decide the behaviour when the lists are of uneven size. We cannot
omit the clauses as the case-expressions need to be total and we cannot throw
away excess values from the longer list. What we do instead is constrict the
function to lists of types which instantiate the \texttt{Default} type class
--- types for which we can define default values. The smaller list is then
padded until it has the same length as the longer list. An example for lists of
natural numbers, where the default value is 0:

\begin{align*}
  \texttt{zip}~[1, 2, 3, 4]~[5,6] = [(1, 5), (2, 6), (3, 0), (4, 0)]
\end{align*}

The original program can be seen in Fig.~\ref{fig:example_light} and the
generated core program can be seen in Fig.~\ref{fig:example_transformed}. We
have added line breaks and white spaces in the presentation of the transformed
program where appropriate, to make the structure of the program more clear.
Note that the presentation breaks the parser's rules regarding white spaces, so
it is only exemplary. Also note that the transformation breaks the naming
convention of every class of fresh variables it generates. These variable names
are well formed as a program, but cannot be parsed! This is a simple method of
ensuring uniqueness of new variables.

\begin{figure}[ht!]
  \begin{rfuncode}
type Nat = Null | Succ Nat
type List 'a = Nil | Cons 'a (List 'a)

class Default 'a where
  def => 1 <-> 'a
instance Default Nat where
  def u => roll [Nat] Null

zip :: Default 'a, Default 'b => 'a 'b . (List 'a) * (List 'b)
zip ls = let (xs, ys) = ls
            in case unroll [List 'a] xs of
              Nil -> case unroll [List 'b] ys of
                Nil -> roll [List ('a * 'b)] Nil,
                Cons y ys' -> let xdef = def 'a ()
                                  nil = roll [List 'a] Nil
                                  ls' = zip 'a 'b (nil, ys')
                              in roll [List ('a * 'b)] (Cons (xdef, y) ls')
                esac,
              Cons x xs' -> case unroll [List 'b] ys of
                Nil -> let ydef = def 'b ()
                           nil = roll [List 'b] Nil
                           ls' = zip 'a 'b (nil, xs')
                       in roll [List ('a * 'b)] (Cons (x, ydef) ls'),
                Cons y ys' -> let ls' = zip 'a 'b (xs', ys')
                              in roll [List ('a * 'b)] (Cons (x, y) ls')
                esac
            esac

unzip :: Default 'a, Default 'b => 'a 'b . List ('a * 'b)
unzip ls = zip! 'a 'b ls
  \end{rfuncode}
  \caption{A reversible \texttt{zip} program written in the light
  language.}\label{fig:example_light}
\end{figure}

\begin{figure}[ht!]
  \begin{rfuncode}
__zipNatNat  (ls:(\A . (1 + (\B . (1 + B) * A)) * \A . (1 + (\B . (1 + B) * A)))) =
  let (xs, ys) = ls
  in case unroll [\A . (1 + (\B . (1 + B) * A))] xs of
    inl(()) -> case unroll [\A . (1 + (\B . (1 + B) * A))] ys of
      inl(()) -> roll [\A . (1 + ((\B . (1 + B) * \B . (1 + B)) * A))] inl(()),
      inr(_a) ->
        let (y, ys') = _a
        in let xdef = __defNat ()
        in let nil = roll [\A . (1 + (\B . (1 + B) * A))] inl(())
        in let ls' = __zipNatNat (nil, ys')
        in roll [\A . (1 + ((\B . (1 + B) * \B . (1 + B)) * A))] inr(((xdef, y), ls')),
    inr(_a) ->
      let (x, xs') = _a
      in case unroll [\A . (1 + (\B . (1 + B) * A))] ys of
        inl(()) ->
          let ydef = __defNat  ()
          in let = nil roll [\A . (1 + (\B . (1 + B) * A))] inl(())
          in let = ls' __zipNatNat  (nil, xs')
          in roll [\A . (1 + ((\B . (1 + B) * \B . (1 + B)) * A))] inr(((x, ydef), ls')),
        inr(_b) ->
          let (y, ys') = _b
          in let = ls' __zipNatNat  (xs', ys')
          in roll [\A . (1 + ((\B . (1 + B) * \B . (1 + B)) * A))] inr(((x, y), ls'))

__unzipNatNat  (ls:\A . (1 + ((\B . (1 + B) * \B . (1 + B)) * A))) =
  __zipNatNat!  ls

__defNat  (u:1) = roll [\A . (1 + A)] inl(())
  \end{rfuncode}
  \caption{The reversible \texttt{zip} function from
  Fig.~\ref{fig:example_light} transformed by the reference
  implementation.}\label{fig:example_transformed}
\end{figure}

\section{Default Programs and Testing}

Initially, we wished to test the reference solution rigorously with something
like QuickCheck~\cite{Claessen:2011}, but soon realized that it is hardly
fitting for highly structured data. Instead, we include a large array of small
programs which are meant to test the limits of the parser, typechecker,
interpreter and transformer. Take note that this is a heuristic approach as we
have not \emph{proven} the correctness of any part of the reference
implementation. In addition to providing test programs, we include an
assortment of exemplary program with numerous examples of well formed functions
over standard data structures like lists and numerals.


%-------------------------------------------------------------------------------
% DISCUSSION
%-------------------------------------------------------------------------------

\chapter{Discussion}\label{sec:discussion}

\section{Updated Grammar}

We present the updated grammar of programs belonging to the light language in
Fig.~\ref{fig:updated_grammar}. We define the domains which variants, records
and type classes range over in Fig.~\ref{fig:domains}. The updated grammar is a
superset of the original grammar. Specifically, any program in the core
language is also a program in the updated language.

\begin{figure}[ht!]
\begin{align*}
         \kappa \in \text{Type class names}
  \qquad \beta  \in \text{Variant names}
  \qquad \gamma \in \text{Record names}
\end{align*}
\caption{Domains}\label{fig:domains}
\end{figure}

\begin{figure}[ht!]
\begin{tabular}{p{0.55\textwidth}p{0.45\textwidth}}
$q ::= d^*$
    & Program definition\\
$d ::= f$
    & Function definition\\
$\hspace{1.9em}\mid \class{\kappa~\alpha}{[f \Rightarrow \tau_f]^+}$
    & Type class definition\\
$\hspace{1.9em}\mid \instance{\kappa\ (\beta~\alpha^*)}{[f \Rightarrow e]^+}$
    & Type class instance\\
$\hspace{1.9em}\mid \beta~\alpha^* = [\texttt{v}~ [\tau\alpha]^*]^+$
    & Variant definition\\
$\hspace{1.9em}\mid \gamma = \{ [l :: \tau]^+ \}$
    & Record definition\\
$e ::= x$
    & Variable name\\
$\hspace{1.9em}\mid ()$
    & Unit term\\
$\hspace{1.9em}\mid \inl{e}$
    & Left of sum term\\
$\hspace{1.9em}\mid \inr{e}$
    & Right of sum term\\
$\hspace{1.9em}\mid \texttt{v}~e^*$
    & Variant term\\
$\hspace{1.9em}\mid (e_1, \dots, e_n)$
    & Product term\\
$\hspace{1.9em}\mid \textbf{let } [l = e]^+ \textbf{ in } e$
    & let-expression \\
$\hspace{1.9em}\mid \smpcase{e}{[e \Rightarrow e]^+}$
    & case-expression\\
$\hspace{1.9em}\mid \caseofs{e}{\inl{x}}{e}{\inr{y}}{e}{e}$
    & \text{Safe case-expression} \\
$\hspace{1.9em}\mid \caseofu{e}{\inl{x}}{e}{\inr{y}}{e}$
    & \text{Unsafe case-expression} \\
$\hspace{1.9em}\mid f~\alpha^*~e^+$
    & Function application\\
$\hspace{1.9em}\mid f!~\alpha^*~e^+$
    & Inverse Function application\\
$\hspace{1.9em}\mid \roll{\tau}{e}$
    & Recursive-type construction \\
$\hspace{1.9em}\mid \unroll{\tau}{e}$
    & Recursive-type destruction \\
$\hspace{1.9em}\mid \within{\gamma}{e^\rho}$
    & Record scope\\
$e^\rho ::= \rho(\gamma, l)$
    & Record projection \\
$\hspace{2.2em}\mid \rho(\gamma, l) = e^\rho$
    & Record cell assignment \\
$\hspace{2.2em}\mid \lett{\rho(\gamma, l)}{e^\rho_1}{e^\rho_2}$
    & Record let-assignment \\
$l ::= x$
    & Definition of variable\\
$\hspace{1.7em}\mid (x_1, \dots, x_n)$
    & Variable product\\
$f ::= f~\alpha^*~\tau^+ [f\ e^+ [\mid g]? = e]^+$
    & New Style Function Definition\\
$\hspace{1.9em}\mid f~\alpha^*~(x:\tau_a)^+ = e$
    & Old Style Function definition
\end{tabular}

\caption{Grammar for the updated language.}\label{fig:updated_grammar}
\end{figure}

\section{Omitted Abstractions}

Proposing additional translations  which could have been considered in
Sec.~\ref{sec:prog} is an indefinite process. Some lesser, mostly stylistic,
propositions will be mentioned as future work in Sec.~\ref{sec:future_work}.
Other translations are a bit more tricky. We present here two translations,
anonymous functions and infix operators, which were considered as translations
but are not as fitting to translate as they seem on the surface. They are
interesting as they do not necessarily work intuitively in a reversible
setting.

\subsection{Anonymous Functions}

Anonymous functions are nameless functions, modeled as abstractions from the
lambda calculus invented by Church~\cite{Church:1936}. An abstraction is
constructed from a variable name $x$ and an inner term $t$ where $x$ is a free
variable in $t$. An application $t~t'$ substitutes a term $t'$ for $x$ in $t$.
A term is closed when it contains no more free variables. Usually abstractions
are generalized to take any number of variables as arguments.

\begin{itemize}

  \item \textbf{Variable:} $x$

  \item \textbf{Abstraction:} $(\lambda x .~t)$

  \item \textbf{Application:} $t~c$

\end{itemize}

We would have to impose certain restrictions on where anonymous functions could
be declared to maintain the restricted higher order property of \rfunc. To
mirror how restricted higher order functions may be used in the core language,
they should only be constructable in two places in the light language: directly
as expressions supplied as an ancillary parameter during function application or
as function applications themselves.

An obvious translation of anonymous functions would be to construct a top-level
function for each anonymous function, taking care of name collisions. Each
newly constructed function would then be subjected to precisely one application
in the full program, at the position where it was previously defined.

The main issue with anonymous functions is determining the scope of variables
involved in the open term of the abstraction's body. \rfunc and its light
counterpart use lexical scoping, meaning the context in which the anonymous
function lives in is fully available. A contrived example could be:

\begin{rfuncode}
f $x$ = let $z$ = inr($x$)
   in let $y$ = (\ $x'$ -> ($z$, $x'$)) $x$
   in $y$
\end{rfuncode}

The only usage of the value $z$ is in the abstraction's body, even though $z$
is not supplied to the anonymous function as a parameter. In any conventional
functional language this has well-defined behaviour, and we expect it to be
well-defined in \rfunc as well. Alas, the abstraction body is removed from its
lexical scope when the translation occurs, as it is moved into a new top-level
function $f_\lambda$ --- meaning $z$ is no longer available for $f_\lambda$.
This indicates that we need a more thorough analysis of which values are
applied to the statically generated function during translation.

\subsection{Infix Operators and Numerals}\label{subsec:numerals}

Natural numbers can be added relatively easily as an abstraction over the unary
numerals. But they are only really practical if we simultaneously add
arithmetic operators. Arithmetic operators are a subset of infix binary
operators. An infix operator $\odot$ is reversible if we may transform either
operand while keeping the other intact, so $x \odot y = (x, x \odot y)$, and if
there is an inverse operation $\odot^{-1}$ so that $x \odot^{-1} (x \odot y) =
(x, y)$. The operator may be self-inverse, so $x \odot (x \odot y)) =
(x, y)$.

The following are amongst the arithmetic operators which are reversible, with
their inverses provided:

\begin{align*}
  (+)^{-1} &\defeq (-) \\
  (\times)^{-1} &\defeq (\div) \\
\end{align*}

Supporting infix operators requires non-trivial design choices because applying
the inverse operation on the resulting value of the forward operation might
require that we go against the intuition of how these operations work if the
inverse operator is not commutative. This is because the transformed value
necessarily is the right operand (because of the design of ancillae parameters
leaning left). This issue is evident from both operator pairs $+/-$ and
$\times/\div$, as we have:

\begin{align*}
  2 + 3 &= 5 \\
  2~+!~5 &= 3
\end{align*}

It appears as if $2 - 5 = 3$, if we naively define $(-)$ as the inverse of
$(+)$. The analogous scenario crops up for $(\div)$ as the inverse of
$(\times)$.  A possible fix would be to instead transform the left operand for
binary operations, but this really does not align well with the type system as
the type system dictates that the right most parameter of a function is always
transformed.

There are a couple of other options. We can forbid the declaration of new infix
operators and fix the behaviour of the operators we do define as additional
syntactic sugar. That is, we can define any infix operator into an equivalent
operator in Polish notation as $l \odot r \defeq (\odot)~r~l$, with the
operands reversed (which works fine in the forward direction as well as both
$+$ and $\times$ are commutative). This also makes it simple to transform the
operations we \emph{do} write, as we simply predefine the functions which
correspond to infix operators.

If we do allow the generation of new infix operators, the programmer should
have access to some mechanism through which they can define what precedence the
operands should have in the inverse direction or if the operator is
commutative.

\paragraph{Encoding Natural Numbers}

Natural numbers as numerals are generalized over the Peano numbers encoded by
the recursive type $\mu X .~1 + X$. They are translated the obvious way by
recursive descent on a numeral $n$ as a \textbf{roll}-term on a $\inl{n - 1}$
expressions until we reach the bottom numeral $0$, which is constructed as
$\inr{()}$. That this construction is correct can be shown with induction.

\begin{proof}

  The induction hypothesis states, by the principle of well-formed induction,
  that any element $n$ in the infinite domain has a non-infinite set of values
  which which it relates to, that is, the set $A = \{ n' \mid n' \prec n \}$ is
  finite. If it is finite there must be a least element $n_0$ for which
  $\forall n' \in A.\ n_0 \prec n'$. For the recursive type $\mu X . 1 + X$
  this is $\roll{\mu X . 1 + X}{\inl{()}}$, the first unary number, which is a
  well-formed expression in the core language.

  For the induction step we assume for any $n$ we have a well-formed expression
  which denotes $n$ in the core language and want to prove that we form a
  well-formed expression for $n + 1$. Then we can simply take the next Peano
  number. Denote $c$ the canonical representation of $n$. Then we take $n + 1$
  to be $\roll{\mu X . 1 + X}{\inr{c}}$.

\end{proof}

\section{Reversible Higher Order Language}\label{sect:higher_order}

We now move on to another famous advantage of functional languages:
higher-order functions. Higher-order functions are functions which take
functions as arguments or return functions. They are overall a sparsely
discussed topic in a reversible setting --- especially general higher order
reversible functional programming, though some sources exist~\cite{Bohne}.
More interesting are reversible effects, which allow various interesting
properties like reversible side effects and concurrency. Reversible effects
have nice models in Category Theory as inverse arrows~\cite{Heunen:2018} and
reversible monads~\cite{Heunen:2015}.

We have seen that taking functions as arguments is unproblematic, given we can
guarantee that the function is statically known. Then what is the roadblock
regarding arbitrary higher-order functions in a reversible setting?

We should reiterate the fact that our ultimate goal is to design a
\emph{garbage-free} language. Being garbage-free especially entails that the
amount of information we need to maintain for a program to remain reversible is
minimal. We also remind the reader about the notion of a \emph{trace}, which is
a particular strategy to transform any non-reversible program into a reversible
program by consistently store information critical to knowing the context in
which a computation took place.

Now, let us temporarily adopt the view that higher-order functions are allowed
in \rfunc. This implies that any variable is allowed to be assigned a function
value. Say we have a function \rfuninl{twice} which takes a function $f$ and
returns a function which applies $f$ twice, defined as $\texttt{twice}~f = f
\circ f$. Its type signature in an irreversible setting is $\texttt{twice}: (A
\rightarrow A) \rightarrow (A \rightarrow A)$. Evidently it transfers well to a
reversible setting, where we can define it as $\texttt{rtwice}: (A
\leftrightarrow A) \leftrightarrow (A \leftrightarrow A)$, as the function $f$
is transformed cleanly. Now, consider the following function:

\begin{rfuncode}
  g $x$ = let $f'$ = rtwice $f$
          in $f'$ $x$
\end{rfuncode}

Where $f$ is some static function defined for the program. We have that $g~x$
will return $c = f(f(x))$. But given just the canonical form $c$, the context
in which it was computed is not immediately clear. Consider computing $g^{-1}$
--- what is the function $f'$? In this particular instance we easily see that
it is the function generated by \rfuninl{twice} on $f$, but this information is
not known presently in the inverse direction. The issue becomes more evident
when we consider some arbitrary $f'$ which may have been dynamically introduced
\emph{anywhere}.

We are therefore forced to store information about \emph{how} a value was
produced if it is not clear from the syntactic construct itself, so we know how
to invert it. Doing this requires that we for a function application of a
dynamically introduced function return a \emph{closure} containing the
definition of the applied function $f'$ as well as the result of $f'~x$. The
closure is, as a value, going to be treated as $f'~x$ going forward, but
storing $f'$ becomes important for inverting the computation.

This one characteristic, which is unavoidable if we want to support full-on
higher order behaviour, adds what surmounts to a pseudo-trace with heavy use
of higher-order functions, making higher-order functions unattractive.

% \begin{itemize}
%
%   \item
%
%   \item Well, what do we do when we uncompute a new function? We must we able
%     to decouple the function and from the information which was used to
%     generate it. This information might just be the function itself.
%
%   \item If we use some other information --- whatever information it was, it
%     needs to be stored individually in the new function scope. This dictates
%     that when a new function is generated, it is actually a closure object.
%     Does any new function need to be uncomputed? It is new information after
%     all. What if we use it once for example?
%
%   \item So obviously, closures end in traces in some sort of way, where the
%     signatures of functions become convoluted because we keep needing to save
%     values which went into creating the new function. A good example is
%     currying. What does an irreversible currying function look like?
%
%     \begin{align*}
%       curry :: (A \times B \rightarrow C) \rightarrow (B \rightarrow (A \rightarrow C))
%     \end{align*}
%
%     $B$ and $A$ are switched as function application is right-associative.
%     Currying transforms a function on $n$ parameters into a function which can
%     take one parameter at the time, allowing for partial application of
%     parameters.
%
%     Let us transform it to something which we know from the types of \rfunc. We
%     write a function signature in for the type system as something which is
%     reminiscent of currying, but it is only a shorthand to ease the
%     formalization of application. Really, the type signature of a function is,
%     after static parameters are included:
%
%     \begin{align*}
%       f &: \tau_1 \rightarrow (\tau_2 \leftrightarrow \tau_3) \\
%       \Leftrightarrow f &: \tau_1 \times \tau_2 \leftrightarrow \tau_1 \times \tau_3 \\
%     \end{align*}
%
%     Now what is the type of currying?
%
%     \begin{align*}
%       \Leftrightarrow curry &:  (\tau_1 \times \tau_2 \leftrightarrow \tau_1 \times \tau_3) \leftrightarrow~?
%     \end{align*}
%
%     A currying function should turn the function which takes two parameters to
%     one where it is possible to supply one and still get something back, namely
%     a new function. But we should also be able to restore the function after
%     one parameter has been applied. Thus it might look something like this:
%
%     \begin{align*}
%       curry &:  (\tau_1 \times \tau_2 \leftrightarrow \tau_1 \times \tau_3)
%       \leftrightarrow (\tau_1 \leftrightarrow (\tau_1 \times (\tau_2 \leftrightarrow \tau_3)))
%     \end{align*}
%
%     So what should we be able to do with a currying function? Well, the
%     function should itself be salvageable from the new function, which is easy
%     as no parameter has been supplied. But what does each side look like? What
%     must a curried function guarantee about the parameters supplied to each
%     partial application? The syntax of \rfunc simply does not allow arbitrary
%     functions (or in fact, creation of new functions)
%
%     % The function (call it $f$) $curry$ is being called on must be reversible,
%     % so its signature should be $f: A \rightarrow (B \leftrightarrow C)$, as $A$
%     % is static. But after $A$ has been applied, how do we know what $A$ was? We
%     % need to be able to extract $A$ back. $A$ is static for the function $curry$
%     % is applied to, so
%
%   \item Let's move away from currying, it is very impractical. How about
%     function composition?
%
%     \begin{align*}
%       (\circ) :: (A \leftrightarrow B) \times (B \leftrightarrow C) \leftrightarrow (A \leftrightarrow C)
%     \end{align*}
%
%     We can use composition for a number of things. How about a function which
%     takes a function and applies it twice?
%
%     \begin{align*}
%       & twice :: (A \leftrightarrow A) \leftrightarrow (A \leftrightarrow A) \\
%       & twice~f = f \circ f
%     \end{align*}
%
%     Now a question is: is there even a good (possible) way to write higher
%     order functions in our language? It is not point free nor has an
%     application operator $(.)$ (as seen in Haskell), so how could we even
%     describe it? We really cannot I think, but if higher order functions were
%     something we would look to support, we could add it. I guess that is the
%     argument that should be made.
%
%     How about a function which calls the inverse? So a reversal function?
%
%     \begin{align*}
%       rev :: (A \leftrightarrow B) \leftrightarrow (B \leftrightarrow A)
%     \end{align*}
%
%     There seems to at least be some class of naturally injective functions
%     which might be composed/altered/generated. What do they have in common, if
%     anything? It might be something like: Functions which consume information
%     to create new functions need to still preserve the information in a
%     closure, so we gain nothing by them. Functions which only use functions as
%     parameters have obvious inverses, since the function parameters must
%     themselves be inverse. So $curry$ and $eval$ are ugly, while $rev$, $twice$
%     and $(\circ)$ are okay. Does it necessarily have something to do with
%     partial evaluation being problematic?
%
%   \item I have not really been able to find a formal argument made for why
%     higher order functions are hairy for reversible languages --- or at least
%     not in the literature, so as not to say that the argument is not there. I
%     am sure that it is just obvious from the theory itself and not quite worthy
%     of a lengthy discussion, due to the obviousness of the situation.
%
%   \item \cite{Bohne} discusses general higher order reversible functional
%     programming. It is the only source I could find. What does he do? The paper
%     is a bit confusing as the language incorporates a tier style of
%     reversibility to the possible functions, so they are not all invertible and
%     it can compute more than the injective functions. Still, it seems to
%     support the notion that for composed functions to be invertible, we look to
%     keep original functions to not throw away information, which means:
%     closures.
%
% \end{itemize}
%
% \subsection{The Category Theory}
%
% \begin{itemize}
%
%   \item Let's try to do some category theory: Symmetric monoidal categories lie
%     at the base of linear (and relevant logic). They are categories enriched
%     with a monoidal function (defined in the standard way) as well as a notion
%     of maximum commutativity, meaning that the tensor operator has a natural
%     isomorphism which ensures that it supports switch of object operands and
%     that the natural isomorphism has an inverse.
%
%   \item Inverse categories lie at the base of reversible programming. There are
%     a couple of ways to define these. They can be said to be restriction
%     categories. Restriction categories are any category with the added
%     restriction structure with some axioms. An example of a restriction
%     category is \textbf{PFn}, the category of partial function. Inverse
%     categories are restriction categories where every morphism is a partial
%     isomorphism.
%
%     A dagger category is a category equipped with a dagger endofunctor
%     $(-)^\dagger$ which distributes over composition and is involutive etc. An
%     example of a dagger category is \textbf{PInj} with $f^\dagger = f^{-1}$.
%     Inverse categories are dagger categories where for each $f$ we have $f
%     \circ f^{-1} \circ f = f$.
%
%     It can also more simply be defined as a category where for each morphism
%     $f$ there is a unique morphism $g$ with domain and image switched so that
%     $f \circ g \circ f = f$. This looks a lot like the definition based on
%     dagger categories.
%
%   \item Reversible effects have literature in inverse arrows and reversible
%     monads.~\cite{Heunen:2015} discusses reversible effects as a special type
%     of monad, while~\cite{Heunen:2018} works with inverse arrows.
%
% \end{itemize}

\section{Future Work}\label{sec:future_work}

% \subsection{Empty Type}
%
% To complete the description of algebraic data types, the empty type $0$ is
% often included. It contains no values on its own. Instead, the empty type
% unifies with any other type $\tau$ by lifting the set of values $\tau$ contains
% to its previous set of values in addition to the bottom element $\bot$. For
% example, the type $1$ is lifted to $1_\bot = \{\bot\} \cup 1$, and thus has two
% inhabitants:
%
% \begin{align*}
%   1_\bot = \{ \bot, () \}
% \end{align*}
%
% Because $\bot$ is added to any type through lifting, we can decide that $\bot$
% can be typed as anything we wish. Adding an empty type explicitly allows us to
% incorporate an additional avenue of partiality of functions, as we now can say
% that a function may evaluate to $\bot$ directly. Evaluation of $\bot$ aborts
% the program. Of course, $\bot$ also unifies with any expression and is its own
% closed term, so care should be taken to defer a case arm which contains a
% $\bot$ expression.

\subsection{Inductive First Match Policy Guarantee}

The static first match policy analysis presented in Sec.~\ref{sec:staticFMP} is
only possible by investigating the open form of the program text but it also
makes it somewhat limited. More specifically, it does not inspect the closed
form of expression of recursive types, meaning case-expressions operating on
recursively defined data structures may never generate static first match
policy guarantees, even though there is a large class of functions for which it
is strictly possible. Expressions of a recursive type require a more thorough
analysis. What could such a method look like?

Here we adhere to an \emph{inductive principle}, which we have to define
clearly. We introduce a \texttt{plus} function to develop the subject:

\begin{rfuncode}
  plus $n_0$: $\mu X . 1 + X$ $n_1$: $\mu X . 1 + X$ =
    case unroll [$\mu X . 1 + X$] $n_1$ of
      inl() => ($n_0$, $n_0$)
      inr($n'$) => let ($n_0'$, $n_1'$) = plus $n_0$ $n'$
                 in let succ = roll [$\mu X . 1 + X$] inr($n_1'$)
                 in ($n_0'$, succ)
\end{rfuncode}

As in the well-known structural or mathematical induction, we must identify
base cases for the induction hypothesis. A simple solution is to define these
as the branches in which a function application to the function which is being
evaluated does not occur. There might be multiple such branches without issue.
Note that this does not work well with mutually recursive functions. For
\texttt{plus} there is only one base case, the left arm of the main
case-expression.

Analogously the inductive step is defined on each branch which contains a
recursive call. For each recursive call the induction hypothesis says that,
granted the arguments given to the recursive call, eventually one of the base
cases will be hit. This is because any instance of the recursive type can only
be finitely often folded, giving a guarantee of the finiteness of the decreasing
chain. Though there is a catch which should be addressed: Inductive proofs are
only valid for \emph{strictly decreasing} chains of elements to ensure that the
recursion actually halts. For example, for \texttt{plus} we need to make sure
that $n' \prec n_1$. Should the chain not be strictly decreasing, we have that
the evaluation is non-terminating, and the function is not defined for this
input.

To tie it all together we need to show that the recursive call in the right arm
of the \texttt{plus} function does indeed result in the base case in the left
arm, allowing us to use the induction hypothesis to conclude that $n_0' =
n_1'$. If we are able to, we may directly treat the return value of the
recursive function call as an instance of the value which the base case
returns. We then continue evaluating the body in the inductive step. For
\texttt{plus} we say that:

\begin{rfuncode}
  $\dots$ => let ($n_0$, $n_0$) = plus $n_0$ $n'$
        in let succ = roll [$\mu X . 1 + X$] inr($n_0$)
        in ($n_0'$, succ)
\end{rfuncode}

And now we can see that the case-arms are provably disjoint, giving us a static
guarantee of the first match policy. However, generalizing an implementation
of inductively derived first match policy guarantees requires usage of proof
assistants, like \emph{Coq}~\cite{Bertot:2013}, on a growing number of
identified cases and is rather complex. Further discussion of this has
therefore been left for future work.

% \subsection{Compilation}
%
% There exist a number of reversible computer architectures and instruction sets.
% Compilation for imperative reversible languages has been shown. Interpretation
% and self interpretation for \rfun has been shown, but compilation for
% functional reversible languages has not been shown.
%
% Consider how to compile it an what the reversible instruction sets are. BobISA
% is one~\cite{Caroe:2012, Thomsen:2011}, PISA is another~\cite{Vieri:1995}.
%
% \subsection{Other?}
%
% Tail recursion and complexity of side condition. Something Michael is
% interested in.

\subsection{Possible Future Abstractions}

The following is a short, informal bullet point list the authors suggest as
additional possibilities for abstractions:

\begin{itemize}

  \item Built in syntactic sugar for a number of common data structures. Lists
    can be further simplified in the following (ubiquitous) way: encoding [] as
    \texttt{Nil}, the empty list, $(x:xs)$ as \rfuninl{Cons $x$ $xs$}, the head
    and tail of a list, and $[x_1, x_2]$ as \rfuninl{Cons $x_1$ (Cons $x_2$ Nil)},
    a literal list construction.

    A built-in \rfuninl{Char} data-type encoding characters can be wrapped in
    additional syntactic sugar to separate it from any regular variant. Then
    strings, simply encoded as lists over characters, can be represented as
    standard in quotation marks, encoding ``$c_1c_2c_3$'' as $[c_1, c_2, c_3]$

  \item A notion of modules, possibly inspired by the Haskell module
    system~\cite{Diatchki:2002}. Code reuse and modularisation are powerful
    concepts which are easily transferable to a reversible setting in the form
    of programs spread over multiple files, as this method is strictly related
    to exposure of code.

\end{itemize}


%-------------------------------------------------------------------------------
% CONCLUSION
%-------------------------------------------------------------------------------

\chapter{Conclusion}\label{sec:conclusion}

Although \rfunc is a continuation of the work that was started with \rfun, its
abstract syntax and evaluation semantics are quite different and include more
explicit primitive language constructs. However, we have also shown that
\rfunc can be made lighter via syntactic sugar to mimic other functional
languages.

We have presented a formal type system for \rfunc, including support for
recursive types through a fix point operator and polymorphic types via
parametric polymorphism. The type system is built on relevance typing, which is
sufficient for reversibility if we accept that functions may be partial.

Evaluation has been presented through a big step semantics. Most evaluation
rules were straightforward, but it was necessary to define a notion of leaves
and a relation for ``unification'' as machinery to describe the side condition
necessary to capture the first match policy.

An advantage offered by the type system is the ability to check the first match
policy statically. A static guarantee that the first match policy should hold
for a function will eliminate the run time overhead of case-expressions, often
leading to more efficient evaluation. By the program syntax, we can check for
orthogonality of inputs and the possible values of leaf expressions. By the
type system, we can argue for the finiteness of a function's domain and exhaust
the possible computational paths for a case-expression. Further, as was shown
in future work, we can apply an induction principle for recursive types.
However, it is difficult to detect exactly when this will yield a first match
policy guarantee.

We noted that, in contrast to many reversible programming languages, the syntax
of \rfunc does not easily support generation of inverse programs. This is not
problematic as the relational semantics do make it possible to inverse
interpret a program. We presented an inference system for inverse evaluation
and showed how inverse application of functions is achieved.

Finally, we have argued that it is possible to enhance the syntax of \rfunc
with high level constructs, which in turn have simple translation schemes back
to the core language. We have presented numerous examples: variants, type
classes (which, as an example, can be used to replace the duplication/equality
operator in the original \rfun language), top-level cases, records, arbitrarily
sized products and abbreviated let-expressions.

%-------------------------------------------------------------------------------
% BIBLIOGRAPHY
%-------------------------------------------------------------------------------

\bibliographystyle{splncs03}
\bibliography{references}

\end{document}
